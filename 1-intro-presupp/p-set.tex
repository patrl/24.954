\documentclass[cronos,landscape,paper=letter]{ling-handout}

\usepackage[margin=0.5in]{geometry}

\usepackage{fontawesome,calc,enumitem,parskip}

\lingset{
    aboveglftskip=0pt,
    belowglpreambleskip=0pt,
    belowpreambleskip=0pt,
    extraglskip=0pt,
    Everyex={\parskip=0pt},
}

\addbibresource[location=remote]{/home/patrl/repos/bibliography/elliott_mybib.bib}

% \renewcommand*{\sectionformat}{}
% \renewcommand*{\subsectionformat}{}
% \renewcommand*{\subsubsectionformat}{}

\title{Presupposition P-Set}
\subtitle{\texttt{24.954:} Pragmatics in Linguistic Theory}

\date{\today}

\author{Patrick~D. Elliott}

 \begin{document}

\maketitle

\section{Propositional logic}

\subsection{Exercise i}

Prove the following:

\[⊢ A ∨ B ⇔ ¬ (¬ A ∧ ¬ B)\]

\section{A compositionalisation of multi-dimensional semantics}

Let's get a little more precise about the compositional assumptions underlying a multi-dimensional semantics.

We can write \textit{pair types}, i.e., the type of a pair of \(a\)s and \(b\)s, where \(a\) is type \(\type{a}\) and \(b\) is type \(\type{b}\), as \(\type{a} ∗ \type{b}\). The type of a pair of propositions is therefore: \(\type{st ∗ st}\).

Presuppositional expressions are of type \(\type{a} ∗ \type{st}\) (where \(\type{a}\) is a variable over types).

In the lecture we defined \enquote{quit smoking} as a function from an individual to a presuppositional proposition, i.e.,

\pex
\a \(\eval{quit smoking} = λ x . \displaystyle\frac{x\ml{ used to smoke}}{x\ml{ doesn't smoke now}}\)
\a \(\eval{quit smoking} ∷ \type{⟨e,st ∗ st⟩}\)
\xe

\subsection{Exercise i}

Define a type-shifter that takes any \textit{uni-}dimensional meaning, and returns a \textit{trivially} presuppositional meaning (i.e., a multi-dimensional meaning which will not give rise to a contentful pragmatic presupposition). Give both (a) the entry for the type-shifter, and (b) its type.

\subsection{Exercise ii}

We can assign a definite description the following multi-dimensional meaning:

\ex
\label{theDog}\(\eval{the dog} = \displaystyle\frac{λ w . ∃!x[\ml{dog}_{w} x]}{λ P . λ w . ∃x[\ml{dog}_{w} x ∧ P x w]}\)
\xe

What is the type of (\ref{theDog})?

\subsection{Exercise iii}

Recall the rule for composing multi-dimensional meanings:

\ex
\(\eval{\begin{array}{c}\begin{forest}
      [{...}
        [{X}]
        [{Y}]
      ]
    \end{forest}\end{array}} = \displaystyle\frac{ℙ \eval{X} ∧ ℙ \eval{Y}}{\ml{FA} (𝔸 \eval{X}) (𝔸 \eval{Y})}\)
\xe

Point out what goes wrong when attempting to compute the meaning of the following:

\ex
\begin{forest}
  [{...}
    [{...} [{the dog},roof]]
    [{...} [{quit smoking},roof]]
  ]
\end{forest}
\xe

Suggest a fix.

\subsection{Exercise iv}

Instead of formalising multi-dimensional semantics using \textit{pairs}, we could also have gone a different route. We'll explore that route in this exercise.

Just for this exercise, let's assume that \(\eval{.}\) always returns an ordinary, \textit{uni-}dimensional at-issue meaning.

Let's supply an additional interpretation function \(⦅.⦆\) that returns the \textit{presupposition} of an expression.\footnote{If you're familiar with Roothian focus semantics, this should be familiar.}

\ex
\(\eval{Paul quit smoking} = \ml{Paul doesn't smoke now}\)
\xe

\ex~
\(⦅\text{Paul quit smoking}⦆ = \ml{Paul used to smoke}\)
\xe

Fill in the definitions of the following composition rules:

\begin{multicols}{2}

\ex~
\(\eval{\begin{array}{c}\begin{forest}
    [{...}
      [{X}]
      [{Y}]
    ]
  \end{forest}\end{array}} = ???\)
\xe

\columnbreak

\ex~
\(\left⦅\begin{array}{c}
          \begin{forest}
            [{...}
              [{X}]
              [{Y}]
            ]
          \end{forest}
          \end{array}\right⦆ = ???\)
\xe

\end{multicols}

\subsection{Exercise v}

Recall that a multi-dimensional theory of presupposition faces the \textit{binding problem}.

Suppose that we assign the presuppositional predicate \enquote{quit smoking} the following entry:

\ex
\(\eval{quit smoking} = λ x . \displaystyle\frac{x \ml{used to smoke}}{x \ml{used to smoke} ∧ x \ml{doesn't smoke now}}\)
\xe

Does the binding problem still arise? Assume that \textit{someone} has the following meaning, in order to bootstrap compositionality:

\ex
\(\eval{someone} ≔ λ P . \displaystyle\frac{∃ x[ℙ (P x)]}{∃ x[𝔸 (P x)]}\)
\xe

\printbibliography

\begin{appendix}

  \section{Solutions}

  \subsection{Compositionalisation}

  \subsubsection{Exercise iii}

  The most obvious fix is to redefine \enquote{the dog} as a function from an individual to a presuppositional proposition, to a presuppositional proposition.

  \ex
  \(\displaystyle \eval{the dog} ≔ λ P . \frac{\ml{there is a unique dog} ∧ ∃x[\ml{dog} x ∧ ℙ (P x)]}{∃ x[\ml{dog} x ∧ 𝔸 (P x)]}\)
  \xe

  \ex~
  type: \(\type{⟨⟨e,st ∗ st⟩, st ∗ st⟩}\)
  \xe

\end{appendix}

\end{document}

%%% Local Variables:
%%% TeX-engine: default
%%% TeX-master: t
%%% End:
