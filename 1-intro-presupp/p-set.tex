\documentclass[cronos,landscape,paper=letter]{ling-handout}

\usepackage[margin=0.5in]{geometry}

\usepackage{fontawesome,calc,enumitem,parskip}

\usepackage{tcolorbox}

\lingset{
    aboveglftskip=0pt,
    belowglpreambleskip=0pt,
    belowpreambleskip=0pt,
    extraglskip=0pt,
    Everyex={\parskip=0pt},
}

\addbibresource[location=remote]{/home/patrl/repos/bibliography/elliott_mybib.bib}

% \renewcommand*{\sectionformat}{}
% \renewcommand*{\subsectionformat}{}
% \renewcommand*{\subsubsectionformat}{}

\title{Presupposition P-Set}
\subtitle{\texttt{24.954:} Pragmatics in Linguistic Theory}

\date{\today}

\author{Patrick~D. Elliott}

 \begin{document}

\maketitle

\section{From \textit{the law of non-contradiction} to \textit{the law of the excluded middle}}

One logical law that is easy to accept is \textit{the law of non-contradiction}, which can be expressed as the following formula of propositional logic:

\[¬ (p ∧ ¬ p)\]

This tells us that no statement can be both true and false. Seems pretty uncontroversial, right?

The exercise here is to show (informally) how \textit{the law of the excluded middle} -- every statement is either true or false -- follows from \textit{the law of non-contradiction}.

Hint: first, render the law of the excluded middle into a formula of propositional logic, and go from there (using the truth-tables for the logical connectives).

% Show (informally) that the following holds:
% \[¬(p∨q)≡¬p∧¬q.\]

\section{Projection in conditional statements}

Treating \textit{if...then...} as a two place sentential operator, (a) state the generalisation concerning how presuppositions project in conditional statements, and (b) write a multi-dimensional lexical entry for \textit{if...then...} which captures the generalisation. Use the entries for conjunction and disjunction as a guideline.

\section{More on compositionality}

Define an operator (call it ({\bind})) which will allow us to compose something of type \(\displaystyle\frac{\type{st}}{a}\) with something of type \(\left⟨a,\displaystyle\frac{\type{st}}{⟨a,b⟩}\right⟩\) (where \(a\) and \(b\) could be any type).

\ex
\(\displaystyle\frac{\ml{st}}{a} \bind m = ???\)\hfill\((\bind) ∷ \displaystyle\left\langle\frac{\ml{st}}{a},\left\langle\left\langle a, \frac{\ml{st}}{b}\right\rangle,\frac{\ml{st}}{b}\right\rangle\right\rangle\)
\xe

Tailor the definition of \(\bind\) such that it makes the right predictions for the presuppositions of the following sentence:

\ex
she\(_{1}\) quit vaping.
\xe

Assume the following semantics for the pronoun:\footnote{Since relativising meanings to assignments hasn't been relevant, we omitted the \(g\) parameter in the handout.}

\ex
\(\eval[g]{she\sub{1}} = \displaystyle\frac{g(1)\ml{ identifies female}}{g(1)}\)
\xe

Give the LF and compute the meaning, using the operator you have just defined.


% \subsection{Exercise iv}

% Instead of formalising multi-dimensional semantics using \textit{pairs}, we could also have gone a different route. We'll explore that route in this exercise.

% Just for this exercise, let's assume that \(\eval{.}\) always returns an ordinary, \textit{uni-}dimensional at-issue meaning.

% Let's supply an additional interpretation function \(⦅.⦆\) that returns the \textit{presupposition} of an expression.\footnote{If you're familiar with Roothian focus semantics, this should be familiar.}

% \ex
% \(\eval{Paul quit smoking} = \ml{Paul doesn't smoke now}\)
% \xe

% \ex~
% \(⦅\text{Paul quit smoking}⦆ = \ml{Paul used to smoke}\)
% \xe

% Fill in the definitions of the following composition rules:

% \begin{multicols}{2}

% \ex~
% \(\eval{\begin{array}{c}\begin{forest}
%     [{...}
%       [{X}]
%       [{Y}]
%     ]
%   \end{forest}\end{array}} = ???\)
% \xe

% \columnbreak

% \ex~
% \(\left⦅\begin{array}{c}
%           \begin{forest}
%             [{...}
%               [{X}]
%               [{Y}]
%             ]
%           \end{forest}
%           \end{array}\right⦆ = ???\)
% \xe

% \end{multicols}

\section{More on the binding problem}

Recall that a multi-dimensional theory of presupposition faces the \textit{binding problem}.

Suppose that we assign the presuppositional predicate \enquote{quit smoking} the following entry:

\ex
\(\eval{quit smoking} = λ x . \displaystyle\frac{x \ml{used to smoke}}{x \ml{used to smoke} ∧ x \ml{doesn't smoke now}}\)
\xe

Does the binding problem still arise? Assume that \textit{someone} has the following meaning, in order to bootstrap compositionality:

\ex
\(\eval{someone} ≔ λ P . \displaystyle\frac{∃ x[ℙ (P x)]}{∃ x[𝔸 (P x)]}\)
\xe

\section{Bonus question on disjunction}

Try to come up with counter-examples to the following generalisation (subscript Greek letters are presuppositions):

\ex
 \textbf{Disjunction}\\If A\(_π\), and B\(_ρ\), then a sentence of the form \enquote{A or B} presupposes \(π\), and unless \enquote{not A} entails \(ρ\), also presupposes \(ρ\).
 \xe


% \printbibliography

% \begin{appendix}

%   \section{Solutions}

%   \subsection{Compositionalisation}

%   \subsubsection{Exercise iii}

%   The most obvious fix is to redefine \enquote{the dog} as a function from an individual to a presuppositional proposition, to a presuppositional proposition.

%   \ex
%   \(\displaystyle \eval{the dog} ≔ λ P . \frac{\ml{there is a unique dog} ∧ ∃x[\ml{dog} x ∧ ℙ (P x)]}{∃ x[\ml{dog} x ∧ 𝔸 (P x)]}\)
%   \xe

%   \ex~
%   type: \(\type{⟨⟨e,st ∗ st⟩, st ∗ st⟩}\)
%   \xe

% \end{appendix}

\end{document}

%%% Local Variables:
%%% TeX-engine: default
%%% TeX-master: t
%%% End:
