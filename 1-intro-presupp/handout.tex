\documentclass[cronos,landscape,paper=letter]{ling-handout}

\usepackage[margin=1in]{geometry}

\usepackage{fontawesome,calc,enumitem,parskip}

\usepackage{tcolorbox}

\lingset{
  belowexskip=0pt,
  aboveglftskip=0pt,
  belowglpreambleskip=0pt,
  belowpreambleskip=0pt,
  interpartskip=0pt,
  extraglskip=0pt,
  Everyex={\parskip=0pt},
}

\addbibresource[location=remote]{/home/patrl/repos/bibliography/elliott_mybib.bib}

% \renewcommand*{\sectionformat}{}
% \renewcommand*{\subsectionformat}{}
% \renewcommand*{\subsubsectionformat}{}

\title{\textit{Presupposition} primer}
\subtitle{\texttt{24.954:} Pragmatics in Linguistic Theory}

\date{\today}

\author{Patrick~D. Elliott}

 \begin{document}

\maketitle

\section{Overview}

\textbf{Roadmap for today}

Today we'll give a wide-ranging overview of the major problems, empirical and conceptual, in the literature on \textit{presupposition} -- a highly significant frontier on the border between semantics and pragmatics. We'll primarily cover:

\begin{itemize}

    \item The empirical properties of presupposition.

  \item A multi-dimensional theory of presupposition, and its failings.

\end{itemize}

\textbf{Homework}

We'll post the first p-set, based on material from this class, on the stellar site today. Please submit your solutions by the Thursday before next class (11:30pm).

\textbf{Readings}

Optional (but strongly recommended) readings for this week:

\begin{itemize}

  \item Beaver \& Guerts (2011) Presupposition. In \textit{Stanford Encyclopaedia of Philosophy}.\\
    \url{https://plato.stanford.edu/entries/presupposition/}

  \item Chapter 5 of Kadmon (1990) \textit{Formal Pragmatics}.

\end{itemize}

\section{Introduction}

Let's start by being maximally non-commital: a \enquote{presupposition}\footnote{When we say \textit{presupposition}, we typically mean \textit{semantic presupposition}, unless otherwise noted.} is a kind of \textit{inference} that sentences in natural languages \textit{may} convey.

\pex\label{basics}
\a Paul quit vaping.\\
presupp: \textit{Paul used to vape}
\a The mayor of Oakland waterskis.\\
presupp: \textit{Oakland has a mayor}
\a I re-read \textit{The Lord of the Rings}.\\
presupp: \textit{I've read the Lord of the Rings}
\a Uli forgot to submit the paperwork.\\
presupp: \textit{there was paperwork that Uli was supposed to submit}
\a None of my students failed her midterm.\\
presupp: \textit{Each of my students prefer \enquote{she/her} pronouns}
\a Sam walks her dog every morning.\\
presupp: \textit{Sam has a dog}
\a It was at the pub that Rob had a revelation.\\
presupp: \textit{Rob had a revelation}
\xe

Issues to flag up immediately:

\begin{itemize}

  \item Of all the inferences that a sentence conveys, how do we know which ones are \textit{presupposed}?

  \item Of all the inferences that a sentence conveys, \textit{why} are just the ones that are presupposed, presupposed?

\end{itemize}

\section{Natural language and the law of the excluded middle}

Eubulides of Miletus' ($±$405-330 BC) noted that, if the following argument is sound, then we should conclude that \textit{everyone is a cuckold}. Why doesn't it go through?\footnote{We can thank Seuren (2005: p.\,89) for pointing out the relevance of Eubulides' paradoxes to contemporary semantic thinking.}

\pex
\a Major: What you haven't lost you still have.
\a\label{horns-minor}Minor: You have not lost your horns.
\a Ergo: You still have your horns.
\xe

\begin{tcolorbox}
  Exercise
  \tcblower
  Try to render this argument into first order logic. Is it valid? What's going wrong here?
\end{tcolorbox}

A non-trivial fact about natural language semantics: sentences don't obey the \textit{law of the excluded middle}, i.e., sentences in natural language can be neither true nor false.

Another non-trivial, and closely related fact about natural language semantics: sentence meanings are \textit{multi-dimensional}. In other words, sentences may convey qualitatively different kinds of meaning at the same time.

We can think of \textit{presupposition} as a dimension of meaning characterised by the following two properties:

\begin{itemize}

  \item backgroundedness

  \item projection

\end{itemize}

\section{Backgroundedness}

From an utterance of (\ref{paulvape}), we can infer (at least) two things: (\ref{paulvapea}) and (\ref{paulvapeb}).

\pex\label{paulvape}
Peter is talking about \textit{Eubulides} again
\a\label{paulvapea}\(⇝\) \textsf{Peter is talking about Eubulides now} \hfill at-issue
\a\label{paulvapeb}\(⇝\) \textsf{Peter has talked about Eubulides before} \hfill presupposition
\xe

There is an intuitive sense in which (\ref{paulvapea}) is the subject matter of the sentence -- in uttering (\ref{paulvape}), the speaker aims to communicate that (\ref{paulvapea}). The presupposition is different -- the speaker \textit{takes for granted} that the information conveyed by the presupposition is already known (in some sense to be made precise).

We can probe the pragmatic status of a given inference using questions -- the at-issue component of a sentence meaning can answer a question, whereas the presupposition cannot, as illustrated by the contrast between (\ref{q1}) and (\ref{q2}).

\ex\label{q1}
Which Greek philosopher is Peter talking about now?\\
\ljudge{\cmark} Peter is talking about \textit{Eubulides} again.
\xe

\ex~\label{q2}
Which Greek philosopher has Peter talked about before?\\
\ljudge{\#}Peter is talking about \textit{Eubulides} again.
\xe

\begin{tcolorbox}
Exercise
\tcblower
Run this diagnostic on the data in (\ref{basics})
\end{tcolorbox}

A famous diagnostic which diagnoses the backgroundedness of presuppositions is \citeauthor{fintel2004}'s (\citeyear{fintel2004}) \textit{Hey, wait a minute!} test (based on Shanon 1976).

\pex
The mathematician who proved Goldbach's conjecture is a woman.
\a Hey, wait a minute. I had no idea that someone proved Goldbach's conjecture.
\a\ljudge{\#}Hey, wait a minute. I had no idea that that was a woman.
\xe

\enquote{Hey, wait a minute} can preface a complaint about the truth \textit{presupposition} being taken for granted, but not the at-issue component of the meaning.

\subsection{Pragmatic vs. semantic presupposition}

If I say the following to you...

\ex
\label{german}Peter redet wieder von \textit{Eubulides}.
\xe

...I \textit{pragmatically} presuppose (at least) the following:

\begin{itemize}

  \item you understand German.

  \item You can hear (or perhaps lip-read).

\end{itemize}

This information however is (probably!) not encoded in the semantics of the NL expressions. Nevertheless, we can draw these inferences from an utterance of (\ref{german}), and furthermore, they are \textit{backgrounded}, and \textit{project} (in the sense we introduce in the subsequent section).

Semantic presupposition relates to pragmatic presupposition, in the sense that semantic presupposition usually (but not always) gives rise to a pragmatic presupposition.

Later on in the class, we'll make precise exactly how this comes about, within a particular theory of presupposition.

\section{Projection}

As well as having a different \textit{pragmatic} status (wrt. backgroundedness), the inferences in (\ref{paulvapea_rep}) and (\ref{paulvapeb_rep}) behave differently in embedded contexts.

\pex\label{paulvape_rep}
Peter is talking about \textit{Eubulides} again
\a\label{paulvapea_rep}\(⇝\) \textsf{Peter is talking about Eubulides now} \hfill at-issue
\a\label{paulvapeb_rep}\(⇝\) \textsf{Peter has talked about Eubulides before} \hfill presupposition
\xe

Consider what happens when we turn (\ref{paulvape_rep}) into a polar question:

\pex
Is Peter talking about \textit{Eubulides} again?
\a \(⇝ ? (\ml{Peter is talking about Eubulides now})\)
\a \(⇝ \ml{Peter has talked about Eubulides before}\)
\xe

The inference in (\ref{paulvapea_rep}) is \textit{questioned}, whereas the inference in (\ref{paulvapeb_rep}) is unaffected. We see the same pattern in other \textit{non-veridical contexts}:

\ex
Peter isn't talking about \textit{Eubulides} again.
\xe

\ex~
There's no way that Peter is talking about \textit{Eubulides} again.
\xe

\ex~
Peter will never talk about \textit{Eubulides} again.
\xe

\ex~
Peter might be talking about \textit{Eubulides} again.
\xe

\ex~
Peter should talking about \textit{Eubulides} again.
\xe

\begin{tcolorbox}
\textbf{Q:} why can't veridical contexts be used to detect presuppositions?
\end{tcolorbox}

We'll see more cases of \textit{presupposition projection} later.

\subsection{Failed projection (or: \textit{Local Accommodation})}

It seems like projection cannot always be guaranteed.

\ex
\label{kf}The king of France isn't bald...there is no king of France!!
\xe

If the presupposition of the first sentence in (\ref{kf}) were to project, the second sentence would be a contradiction.


Another example from \cite{beaverZeevat2007}:

\ex\label{accom}
Context: \textit{Our heroine has landed herself in a difficult spot. From all sides dangerous criminals are approaching.}\\
\textsc{The Heroine:} I knew they would show no mercy.
\xe

\textit{to know \(P\)} typically presupposes \(P\), but (\ref{accom}) is felicitous in a context where this is the first time the heroine has informed us of the treatment she expects at the hands of the villains.

In this instance the semantic presupposition fails to translate into a pragmatic presupposition.

More examples:

\pex
\a I don't have a dog.
\a So at least you don't have to \textsc{walk} your dog.\hfill \cite[p.\,145]{kadmon2001}
\xe

\ex~
The king thinks it's the knave that stole the tarts, but he's obviously gone mad, since there is no knave here.\hfill \cite{beaverGuerts2014}
\xe

\ex~
Have you quit smoking recently?
\xe

\ex~
\label{spouse}If your spouse works at MIT, then you will receive a discount.
\xe

This last example is especially interesting. If the presupposition were accommodated globally, we would predict that it should mean: \textit{You have a spouse, and if they work at MIT, then you will receive a discount}.

Rather, what we infer from (\ref{spouse}) is: \textit{if you have a spouse and they work at MIT, then you will receive a discount}.

In the literature, such cases are referred to as \textit{local accommodation}, since the presupposition is seemingly accommodated at some subsentential level, as diagnosed by its interaction with some other operator in the sentence.

\subsection{Weakened projection}

\ex\label{weakened}Patrick hopes that Peter will talk about \textit{Eubulides} again.
\xe

The complement of \enquote{hope} (\enquote{Peter will talk about Eubulides again}), as we've seen, presupposes: \textsf{Peter has talked about Eubulides before}.

\begin{tcolorbox}
\textbf{Q} what does (\ref{weakened}) presuppose?
\end{tcolorbox}

% What does (\ref{weakened}) presuppose globally?
% \begin{itemize}

%     \item \(⇝ \textsf{Peter has talked about Eubulides before.}\)

%     \item \(⇝ \textsf{Patrick believes that Peter has talked about Eubulides before.}\)

% \end{itemize}

% It has been observed that presuppositions give rise to \textit{weakened} projection, when embedded under attitude verbs (\citealt{heim1992}). We'll talk more about this in session 3!


\section{Projective meaning beyond presupposition}

Presupposition can be difficult to distinguish from \textit{conventional implicature}, which also displays both \textit{backgroundedness} and \textit{projectivity}.

\textit{Expressives/supplements} (as in (\ref{exp})) and \textit{appositives} are often taken to be two representative kinds of conventional implicature, starting from \citet{potts2005} (see, e.g., \citet{mccready2010a,gutzmann_use-conditional_2015} for other kinds).

\ex\label{exp}
This \textit{fucking} class is three hours long!\\
\(⇝ ☹ (\ml{this class is three hours long})\)\hfill\textit{expressive}
\xe

\ex\label{app}
The class, which lasted for three hours, was tedious.\\
\(⇝ \ml{the class lasted for three hours}\)\hfill\textit{appositive}
\xe

\textbf{Appositives vs.\,presuppositions}

Despite both being backgrounded, presuppositions typically express \textit{old} information, whereas appositives convey \textit{new} information.

Furthermore, whereas presuppositions can sometimes interact with other operators (cf. our discussion of \textit{weakened projection}, appositives always project all the way out).

\ex
Patrick hopes that Peter is talking about Eubulides again.\\
\(⇝ \ml{Patrick believes that Peter has talked about Eubulides before}\)\\
\phantom{,}\hfill weakened projection
\xe

\ex~
Patrick hopes that Peter is talking about Eubulides, who is a Greek philosopher.\\
\(⇝ \ml{Eubulides is a Greek philosopher}\)\hfill No weakened projection
\xe

\textbf{Expressives vs.\,presuppositions}

Expressives are harder to distinguish from presuppositions, although they also tend to display stubborn projection behaviour:

\ex\label{expproj}
Patrick hopes that Peter isn't talking about that fucking Greek philosopher.\\
\(⇝ ☹ \ml{that Greek philosopher}\)
\xe

Nevertheless, some authors argue that expressives are in fact presuppositional, although this is still an area of ongoing debate (\citealt{schlenker2007,sauerland2007,lasersohn2007}).

\section{Triggering presupposition}

In many cases, the presence of a presupposition is traceable to a certain NL expression or construction. See, e.g., the contrast in (\ref{again}).

\pex\label{again}
\a Paul is vaping \textbf{again}.\hfill\textit{presupposes that Paul vaped before}
\a Paul is vaping.\hfill\textit{presuppositionless}
\xe

An \textit{it-cleft}, i.e., \textit{it was \(x\) that \(V-\)ed} presupposes that \textit{someone \(V-\)ed}.

\pex
\a It was \textsc{Paul} that vaped.
\a It wasn't \textsc{Paul} that vaped.
\xe

Factive predicates such as \textit{know} presuppose the truth of their complement.

\pex
\a Hubert knows that Paul vapes
\a Hubert doesn't know that Paul vapes.
\xe

Certain attitude predicates induce more idiosyncratic presuppositions. \textit{pretend} gives rise to the presupposition that it's complement is false.

\pex
\a Patrick pretends that he knows about syntax.\\
\(⇝ \ml{Patrick doesn't know about syntax}\)
\a Patrick never pretends that he knows about syntax.\\
\(⇝ \ml{Patrick doesn't know about syntax}\)
\xe

\textit{forget to V} and \textit{pretend to V} both give rise to the presupposition that they subject was \textit{supposed to V}.

\pex
\a Andy forgot to go to bed.\hfill\(⇝ \ml{Andy should go to bed}\)
\a Andy didn't forget to go to bed.\hfill\(⇝ \ml{Andy should go to bed}\)
\xe

We call such expressions and constructions \textbf{presupposition triggers}.

Expressions with similar at-issue meanings cross-linguistically tend to trigger the same presuppositions. See, e.g., \citet{levinsonAnnamalai1992} for a detailed study comparing English and Tamil. Despite not being related, in both languages, exactly the same components of meaning are expressed as presuppositions vs. at-issue entailments.

This suggests that we want a unified theory of which inferences associated with NL expressions/constructions end up being presupposed, based on their at-issue meanings.

This is known as the \textbf{triggering problem} of presupposition.

See \citet{schlenker2019} for an independent argument from pro-speech gestures that a triggering algorithm is independently necessary.

One major problem for any triggering algorithm is the existence of expressions whose meanings are apparently \textit{purely presuppositional}:

\begin{itemize}

    \item Triggers such as \textit{also}, \textit{even}, and \textit{again}.

    \item Phi-features, according to some analyses (see, e.g., \citealt{sauerland2013}).

\end{itemize}

Furthermore, there are pairs of expressions such as \textit{come} and \textit{go}, which seem to express the same at-issue meaning, and yet differ in their presuppositions.

We'll come back to the triggering problem later in the course. For now, we'll focus on how to represent presupposition in compositional semantics.

\section{A multi-dimensional theory of presupposition}

In this section, I'll spell out explicitly a \textit{multi-dimensional} account of presupposition, so that we can see its shortcomings clearly.\footnote{This is essentially a more compositionally explicit version of \citet{karttunenPeters1979}. It most closely mirrors \citeauthor{asudehGiorgolo2012}'s (\citeyear{asudehGiorgolo2012}) analysis of conventional implicature.}

Each natural language expression conveys two kinds of meanings: (a) an \textit{at-issue} meaning, and (b) a \textit{presupposition}.

We'll write this using \citeauthor{sauerland_2008}'s (\citeyear{sauerland_2008}) fraction notation; the presupposition goes on the top, and the at-issue meaning goes on the bottom. We'll assume that presuppositions are propositions.

\[\displaystyle\frac{\text{\it presupposition}}{\text{\it assertion}}\]

\[\eval{Paul quit vaping}= \displaystyle\frac{\ml{Paul used to vape}}{\ml{Paul doesn't vape now}}\]

Let's also define helper functions to retrieve the presupposition and assertion:

\begin{multicols}{2}
\ex
\(𝔸 \displaystyle\frac{p}{a} = a\)
\xe
\columnbreak
\ex
\(ℙ \displaystyle\frac{p}{a} = p\)
\xe
\end{multicols}

For us, presuppositional meanings are \textit{pairs} of ordinary meanings, and presupposed propositions. We're using the fraction notation here as sugar for a pair of semantic objects. We could, for instance, also write the following, but it's a little more difficult to distinguish the presupposition and at-issue meaning at a glance.

\[\eval{Paul quit vaping} = (\ml{Paul used to vape}, \ml{Paul doesn't vape now})\]

We can also use fraction notation as a sugar for \textit{pair-types}: let's treat presuppositional meanings as type \(\displaystyle\frac{\type{st}}{a}\).

We're essentially building the idea that presuppositions are a distinct dimension of meaning directly into our theory. Since presuppositions have a distinct pragmatic status, and furthermore display a distinct behaviour with respect to other operators in the sentence (projection), this seems like a reasonable starting point.

\subsection{Bridging between semantics and pragmatics}

Let's characterise more formally what it means for a proposition to be pragmatically presupposes (after \citealt{stalnaker1976}).

\pex
Agents \(a_{1} … a_{n}\) pragmatically presuppose \(p\) iff the following hold:
\a Each \(a_{i}\) believes \(p\).
\a Each \(a_{i}\) believes that each \(a_{j}\) believes that \(p\).
\a Each \(a_{i}\) believes that each \(a_{j}\) believes that each \(a_{k}\) believes that \(p\)\\
\ldots
\xe

We can now bridge the gap between semantic presupposition and pragmatic presupposition by stating a felicity condition on utterances sensitive to the semantic presupposition carried by a sentence.

\ex
An utterance of sentence \(S\) by agents \(a_{1}…a_{n}\) is infelicitous unless \(a_{1}…a_{n}\) pragmatically presuppose \(ℙ \eval{S}\) (i.e., the semantic presupposition of \(S\))
\xe

The prediction now is that an utterance of...

\ex
\label{quit}Paul quit vaping
\xe

...is only felicitous if it is \textit{pragmatically presupposed} (informally: \enquote{taken for granted}) by both speaker and hearer that Paul used to vape.

By separating out presupposition as a separate dimension of meaning, we have made it easy to \textit{capture} its special pragmatic status (although one may of course question how explanatory such an account can ultimately be).

\subsection{Compositionality}

We'll assume that any uni-dimensional (i.e., non-presuppositional) meaning can be lifted into a multidimensional meaning with a trivial presupposition, via an operator \(π\).

\[\eval{Paul doesn't vape}^{\pi} = \displaystyle\frac{⊤}{\ml{Paul doesn't vape}}\]

Composition proceeds by (a) doing \textit{function application} (FA) in the assertive dimension, and (b) \textit{conjunction} in the presuppositional dimension. We can define an operation of \textit{multi-dimensional function application} (MA) in order to formalise this.\footnote{Here we're using \(\ml{FA}\) to stand in for \textit{bi-directional} (or \enquote{overloaded}) function application.}

\[\ml{MA} \displaystyle\frac{p}{x} \frac{q}{y} ≔ \frac{p ∧ q}{\ml{FA} x y}\]

      We can treat \textit{presupposition triggers} as functions from uni-dimensional individuals to multi-dimensional meanings. This gives us sub-sentential compositionality.

      \ex
      \(\eval{quit vaping} = λ x . \displaystyle\frac{x \ml{used to vape}}{x \ml{doesn't vape now}}\)
      \xe

      \ex~
      \begin{forest}
        [{\(\displaystyle\frac{\ml{Paul used to vape}}{\ml{Paul doesn't vape now}}\)\\\(\ml{FA}\)}
          [{Paul}]
          [{...} [{quit vaping},roof]]
        ]
      \end{forest}
      \xe

      \begin{tcolorbox}
        Exercise
        \tcblower
        We might start with the assumption that a presuppositional DP, such as \textit{the Frenchman}, has the following semantics:

                \[\displaystyle\frac{∃!x[\ml{frenchman} x]}{\(λ P . ∃x[\ml{frenchman} x ∧ P x]\)}\]

                \textit{the Frenchman quit vaping} fails to compose -- demonstrate this. How do we fix it?

                Hint: consider changing the meaning of the DP.
      
        \end{tcolorbox}

\subsection{Projection}

\subsubsection{Negation}

NL negation negates the assertive component of a sentence meaning, yet leaves the presupposition unaffected (to use \citeauthor{karttunenPeters1979}'s terminology, it is a \enquote{hole}).

\pex
\a Paul quit vaping.\hfill\textit{presupposes that Paul vaped before}
\a Paul didn't quit vaping.\hfill\textit{Presupposes that Paul vaped before}
\xe

\ex
\textbf{Negation}\\If A\(_π\), then a sentence of the form \enquote{not A} presupposes \(π\).
\xe

We can capture this behaviour simply by assigning it a trivially multi-dimensional meaning:

\ex
\(\eval{not} = \displaystyle\frac{⊤}{\ml{not}}\)
\xe

\ex~
\begin{forest}
  [{\(\displaystyle\frac{\ml{paul used to vape}}{\ml{paul vapes now}}\)\\\(\ml{MA}\)}
    [{\(\displaystyle\frac{⊤}{\ml{not}}\)}]
    [{$\displaystyle\frac{\ml{paul used to vape}}{\ml{paul doesn't vape now}}$} [{Paul quit vaping},roof]]
  ]
\end{forest}
\xe

\subsubsection{Conjunction}

How do presuppositions project in conjunctive sentences?

\ex
Sam visited Rome again and Ka visited Venice.\\
presupp.: \(\ml{Sam has visited Rome before}\)
\xe

\ex~
Sam visited Rome and Ka visited Venice again.\\
presupp.: \(\ml{Ka has visited Venice before}\)
\xe

\ex
Sam and Ka visited Rome and Venice last Summer, and Ka visited Venice again.\\
\textit{presuppositionless}
\xe

\ex~
Ka visited Venice again, and Sam and Ka visited Rome and Venice last Summer.\\
presupp.: \(\ml{Ka has visited Venice before}\)
\xe

As famously observed by \citeauthor{karttunenPeters1979}, presuppositions embedded in a conjunctive sentence do not always project all the way out.

\ex
\textbf{Conjunction}\\If A\(_π\), and B\(_ρ\), then a sentence of the form \enquote{A and B} presupposes \(π\), and unless A entails \(ρ\), also presupposes \(ρ\)
\xe

We'll adopt the following entry for sentential conjunction (assuming a left-branching structure for conjunctive sentences).

\ex
\(\displaystyle\eval{and} ≔ λ \frac{q'}{q} . λ \frac{p'}{p} . \frac{p' ∧ (p ⇒ q')}{p ∧ q}\)
\xe

\ex~
\begin{forest}
    [{\(\overbrace{\displaystyle\frac{\ml{if s and k visited r and v last summer then k has visited v before}}{\ml{s and k visited r and v last summer and k visited v}}}^{\top}\)}
      [{\(\displaystyle\frac{⊤}{\ml{s and k visited r and v last summer}}\)\\\(\pi\)} [{Sam and Ka visited\\Rome and Venice last Summer},roof]]
      [{...}
        [{and}]
        [{\(\displaystyle\(\frac{\ml{k has visited v before}}{\ml{k visited v}}\)\)} [{Ka visited Venice again},roof]]
      ]
    ]
  \end{forest}
\xe

\textbf{A preview of the proviso problem}

\ex
\label{proviso}Mary is pregnant and her brother is happy.
\xe

\ex~
\(ℙ (\eval{her brother is happy}) = \ml{Mary has a brother}\)
\xe

Our entry for \enquote{and} predicts the following:

\ex
\(\displaystyle\frac{\ml{if Mary is pregnant then Mary has a brother}}{\ml{Mary is pregnant and Mary's brother is happy}}\)
\xe

This seems too weak! (\ref{proviso}) is only felicitous in a context where the common ground entails that Mary has a brother.

Other theories of presupposition projection inherit the problem of assigning certain sentences weak, conditional presuppositions; this is known as the \textit{proviso problem}. We'll come back to this in week three, after introducing the satisfaction theory of presupposition projection (i.e., dynamic semantics).

\subsection{Disjunction}

\ex
Either Sam visited London again, or Ka visited Venice.\\
presupp.: \(\ml{Sam has visited London before}\)
\xe

\ex~
Either Sam visited London, or Ka visited Venice again.\\
presupp.: \(\ml{Ka has visited Venice before}\)
\xe

\ex~
Either Ka has never visited Venice, or she visited Venice again.\\
\textit{presuppositionless}
\xe

\ex~
Either Ka visited Venice again, or she has never visited Venice.\\
???
\xe

\ex~
\(\eval{or} ≔ \displaystyle λ\frac{q'}{q} . λ\frac{p'}{p} . \frac{p' ∧ ¬p ⇒ q'}{p ∨ q}\)
\xe

\ex~
 \textbf{Disjunction}\\If A\(_π\), and B presupposes B\(_ρ\), then a sentence of the form \enquote{A or B} presupposes \(π\), and unless \enquote{not A} entails \(ρ\), also presupposes \(ρ\).
 \xe


\ex~
\begin{forest}
  [{\(\overbrace{\displaystyle\frac{\ml{if k has visited v before she has visited v before}}{\ml{Ka hasn't visited v before or she visited v}}}^{\top}\)}
   [{\(\displaystyle\frac{\top}{\ml{k hasn't visited Venice before}}\)\\\(\pi\)} [{Ka has never visited Venice},roof]]
   [{...}
     [{or}]
     [{\(\displaystyle\frac{\ml{k visited v before}}{\ml{k visited v}}\)} [{Ka visited Venice again},roof]]
   ]
  ]
\end{forest}
\xe

\begin{tcolorbox}
Exercise
\tcblower
Figure out the projection properties of an \textit{if...then...} conditional, and write a lexical entry which captures them (you may treat \textit{if...then...} as a single connective).
\end{tcolorbox}

\section{Problems for a multi-dimensional semantics}

\subsection{The binding problem}

As noted by \citet{karttunenPeters1979}, the multi-dimensional theory has a (potentially fatal) flaw.

We can illustrate the problem simply by trying to give a multi-dimensional semantics for (\ref{binding-prob}).

\ex\label{binding-prob}Someone quit vaping.
\xe

\ex~
\(\displaystyle\frac{∃ x[x \ml{used to vape}]}{∃ x[x \ml{doesn't vape now}]}\)
\xe

Does this satisfactorily capture the meaning of the utterance in (\ref{binding-prob})? If not, why not?\footnote{In \citet{elliott-fuck} I give a multi-dimensional semantics for conventional implicature which doesn't face the binding problem, essentially by adopting a version of alternative semantics where alternatives are themselves multi-dimensional.

It would be natural to extend this system to presupposition. This could be a cool squib topic!
}

\subsection{Explanatory adequacy}

We observed a linear asymmetry in the projection pattern observed with conjunction.

Note that different predictions for presupposition projection are simply a matter of reversing the lambdas:

\ex
\(\displaystyle\eval{and} = λ \frac{p'}{p} . λ \frac{q'}{q} . \frac{p' ∧ (p ⇒ q')}{p ∧ q}\)
\xe

Now we (erroneously?) predict the following sentence to be presuppositionless:

\ex
Paul quit vaping and Paul used to vape.
\xe

Are we doing anything more than simply building the projection behaviour we observe into the lexical semantics of individual operators? This doesn't seem very satisfying. See \citet{schlenker_local_2009,schlenker_local_2010} for related criticisms.

Note that the problem of explanatory adequacy extends to the \textit{satisfaction} theory, which we'll be covering next week.

\section{Next time}

The remainder of the presupposition block will be devoted to an exposition of two of the major accounts of \textit{projection}:

\begin{itemize}

    \item The satisfaction theory (i.e., dynamic semantics)

    \begin{description}

        \item[part i]introduction to File Change Semantics.

      \item[part ii]projection from the complement of attitude verbs, the proviso problem, and other issues with the satisfaction theory; Mandelkern's dissatisfaction theory.

    \end{description}

    \item The trivalent theory (Roger).

\end{itemize}

\printbibliography

\end{document}

%%% Local Variables:
%%% TeX-engine: xetex
%%% TeX-master: t
%%% End:
