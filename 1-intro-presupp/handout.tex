\documentclass[cronos,landscape,paper=letter]{ling-handout}

\usepackage[margin=0.5in]{geometry}

\usepackage{fontawesome,calc,enumitem,parskip}

\lingset{
  belowexskip=0pt,
  aboveglftskip=0pt,
  belowglpreambleskip=0pt,
  belowpreambleskip=0pt,
  interpartskip=0pt,
  extraglskip=0pt,
  Everyex={\parskip=0pt},
}

\addbibresource[location=remote]{/home/patrl/repos/bibliography/elliott_mybib.bib}

% \renewcommand*{\sectionformat}{}
% \renewcommand*{\subsectionformat}{}
% \renewcommand*{\subsubsectionformat}{}

\title{\textit{Presupposition} primer}
\subtitle{\texttt{24.954:} Pragmatics in Linguistic Theory}

\date{\today}

\author{Patrick~D. Elliott}

 \begin{document}

\maketitle

\section{Overview}

\textbf{Roadmap}

Today we'll give a wide-ranging overview of the major problems, empirical and conceptual, in the literature on \textit{presupposition} -- a highly significant frontier on the border between semantics and pragmatics. We'll cover:

\begin{itemize}

    \item Natural Language Semantics and the law of the excluded middle

    \item The pragmatic status of presuppositions (i.e., \textit{backgroundedness})

    \item Semantic vs.\,pragmatic presupposition

  \item The projection problem

    \begin{itemize}

      \item Weakened projection

        \item Failed projection (i.e., \textit{local accommodation})

        \item Presupposition vs.\,other kinds of projective meaning

    \end{itemize}

  \item The triggering problem

  \item A multi-dimensional theory of presupposition (after \citealt{karttunenPeters1979})

\end{itemize}

\textbf{Readings}

Optional (but strongly recommended) readings for this week:

\begin{itemize}

  \item Beaver \& Guerts (2011) Presupposition. In \textit{Stanford Encyclopaedia of Philosophy}.\\
    \url{https://plato.stanford.edu/entries/presupposition/}

  \item Chapter 5 of Kadmon (1990) \textit{Formal Pragmatics}.

\end{itemize}

\textbf{The data under consideration}

Let's start by being maximally non-commital: a \enquote{presupposition}\footnote{When we say \textit{presupposition}, we typically mean \textit{semantic presupposition}, unless otherwise noted.} is a kind of \textit{inference} that sentences in natural languages \textit{may} convey.

\pex
\a Paul quit vaping.\\
presupp: \textit{Paul used to vape}
\a The mayor of Oakland waterskis.\\
presupp: \textit{Oakland has a mayor}
\a I re-read \textit{The Lord of the Rings}.\\
presupp: \textit{I've read the Lord of the Rings}
\a Uli forgot to submit the paperwork.\\
presupp: \textit{there was paperwork that Uli was supposed to submit}
\a None of my students failed her midterm.\\
presupp: \textit{Each of my students prefer \enquote{she/her} pronouns}
\a Sam walks her dog every morning.\\
presupp: \textit{Sam has a dog}
\a It was at the pub that Rob had a revelation.\\
presupp: \textit{Rob had a revelation}
\xe

Issues to flag up immediately:

\begin{itemize}

  \item Of all the inferences that a sentence conveys, how do we know which ones are \textit{presupposed}?

  \item Of all the inferences that a sentence conveys, \textit{why} are just the ones that are presupposed, presupposed?

\end{itemize}

\section{Eubulides' \textit{Horns paradox}}

Eubulides of Miletus' ($±$405-330 BC) noted that, if the following argument is sound, then we should conclude that \textit{everyone is a cuckold}. Why doesn't it go through?\footnote{We can thank Seuren (2005: p.\,89) for pointing out the relevance of Eubulides' paradoxes to contemporary semantic thinking.}

\pex
\a Major: What you haven't lost you still have.
\a\label{horns-minor}Minor: You have not lost your horns.
\a Ergo: You still have your horns.
\xe

To see what's going on more clearly, we can render the argument into predicate logic:

\pex
\a \(∀x[¬ \ml{youLost} x → \ml{youStillHave} x]\)\hfill Major
\a \(¬ \ml{youLost} \ml{YourHorns}\)\hfill Minor
\a \(¬ \ml{youLost} \ml{yourHorns} → \ml{youStillHave} \ml{yourHorns}\)\hfill \textit{Universal Instantiation}
\a \(\ml{youStillHave} \ml{yourHorns}\)\hfill\textit{Modus Ponens}
\xe

What went wrong here?

A non-trivial fact about natural language semantics: sentences don't obey the \textit{law of the excluded middle}, i.e., sentences in natural language can be neither true nor false.

Another non-trivial, and closely related fact about natural language semantics: sentence meanings are \textit{multi-dimensional}. In other words, sentences may convey qualitatively different kinds of meaning at the same time.

We can think of \textit{presupposition} as a dimension of meaning characterised by the following two properties:

\begin{itemize}

  \item backgroundedness

  \item projection

\end{itemize}

\section{Backgroundedness}

\todo[inline]{Mention Schlenker's description of presuppositions a ``epistemic preconditions''}

From an utterance of (\ref{paulvape}), we can infer (at least) two things: (\ref{paulvapea}) and (\ref{paulvapeb}).

\pex\label{paulvape}
Peter is talking about \textit{Eubulides} again
\a\label{paulvapea}\(⇝\) \textsf{Peter is talking about Eubulides now} \hfill at-issue
\a\label{paulvapeb}\(⇝\) \textsf{Peter has talked about Eubulides before} \hfill presupposition
\xe

There is an intuitive sense in which (\ref{paulvapea}) is the subject matter of the sentence -- in uttering (\ref{paulvape}), the speaker aims to communicate that (\ref{paulvapea}).

We can probe the pragmatic status of a given inference using questions -- the at-issue component of a sentence meaning can answer a question, whereas the presupposition cannot, as illustrated by the contrast between (\ref{q1}) and (\ref{q2}).

\ex\label{q1}
Which Greek philosopher is Peter talking about now?\\
\ljudge{\cmark} Peter is talking about \textit{Eubulides} again.
\xe

\ex~\label{q2}
Which Greek philosopher has Peter talked about before?\\
\ljudge{\#}Peter is talking about \textit{Eubulides} again.
\xe

\subsection{Pragmatic vs. semantic presupposition}

If I say the following to you...

\ex
Peter redet wieder von \textit{Eubulides}.
\xe

...I pragmatically presuppose (at least) the following:

\begin{itemize}

  \item you understand German.

  \item You can hear (or perhaps lip-read).

\end{itemize}

This information however is (probably!) not encoded in the semantics of the NL expressions.

Semantic presupposition relates to pragmatic presupposition, in the sense that semantic presupposition usually (but not always) gives rise to a pragmatic presupposition.

Later on in the class, we'll make precise exactly how this comes about, within a particular theory of presupposition.

\section{Projection}

As well as having a different \textit{pragmatic} status (wrt. backgroundedness), the inferences in (\ref{paulvapea_rep}) and (\ref{paulvapeb_rep}) behave differently in embedded contexts.

\pex\label{paulvape_rep}
Peter is talking about \textit{Eubulides} again
\a\label{paulvapea_rep}\(⇝\) \textsf{Peter is talking about Eubulides now} \hfill at-issue
\a\label{paulvapeb_rep}\(⇝\) \textsf{Peter has talked about Eubulides before} \hfill presupposition
\xe

Consider what happens when we turn (\ref{paulvape_rep}) into a polar question:

\pex
Is Peter talking about \textit{Eubulides} again?
\a \(⇝ ? (\ml{Peter is talking about Eubulides now})\)
\a \(⇝ \ml{Peter has talked about Eubulides before}\)
\xe

The inference in (\ref{paulvapea_rep}) is \textit{questioned}, whereas the inference in (\ref{paulvapeb_rep}) is unaffected. We see the same pattern in other \textit{non-veridical contexts}:

\ex
Peter isn't talking about \textit{Eubulides} again.
\xe

\ex~
There's no way that Peter is talking about \textit{Eubulides} again.
\xe

\ex~
Peter will never talk about \textit{Eubulides} again.
\xe

\ex~
Peter might be talking about \textit{Eubulides} again.
\xe

\ex~
Peter should talking about \textit{Eubulides} again.
\xe

\subsection{Weakened projection}

\ex\label{weakened}Patrick hopes that Peter will talk about \textit{Eubulides} again.
\xe

The complement of \enquote{hope} (\enquote{Peter will talk about Eubulides again}), as we've seen, presupposes: \textsf{Peter has talked about Eubulides before}.

What does (\ref{weakened}) presuppose globally?

\begin{itemize}

    \item \(⇝ \textsf{Peter has talked about Eubulides before.}\)

    \item \(⇝ \textsf{Patrick believes that Peter has talked about Eubulides before.}\)

\end{itemize}

It has been observed that presuppositions give rise to \textit{weakened} projection, when embedded under attitude verbs (\citealt{heim1992}). We'll talk more about this in session 3!

\subsection{Failed projection (or: \textit{Local Accommodation})}

\todo[inline]{Add some examples of local accommodation}

An example from Beaver \& Zeevat:
\todo[inline]{Add proper reference here}

\ex\label{accom}
Context: \textit{Our heroine has landed herself in a difficult spot. From all sides dangerous criminals are approaching.}\\
\textsc{the Heroine:} I knew they would show no mercy.
\xe

\textit{to know \(P\)} typically presupposes \(P\), but (\ref{accom}) is felicitous in a context where this is the first time the heroine has informed us of the treatment she expects at the hands of the villains.

In this instance the semantic presupposition fails to translate into a pragmatic presupposition.

\section{Projective meaning beyond presupposition}

Presupposition can be difficult to distinguish from \textit{conventional implicature}, which also displays both \textit{backgroundedness} and \textit{projectivity}.

\textit{Expressives/supplements} (as in (\ref{exp})) and \textit{appositives} two representative kinds of conventional implicature.
\todo[inline]{cite potts, mccready, gutzmann etc.}

\ex\label{exp}
This \textit{fucking} class is three hours long!\\
\(⇝ ☹ (\ml{this class is three hours long})\)
\xe

\ex\label{app}
The class, which lasted for three hours, was tedious.\\
\(⇝ \ml{the class lasted for three hours}\)
\xe

\textbf{Appositives vs.\,presuppositions}

Despite both being backgrounded, presuppositions typically express \textit{old} information, whereas appositives convey \textit{new} information.

Furthermore, whereas presuppositions can sometimes interact with other operators (cf. our discussion of \textit{weakened projection}, appositives always project all the way out).

\ex
Patrick hopes that Peter is talking about Eubulides again.\\
\(⇝ \ml{Patrick believes that Peter has talked about Eubulides before}\)\\
\phantom{,}\hfill weakened projection
\xe

\ex~
Patrick hopes that Peter is talking about Eubulides, who is a Greek philosopher.
\(⇝ \ml{Eubulides is a Greek philosopher}\)\hfill No weakened projection
\xe

\textbf{Expressives vs.\,presuppositions}

Expressives are harder to distinguish from presuppositions, and in fact, some authors group the two phenomena together.

\todo[inline]{Add references for this here}

\section{Triggering presupposition}

In many cases, the presence of a presupposition is traceable to a certain NL expression or construction. See, e.g., the contrast in (\ref{again}).

\pex\label{again}
\a Paul is vaping \textbf{again}.\hfill\textit{presupposes that Paul vaped before}
\a Paul is vaping.\hfill\textit{presuppositionless}
\xe

An \textit{it-cleft}, i.e., \textit{it was \(x\) that \(V-\)ed} presupposes that \textit{someone \(V-\)ed}.

\pex
\a It was \textsc{Paul} that vaped.
\a It wasn't \textsc{Paul} that vaped.
\xe

Factive predicates such as \textit{know} presuppose the truth of their complement.

\pex
\a Hubert knows that Paul vapes
\a Hubert doesn't know that Paul vapes.
\xe

We call such expressions and constructions \textbf{presupposition triggers}.

Expressions with similar at-issue meanings cross-linguistically tend to trigger the same presuppositions. This suggests that we want a unified theory of which inferences associated with NL expressions/constructions end up being presupposed.

This is known as the \textbf{triggering problem} of presupposition.

See \citet{schlenker2019} for an independent argument from pro-speech gestures that a triggering algorithm is independently necessary.





\section{A multi-dimensional theory of presupposition}

In this section, I'll spell out explicitly a \textit{multi-dimensional} account of presupposition, so that we can see its shortcomings clearly.\footnote{This is essentially a more compositionally explicit version of \citet{karttunenPeters1979}. It most closely mirrors \citeauthor{asudehGiorgolo2012}'s (\citeyear{asudehGiorgolo2012}) analysis of conventional implicature.}

Each natural language expression conveys two kinds of meanings: (a) an \textit{at-issue} meaning, and (b) a \textit{presupposition}.

We'll write this using \citeauthor{sauerland_2008}'s (\citeyear{sauerland_2008}) fraction notation; the presupposition goes on the top, and the at-issue meaning goes on the bottom. We'll assume that presuppositions are propositions.

\[\displaystyle\frac{\text{\it presupposition}}{\text{\it assertion}}\]

\[\eval{Paul quit vaping}= \displaystyle\frac{\ml{Paul used to vape}}{\ml{Paul doesn't vape now}}\]

Let's also define helper functions to retrieve the presupposition and assertion:

\begin{multicols}{2}
\ex
\(𝔸 \displaystyle\frac{p}{a} = a\)
\xe
\columnbreak
\ex
\(ℙ \displaystyle\frac{p}{a} = p\)
\xe
\end{multicols}

\subsection{Bridging between semantics and pragmatics}

Formal definition of a pragmatic presupposition:

\pex
Agents \(a_{1} … a_{n}\) pragmatically presuppose \(p\) iff the following hold:
\a Each \(a_{i}\) believes \(p\).
\a Each \(a_{i}\) believes that each \(a_{j}\) believes that \(p\).
\a Each \(a_{i}\) believes that each \(a_{j}\) believes that each \(a_{k}\) believes that \(p\)\\
\ldots
\xe

Semantic presuppositions as felicity conditions on utterances:

\ex
An utterance of sentence \(S\) by agents \(a_{1}…a_{n}\) is infelicitous unless \(a_{1}…a_{n}\) pragmatically presuppose \(ℙ \eval{S}\) (i.e., the semantic presupposition of \(S\))
\xe

\subsection{Compositionality}

We'll assume that any uni-dimensional (i.e., non-presuppositional) meaning can be lifted into a multidimensional meaning with a trivial presupposition, via an operator \(π\).

\[\eval{Paul doesn't vape}^{\pi} = \displaystyle\frac{⊤}{\ml{Paul doesn't vape}}\]

Composition proceeds by (a) doing \textit{function application} (FA) in the assertive dimension, and (b) \textit{conjunction} in the presuppositional dimension. We can define an operation of \textit{multi-dimensional function application} (MA) in order to formalise this.

\[\ml{MA} \displaystyle\frac{p}{x} \frac{q}{y} ≔ \frac{p ∧ q}{\ml{FA} x y}\]

      We can treat \textit{presupposition triggers} as functions from uni-dimensional individuals to multi-dimensional meanings. This gives us sub-sentential compositionality.

      \ex
      \(\eval{quit vaping} = λ x . \displaystyle\frac{x \ml{used to vape}}{x \ml{doesn't vape now}}\)
      \xe

      \ex~
      \begin{forest}
        [{\(\displaystyle\frac{\ml{Paul used to vape}}{\ml{Paul doesn't vape now}}\)\\\(\ml{FA}\)}
          [{Paul}]
          [{...} [{quit vaping},roof]]
        ]
      \end{forest}
      \xe

\subsection{Projection}

\subsubsection{Negation}

NL negation negates the assertive component of a sentence meaning, yet leaves the presupposition unaffected (to use \citeauthor{karttunenPeters1979}'s terminology, it is a \enquote{hole}).

\pex
\a Paul quit vaping.\hfill\textit{presupposes that Paul vaped before}
\a Paul didn't quit vaping.\hfill\textit{Presupposes that Paul vaped before}
\xe

We can capture this behaviour simply by assigning it a trivially multi-dimensional meaning:

\ex
\(\eval{not} = \displaystyle\frac{⊤}{\ml{not}}\)
\xe

\ex~
\begin{forest}
  [{\(\displaystyle\frac{\ml{paul used to vape}}{\ml{paul vapes now}}\)\\\(\ml{MA}\)}
    [{\(\displaystyle\frac{⊤}{\ml{not}}\)}]
    [{$\displaystyle\frac{\ml{paul used to vape}}{\ml{paul doesn't vape now}}$} [{Paul quit vaping},roof]]
  ]
\end{forest}
\xe

\subsubsection{Conjunction}

How do presuppositions project in conjunctive sentences?

\ex
Sam visited Rome again and Ka visited Venice.\\
presupp.: \(\ml{Sam has visited Rome before}\)
\xe

\ex~
Sam visited Rome and Ka visited Venice again.\\
presupp.: \(\ml{Ka has visited Venice before}\)
\xe

\ex
Sam and Ka visited Rome and Venice last Summer, and Ka visited Venice again.\\
\textit{presuppositionless}
\xe

\ex~
Ka visited Venice again, and Sam and Ka visited Rome and Venice last Summer.\\
presupp.: \(\ml{Ka has visited Venice before}\)
\xe

As famously observed by \citeauthor{karttunenPeters1979}, presuppositions embedded in a conjunctive sentence do not always project all the way out.

We'll adopt the following entry for sentential conjunction (assuming a left-branching structure for conjunctive sentences).

\ex
\(\displaystyle\eval{and} ≔ λ \frac{q'}{q} . λ \frac{p'}{p} . \frac{p' ∧ (p ⇒ q')}{p ∧ q}\)
\xe

\ex~
\begin{forest}
    [{\(\overbrace{\displaystyle\frac{\ml{if s and k visited r and v last summer then k has visited v before}}{\ml{s and k visited r and v last summer and k visited v}}}^{\top}\)}
      [{\(\displaystyle\frac{⊤}{\ml{s and k visited r and v last summer}}\)\\\(\pi\)} [{Sam and Ka visited\\Rome and Venice last Summer},roof]]
      [{...}
        [{and}]
        [{\(\displaystyle\(\frac{\ml{k has visited v before}}{\ml{k visited v}}\)\)} [{Ka visited Venice again},roof]]
      ]
    ]
  \end{forest}
\xe

\textbf{A preview of the proviso problem}

\ex
\label{proviso}Mary is pregnant and her brother is happy.
\xe

\ex~
\(ℙ (\eval{her brother is happy}) = \ml{Mary has a brother}\)
\xe

Our entry for \enquote{and} predicts the following:

\ex
\(\displaystyle\frac{\ml{if Mary is pregnant then Mary has a brother}}{\ml{Mary is pregnant and Mary's brother is happy}}\)
\xe

This seems too weak! (\ref{proviso}) is only felicitous in a context where the common ground entails that Mary has a brother.

Other theories of presupposition projection inherit the problem of assigning certain sentences weak, conditional presuppositions; this is known as the \textit{proviso problem}. We'll come back to this in week three, after introducing the satisfaction theory of presupposition projection (i.e., dynamic semantics).

\subsection{Disjunction}

\ex
Either Sam visited London again, or Ka visited Venice.\\
presupp.: \(\ml{Sam has visited London before}\)
\xe

\ex~
Either Sam visited London, or Ka visited Venice again.\\
presupp.: \(\ml{Ka has visited Venice before}\)
\xe

\ex~
Either Ka has never visited Venice, or she visited Venice again.\\
\textit{presuppositionless}
\xe

\ex~
Either Ka visited Venice again, or she has never visited Venice.\\
???
\xe

\ex~
\(\eval{or} ≔ \displaystyle λ\frac{q'}{q} . λ\frac{p'}{p} . \frac{p' ∧ ¬p ⇒ q'}{p ∨ q}\)
\xe

\ex~
\begin{forest}
  [{\(\overbrace{\displaystyle\frac{\ml{if k has visited v before she has visited v before}}{\ml{Ka hasn't visited v before or she visited v}}}^{\top}\)}
   [{\(\displaystyle\frac{\top}{\ml{k hasn't visited Venice before}}\)\\\(\pi\)} [{Ka has never visited Venice},roof]]
   [{...}
     [{or}]
     [{\(\displaystyle\frac{\ml{k visited v before}}{\ml{k visited v}}\)} [{Ka visited Venice again},roof]]
   ]
  ]
\end{forest}
\xe

\section{Problems for a multi-dimensional semantics}

\subsection{The binding problem}

As noted by \citet{karttunenPeters1979}, the multi-dimensional theory has a (potentially fatal) flaw.

We can illustrate the problem simply by trying to give a multi-dimensional semantics for (\ref{binding-prob}).

\ex\label{binding-prob}Someone quit vaping.
\xe

\ex~
\(\displaystyle\frac{∃ x[x \ml{used to vape}]}{∃ x[x \ml{doesn't vape now}]}\)
\xe

Does this satisfactorily capture the meaning of the utterance in (\ref{binding-prob})? If not, why not?\footnote{In \citet{elliott-fuck} I give a multi-dimensional semantics for conventional implicature which doesn't face the binding problem, essentially by adopting a version of alternative semantics where alternatives are themselves multi-dimensional.

It would be natural to extend this system to presupposition. This could be a cool squib topic!
}

\subsection{Explanatory adequacy}

We observed a linear asymmetry in the projection pattern observed with conjunction.

Note that different predictions for presupposition projection are simply a matter of reversing the lambdas:

\ex
\(\displaystyle\eval{and} = λ \frac{p'}{p} . λ \frac{q'}{q} . \frac{p' ∧ (p ⇒ q')}{p ∧ q}\)
\xe

Now we (erroneously?) predict the following sentence to be presuppositionless:

\ex
Paul quit vaping and Paul used to vape.
\xe

Are we doing anything more than simply building the projection behaviour we observe into the lexical semantics of individual operators? This doesn't seem very satisfying. See \citet{schlenker_local_2009,schlenker_local_2010} for related criticisms.

Note that the problem of explanatory adequacy extends to the \textit{satisfaction} theory, which we'll be covering next week.

\section{Next time}

The remainder of the presupposition block will be devoted to an exposition of two of the major accounts of \textit{projection}:

\begin{itemize}

    \item The satisfaction theory (i.e., dynamic semantics)

    \begin{description}

        \item[part i]overview of dynamic semantics: \cite{heim1983,rothschild2017}
        \todo[inline]{Add more references here}

      \item[part ii]projection from the complement of attitude verbs, the proviso problem, etc.: \cite{heim1992}
        \todo[inline]{Add, e.g., a reference to Mandelkern here}

    \end{description}

    \item The trivalent theory
    \todo[inline]{Add references for trivalent theory}

\end{itemize}

\printbibliography

\end{document}

%%% Local Variables:
%%% TeX-engine: xetex
%%% TeX-master: t
%%% End:
