\documentclass[cronos,landscape,paper=letter]{ling-handout}

\usepackage[margin=0.5in]{geometry}

\usepackage{fontawesome,calc,enumitem,parskip}

\lingset{
    aboveglftskip=0pt,
    belowglpreambleskip=0pt,
    belowpreambleskip=0pt,
    extraglskip=0pt,
    Everyex={\parskip=0pt},
}

\addbibresource[location=remote]{/home/patrl/repos/bibliography/elliott_mybib.bib}

% \renewcommand*{\sectionformat}{}
% \renewcommand*{\subsectionformat}{}
% \renewcommand*{\subsubsectionformat}{}

\title{Presupposition}
\subtitle{\texttt{24.954:} Pragmatics in Linguistic Theory}

\date{\today}

\author{Patrick~D. Elliott}

 \begin{document}

\maketitle

Readings for this week:

\begin{itemize}

  \item Beaver \& Guerts (2011) Presupposition. In \textit{Stanford Encyclopaedia of Philosophy}.\\
    \url{https://plato.stanford.edu/entries/presupposition/}

  \item Chapter 5 of Kadmon (1990) \textit{Formal Pragmatics}.

\end{itemize}

Let's start by being maximally non-commital: a \enquote{presupposition}\footnote{When we say \textit{presupposition}, we typically mean \textit{semantic presupposition}, unless otherwise noted.} is a kind of \textit{inference} that sentences in natural languages \textit{may} convey.

\pex
\a Paul quit vaping.\\
presupp: \textit{Paul used to vape}
\a The mayor of Oakland waterskis.\\
presupp: \textit{Oakland has a mayor}
\a I re-read \textit{The Lord of the Rings}.\\
presupp: \textit{I've read the Lord of the Rings}
\a Uli forgot to submit the paperwork.\\
presupp: \textit{there was paperwork that Uli was supposed to submit}
\a None of my students failed her midterm.\\
presupp: \textit{Each of my students prefer \enquote{she/her} pronouns}
\a Sam walks her dog every morning.\\
presupp: \textit{Sam has a dog}
\a It was at the pub that Rob had a revelation.\\
presupp: \textit{Rob had a revelation}
\xe

Issues to flag up immediately:

\begin{itemize}

  \item Of all the inferences that a sentence conveys, how do we know which ones are \textit{presupposed}?

  \item Of all the inferences that a sentence conveys, \textit{why} are just the ones that are presupposed, presupposed?

\end{itemize}

\section{Eubulides' \textit{Horns paradox}}

Eubulides of Miletus' ($±$405-330 BC) noted that, if the following argument is sound, then we should conclude that \textit{everyone is a cuckold}. Why doesn't it go through?\footnote{We can thank Seuren (2005: p.\,89) for pointing out the relevance of Eubulides' paradoxes to contemporary semantic thinking.}

\pex
\a Major: What you haven't lost you still have.
\a\label{horns-minor}Minor: You have not lost your horns.
\a Ergo: You still have your horns.
\xe

Crucially, (\ref{horns-minor}) \textit{presupposes} that you have horns.

A non-trivial fact about natural language semantics: sentences don't obey the \textit{law of the excluded middle}, i.e., sentences in natural language can be neither true nor false.

Another non-trivial, and closely related fact about natural language semantics: sentence meanings are \textit{multi-dimensional}. In other words, sentences may convey qualitatively different kinds of meaning at the same time.

We can think of \textit{presupposition} as a dimension of meaning characterised by the following two properties:

\begin{itemize}

  \item backgroundedness

  \item projection

\end{itemize}

\section{Backgroundedness}

\section{Projection}

\section{Failed projection}

\section{Other dimensions of meaning}

\section{A multi-dimensional theory of presupposition}

Each natural language expression conveys two kinds of meanings: (a) an \textit{at-issue} meaning, and (b) a \textit{presupposition}.

We'll write this using Sauerland's fraction notation; the presupposition goes on the top, and the at-issue meaning goes on the bottom. We'll assume that presuppositions are propositions.

\[\displaystyle\frac{\text{\it presupposition}}{\text{\it assertion}}\]

\[\eval{Paul quit vaping}= \displaystyle\frac{\ml{Paul used to vape}}{\ml{Paul doesn't vape now}}\]

Let's also define helper functions to retrieve the presupposition and assertion:

\begin{multicols}{2}
\ex
\(𝔸 \displaystyle\frac{p}{a} = a\)
\xe
\columnbreak
\ex
\(ℙ \displaystyle\frac{p}{a} = p\)
\xe
\end{multicols}

\subsection{Bridging between semantics and pragmatics}

Formal definition of a pragmatic presupposition:

\pex
Agents \(a_{1} … a_{n}\) pragmatically presuppose \(p\) iff the following hold:
\a Each \(a_{i}\) believes \(p\).
\a Each \(a_{i}\) believes that each \(a_{j}\) believes that \(p\).
\a Each \(a_{i}\) believes that each \(a_{j}\) believes that each \(a_{k}\) believes that \(p\)\\
\ldots
\xe

Semantic presuppositions as felicity conditions on utterances:

\ex
An utterance of sentence \(S\) by agents \(a_{1}…a_{n}\) is infelicitous unless \(a_{1}…a_{n}\) pragmatically presuppose \(\ml{presupp} \eval{S}\) (i.e., the semantic presupposition of \(S\))
\xe

\subsection{Compositionality}

We'll take NL expressions that don't carry a presupposition to presuppose a tautology (\(⊤\)).

\[\eval{Paul doesn't vape} = \displaystyle\frac{⊤}{\ml{Paul doesn't vape}}\]

Composition proceeds by (a) doing \textit{function application} (FA) in the assertive dimension, and (b) \textit{conjunction} in the presuppositional dimension. We can define an operation of \textit{multi-dimensional function application} (MA) in order to formalise this.

\[\ml{MA} \displaystyle\frac{p}{x} \frac{q}{y} ≔ \frac{p ∧ q}{\ml{FA} x y}\]

      We can treat \textit{presupposition triggers} as functions from uni-dimensional individuals to multi-dimensional meanings:

      \ex
      \(\eval{quit vaping} = λ x . \displaystyle\frac{x \ml{used to vape}}{x \ml{doesn't vape now}}\)
      \xe

\subsection{Projection}

\subsubsection{Negation}

NL negation negates the assertive component of a sentence meaning, yet leaves the presupposition unaffected (to use \citeauthor{karttunenPeters1979}'s terminology, it is a \enquote{hole}).

We can capture this behaviour simply by assigning it a trivially multi-dimensional meaning:

\ex
\(\eval{not} = \displaystyle\frac{⊤}{\ml{not}}\)
\xe

\ex~
\begin{forest}
  [{\(\displaystyle\frac{\ml{paul used to vape}}{\ml{paul vapes now}}\)\\\(\ml{MA}\)}
    [{\(\displaystyle\frac{⊤}{\ml{not}}\)}]
    [{$\displaystyle\frac{\ml{paul used to vape}}{\ml{paul doesn't vape now}}$} [{Paul quit vaping},roof]]
  ]
\end{forest}
\xe

\subsubsection{Conjunction}

We'll adopt the following entry for sentential conjunction (assuming a left-branching structure for conjunctive sentences).

\ex
\(\displaystyle\eval{and} = λ \frac{q'}{q} . λ \frac{p'}{p} . \frac{p' ∧ (p ⇒ q')}{p ∧ q}\)
\xe

\ex~
\(\displaystyle\eval{\begin{array}{c}\begin{forest}
    [{...}
      [{...} [{Paul used to smoke},roof]]
      [{...}
        [{and}]
        [{...} [{Paul quit smoking},roof]]
      ]
    ]
  \end{forest}\end{array}}\)\\
\(= \displaystyle\frac{\ml{if Paul used to vape then Paul used to vape}}{\ml{Paul used to vape and Paul doesn't vape now}}\)\\
\(= \displaystyle\frac{⊤}{\ml{Paul used to vape and Paul doesn't vape now}}\)
\xe

\textbf{A preview of the proviso problem}

\ex
\label{proviso}Mary is pregnant and her brother is happy.
\xe

\ex~
\(ℙ (\eval{her brother is happy}) = \ml{Mary has a brother}\)
\xe

Our entry for \enquote{and} predicts the following:

\ex
\(\displaystyle\frac{\ml{if Mary is pregnant then Mary has a brother}}{\ml{Mary is pregnant and Mary's brother is happy}}\)
\xe

This seems too weak! (\ref{proviso}) is only felicitous in a context where the common ground entails that Mary has a brother.

Other theories of presupposition projection inherit the problem of assigning certain sentences weak, conditional presuppositions; this is known as the \textit{proviso problem}. We'll come back to this in week three, after introducing the satisfaction theory of presupposition projection (i.e., dynamic semantics).

\subsection{The binding problem}

As noted by \citet{karttunenPeters1979}, the multi-dimensional theory has a (potentially fatal) flaw.

We can illustrate the problem simply by trying to give a multi-dimensional semantics for (\ref{binding-prob}).

\ex\label{binding-prob}Someone quit vaping.
\xe

\ex~
\(\displaystyle\frac{∃ x[x \ml{used to vape}]}{∃ x[x \ml{doesn't vape now}]}\)
\xe

Does this satisfactorily capture the meaning of the utterance in (\ref{binding-prob})? If not, why not?

\subsection{Explanatory adequacy}

We built a linear asymmetry directly into the entry for sentential conjunction. Different predictions for presupposition projection are simply a matter of reversing the lambdas:

\ex
\(\displaystyle\eval{and} = λ \frac{p'}{p} . λ \frac{q'}{q} . \frac{p' ∧ (p ⇒ q')}{p ∧ q}\)
\xe

Now we (erroneously?) predict the following sentence to be presuppositionless:

\ex
Paul quit vaping and Paul used to vape.
\xe

\section{Next time}

The next two sessions will be devoted to an exposition one of the major accounts of projection: the \textit{satisfaction} theory. In order to understand the satisfaction theory, we'll learn about dynamic semantics.

\printbibliography

\end{document}

%%% Local Variables:
%%% TeX-engine: default
%%% TeX-master: t
%%% End:
