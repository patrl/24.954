\documentclass[cronos,landscape,paper=letter]{ling-handout}

\usepackage[margin=0.5in]{geometry}

\usepackage{fontawesome,calc,enumitem,parskip}

\lingset{
    aboveglftskip=0pt,
    belowglpreambleskip=0pt,
    belowpreambleskip=0pt,
    extraglskip=0pt,
    Everyex={\parskip=0pt},
}

\usepackage{tcolorbox}

\addbibresource[location=remote]{/home/patrl/repos/bibliography/elliott_mybib.bib}

% \renewcommand*{\sectionformat}{}
% \renewcommand*{\subsectionformat}{}
% \renewcommand*{\subsubsectionformat}{}

\title{Anaphora P-Set}
\subtitle{\texttt{24.954:} Pragmatics in Linguistic Theory}

\date{\today}

\author{Patrick~D. Elliott}

\begin{document}

\maketitle

\begin{tcolorbox}
  Questions
  \tcblower

Compute the CCP of the following sentence, using our dynamic fragment extended to account for anaphora. You can ignore gender.

\ex
A\(^{1}\) son of some\(^{2}\) famous actress came and he\(_{1}\) told us about her\(_{2}\).
\xe

\end{tcolorbox}

In our system, we have special (syncategorematic) interpretation rules for sentences of the form \enquote{P and Q}, and sentences of the form \enquote{a\(^{n}\) \(ϕ ψ\)}. This isn't a \textit{necessary} feature of FCS (i.e., we can make things more compositional, if we like) but it did simplify the exposition. Let's remind ourselves of these rules:

\ex
\(\text{a}^{n} ϕ ψ ≔ \eval{a\(^{n}\)};\eval{\(ϕ\)};\eval{\(ψ\)}\)
\xe

\ex~
\(\text{P and Q} ≔ \eval{P};\eval{Q}\)
\xe


We'll assume the following LF for the sentence we're interested in. Since our interpretation rules are syncategorematic, the syntactic structure doesn't make a large difference, so to simplify, let's assume that both coordinated sentences and quantificational structures are ternary.

  \ex
  \begin{forest}
    [{\(④\)}
    [{\(②\)}
      [{a\(^{1}\)}]
      [{\(①\)}
        [{some\(^{2}\)}]
        [{\(t_{2}\) famous actress}]
        [{\(t_{1}\) son of \(t_{2}\)}]
      ]
      [{\(t_{1}\) came}]
    ]
      [{and}]
      [{\(③\)} [{he\(_{1}\) told us about her\(_{2}\)},roof]]
    ]
  \end{forest}
  \xe

  Let's start by stating the interpretation of each of the daughters of \(①\):

  \ex
  \(\eval{some\(^{2}\)} ≔ λ c:2 ∉ \ml{dom} c . \set{⟨g'[2 ↦ x],w⟩|⟨g,w⟩ ∈ c ∧ x ∈ D_{\ml{e}}}\)
  \xe

  \ex~
  \(\eval{\(t_{2}\) famous actress} = λ c:2 ∈ \ml{dom} c . \set{⟨g,w⟩ ∈ c|\ml{famous-actress}_w g_2}\)
  \xe

  \ex~
  \(\eval{\(t_{1}\) son of \(t_{2}\)} = λ c:1,2 ∈ \ml{dom} c . \set{⟨g,w⟩ ∈ c|g_1 \ml{son-of} g_2}\)
  \xe

  Now we can compute the meaning of \(①\) using the special rule for sentences of the form \enquote{a\(^{n} ϕ ψ\)}:

  \pex
  \a \(\eval{\(①\)} = \eval{some\(^{n}\)};\eval{\(t_{2}\) famous actress};\eval{\(t_{1}\) son of \(t_{2}\)}\)
  \a \(= λ c:\begin{aligned}[t]
    &1 ∈ \ml{dom} c\\
    &∧ 2 ∉ \ml{dom} c
  \end{aligned} . \Set{⟨g'[2 ↦ x],w⟩|\begin{aligned}[c]
      &⟨g,w⟩ ∈ c\\
      &∧ \ml{famous-acress} x\\
      &∧ g_{1} \ml{son-of} g_{2}
    \end{aligned}}\)
  \xe

  Now let's compute the interpretation of the two remaining daughters of \(②\):

  \ex
  \(\eval{a\(^{1}\)} ≔ λ c:1 ∉ \ml{dom} c . \set{⟨g'[1 ↦ x],w⟩|⟨g,w⟩ ∈ c ∧ x ∈ D_{\ml{e}}}\)
  \xe

  \ex~
  \(\eval{\(t_{1}\) came} = λ c:1 ∈ \ml{dom} c . \set{⟨g,w⟩ ∈ c|\ml{came} g_1}\)
  \xe

  Again, based on our interpretation rule for a sentence with an indefinite, we sequence the three daughters of \( ②\). We get:

  \ex
  \(λ c:2,1 ∉ \ml{dom} c . \Set{\left\langle g'\left[\begin{aligned}[c]
        &1 ↦ y\\
        &2 ↦ x
      \end{aligned} \right],w \right \rangle |\begin{aligned}[c]
    &⟨g,w⟩ ∈ c\\
    &∧ \ml{famous-actress} x\\
    &∧ y \ml{son-of} x\\
    &∧ \ml{came} y
  \end{aligned}}\)
  \xe

  Now, the interpretation of \(③\) is easy.

  \ex
  \(③ = \eval{\(t_{1}\) told us about \(t_{2}\)} = λ c:1,2 ∈ \ml{dom} c . \set{⟨g,w⟩|g_1 \ml{told-about} g_2}\)
  \xe

  FINALLY, we compute the meaning of \( ④\) by sequencing \( ②\) and \( ③\), which we already computed:

  \ex
  \(④ = λ c:2,1 ∉ \ml{dom} c . \Set{\left\langle g'\left[\begin{aligned}[c]
        &1 ↦ y\\
        &2 ↦ x
      \end{aligned} \right],w \right \rangle |\begin{aligned}[c]
    &⟨g,w⟩ ∈ c\\
    &∧ \ml{famous-actress} x\\
    &∧ y \ml{son-of} x\\
    &∧ \ml{came} y\\
    &∧ y \ml{told-about} x
  \end{aligned}}\)\)
  \xe

  \hline

  \begin{tcolorbox}
    Question
    \tcblower
In dynamic semantics, we can define a universal quantifier as follows (where \(p\) and \(q\) are CCPs):

\ex
\(\ml{every}^{n} p q ≔ λ c\begin{aligned}[t]
  &: i ∉ \ml{dom} c\\
  &. \Set{⟨g,w⟩ ∈ c| \begin{aligned}[c]
    &\set{g'_{n} | ⟨g',w⟩ ∈ p (\ml{a}^{n} c) ∧ g ≤ g'}\\
    &⊆ \set{g''_{n}|⟨g'',w⟩ ∈ q (p (\ml{a}^{n} c)) ∧ g ≤ g''}
  \end{aligned}}
\end{aligned}\)
\xe

\ex~
\(g ≤ g'\) iff for each \(i ∈ dom g, g_{i} = g'_{i}\)
\xe

Compute the CCPs of the following sentences, step by step:

\ex
Every linguist cried.\\
LF: Every\(^{3}\) [\(t_{3}\) linguist] [\(t_{3}\) cried]
\xe

\ex~
Every\(^{4}\) farmer who owns a\(^{7}\) donkey cares from it\(_{7}\).\\
LF: Every\(^{4}\) [a\(^{7}\) [\(t_{7}\) donkey] [\(t_{4}\) owns \(t_{7}\)]] [\(t_{4}\) cares for \(t_{7}\)].
\xe


\end{tcolorbox}

Let's start with \enquote{Every linguist cried}:


\ex
\(\eval{t\(_{3}\) linguist} = λ c:3 ∈ \ml{dom} c . \set{⟨w,g⟩ ∈ c|\ml{linguist} g_3}\)
\xe

\ex~
\(\eval{t\(_{3}\) cried} = λ c:3 ∈ \ml{dom} c . \set{⟨w,g⟩ ∈ c|\ml{cried} g_3}\)
\xe

Now we can simply apply our syncategorematic rule for \enquote{every\(^{n}\) \(p q\)}:

\ex
\(\begin{aligned}[t]
  &\eval{every linguist cried}\\
  &= λ c\begin{aligned}[t]
    &:3 ∉ \ml{dom} c\\
    &. \Set{⟨g,w⟩ ∈ c | \begin{aligned}[c]
        &\set{g'_{3}|⟨g',w⟩ ∈ \set{⟨g''[3 ↦ x],w⟩|⟨g'',w⟩ ∈ c ∧ \ml{linguist} x} ∧ g ≤ g'}\\
        & ⊆ \set{h_{3}|⟨h,w⟩ ∈ \set{⟨h'[3 ↦ x],w⟩|⟨h',w⟩ ∈ c ∧ \ml{cried} x} ∧ g ≤ h}
      \end{aligned}}
    \end{aligned}
  \end{aligned}\)
\xe

Note importantly that this sentence doesn't result in an updated context with an extended domain, capturing the fact that \textit{every}, unlike \textit{some} doesn't license cross-sentential anaphora.

% Now let's move onto the donkey sentence. Here is the LF, again we're just going to make it ternary branching to more easily see what's going on:

% \ex
% \begin{forest}

% \end{forest}
% \xe

% \hline

% What do we now predict that the following sentence should presuppose?

% \ex
% Every fat man pushed his bike.
% \xe

% What about the following:

% \ex
% Every fat man who stopped smoking was healthier.
% \xe

% \textbf{Bonus: }what is the general schema for going from a static generalised quantifier to a dynamic generalised quantifier?

% \printbibliography

\end{document}

%%% Local Variables:
%%% TeX-engine: default
%%% TeX-master: t
%%% End:
