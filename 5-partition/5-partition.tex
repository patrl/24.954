\documentclass[portrait,cronos,paper=letter]{ling-handout}


\usepackage[margin=1in]{geometry}
\usepackage{fontawesome,calc,enumitem,parskip,tcolorbox}

\lingset{
  belowexskip=0pt,
  aboveglftskip=0pt,
  belowglpreambleskip=0pt,
  belowpreambleskip=0pt,
  interpartskip=0pt,
  extraglskip=0pt,
  Everyex={\parskip=0pt},
}


\usepackage{float,soul}

\addbibresource[location=remote]{/home/patrl/repos/bibliography/elliott_mybib.bib}

% \renewcommand*{\sectionformat}{}
% \renewcommand*{\subsectionformat}{}
% \renewcommand*{\subsubsectionformat}{}

\title{Partition by Exhaustification}
\subtitle{Question semantics and the logic of scalar strengthening}

\date{November 1, 2019}

\author{Patrick~D. Elliott
  \and
Roger Schwarzchild}

\begin{document}

\maketitle

\begin{tcolorbox}
  Squib deadline
  \tcblower
  For those of you registered for this class, the first squib (on
  presupposition) is due in today.
\end{tcolorbox}

\section*{Homework}

\begin{itemize}

    \item Next week we'll be talking about plurality, and, specifically,
    \textit{multiplicity inferences}.

    \item \textbf{Presupposition and implicature will collide}.

    \item You're required to read either of the following papers (free choice
    inference!). Send me at least one question via email by next Wednesday.

    \begin{itemize}

        \item \fullcite{spector2007}

        \item \fullcite{sauerlandAnderssenYatsushiro2005}

    \end{itemize}

    \item I'll upload both papers to stellar later today.

\end{itemize}

\section{Introduction to Partition semantics}


\begin{itemize}

  \item \citet{groenendijk_studies_1984} capture the difference between
    \textit{indicatives} and \textit{interrogatives} by treating the latter as
    partitions/propositional concepts.

    \ex
    \(\eval{it's raining} ≔ \set{w| \ml{raining}_{w}}\)
    \xe

    \ex~
    \(\eval{is it raining?} ≔ λ w . \set{w'|\ml{raining}_{w} = \ml{raining}_{w'}}\)
    \xe

    \item A partition divides up the logical space (i.e., the set of
    possible worlds) into mutually exclusive possibilities: \enquote{cells}.

  \item Consider the following illustration:

  \begin{table}[H]
\begin{tabular}{llllll}
 & $w_{r,s}$ & $w_{r,¬s}$ & $w_{¬r,s}$ & $w_{¬r,¬s}$ &  \\
  \hline
\ml{raining} & 1 & 1 & 0 & 0 &  \\
\ml{sunny} & 1 & 0 & 1 & 0 &
\end{tabular}
\end{table}

    \item The meaning of the interrogative tells us, for any evaluation world, which
    cell in the partition it belongs to.

    \[\eval{is it raining?} = \left[\begin{aligned}[c]
          &w_{r,s} &↦ &\set{w_{r,s},w_{r,¬s}}\\
          &w_{r,¬s} &↦ &\set{w_{r,s},w_{r,¬s}}\\
          &w_{¬r,s} &↦ &\set{w_{¬r,s},w_{¬r,¬s}}\\
          &w_{¬r,¬s} &↦ &\set{w_{¬r,s},w_{¬r,¬s}}
        \end{aligned}\right]\]

    \item The interrogative \textit{is it sunny?} of course delivers a distinct
    partition:

    \[\eval{is it raining?} = \left[\begin{aligned}[c]
          &w_{r,s} &↦ &\set{w_{r,s},w_{¬r,s}}\\
          &w_{r,¬s} &↦ &\set{w_{r,¬s},w_{¬r,¬s}}\\
          &w_{¬r,s} &↦ &\set{w_{r,s},w_{¬r,s}}\\
          &w_{¬r,¬s} &↦ &\set{w_{r,¬s},w_{¬r,¬s}}
        \end{aligned}\right]\]

    \item G\&S's partition semantics extends straightforwardly to constituent
    questions:

    \ex
    \(\eval{who arrived} = λ w . \set{w'|\set{x|\ml{arrived}_{w} x} = \set{x'|\ml{arrived}_{w'}}}\)
    \xe

    \item The G\&S meaning for \textit{who arrived} partitions the logical space
    according to the person(s) who arrived. Assume that the domain of
    individuals is \(\ml{Paul, Sophie}\).

    \[\begin{aligned}[c]
&\set{w| \set{x|\ml{arrived}_{w} x} = \set{\ml{Paul,Sophie}}}\\
&\set{w| \set{x|\ml{arrived}_{w} x} = \set{\ml{Paul}}}\\
&\set{w| \set{x|\ml{arrived}_{w} x} = \set{\ml{Sophie}}}\\
&\set{w| \set{x|\ml{arrived}_{w} x} = ∅}
      \end{aligned}\]

        \[\begin{aligned}[c]
&\set{w| \ml{Paul and Sophie arrived in }w}\\
&\set{w| \ml{only Paul arrived in }w}\\
&\set{w| \ml{only Sophie arrived in }w}\\
&\set{w| \ml{nobody arrived in }w}
\end{aligned}\]

    \end{itemize}

    \subsection{Exhaustivity}

    \begin{itemize}

      \item Two forms of exhaustivity (discussed by G\&S):

        \pex
        Dani knows who came.
        \a If $x$ came, Dani believes that $x$ came.\hfill \textit{weak exhaustivity}
        \a If $x$ came, Dani believes that $x$ came; if $x$ didn't come, then
        Dani believes that $x$ didn't come.\hfill \textit{strong exhaustivity}
        \xe

      \item A partition semantics for questions directly captures the strongly
        exhaustive interpretation; the partition denoted by a question is a
        function from a world $w$ to the \textit{strongly exhaustive} answer to the
        question in $w$.

        \pex\label{ex:part}Dani knows who came.\\
        $λ w . \ml{d} \ml{know}_{w} \set{w'|\set{x|\ml{came}_{w} x} =
        \set{x|\ml{came}_{w'} x}}$
        \a If only Yasu came in $w$, then:\\
        $\eval{\ref{ex:part}} w = \ml{d
        know}_{w} \set{w'|\set{x|\ml{came}_{w'} x} = \set{\ml{y}}}$
        \a If only Andy came in $w$, then:\\
        $\eval{\ref{ex:part}} w = \ml{d
        know}_{w} \set{w'|\set{x|\ml{came}_{w'} x} = \set{\ml{a}}}$
        \a If only Yasu and Andy came in $w$, then:\\
        $\eval{\ref{ex:part}} w = \ml{d
        know}_{w} \set{w'|\set{x|\ml{came}_{w'} x} = \set{\ml{y},\ml{a},\ml{y}⊕\ml{a}}}$
        \a If nobody came in $w$, then:\\
        $\eval{\ref{ex:part}} w = \ml{d
        know}_{w} \set{w'|\set{x|\ml{came}_{w'} x} = ∅}$
        \xe

        \item A strongly exhaustive interpretation seems to be motivated, at
        least for the interpretation of \textit{know}, as noted by G\&S:

        \pex
        \a Dani believes that Yasu and Andy are here.
        \a Only Andy is here.
        \a $⇏$ Dani knows who is here.
        \xe

        \item As famously observed by \citet{heim1994}, there are certain
        predicates for which this doesn't seem right.

        \pex
        Dani is surprised by who came.
        \a \cmark\,If $x$ came, Dani is surprised that $x$ came.
        \a ???\,If $x$ came, Dani is surprised that $x$ came; if $x$ didn't
        come, then Dani is surprised that $x$ didn't come.
        \xe

        \pex Context: \textit{only Andy came, and Dani is surprised that Andy came. Yasu
        didn't come, and this doesn't surprise Dani one bit}.\\
        \a Dani is surprised by who came.
        \a\ljudge{\#}Dani isn't surprised by who came.
        \xe

\end{itemize}

\subsection{From Hamblin sets to partitions}

\subsubsection{Background on equivalence classes}

\begin{itemize}

\item For any set, $X$, if we have some notion of equivalence (called
      an \textit{equivalence relation} ($\sim$)) for the members $x ∈ X$, then we may
      divide up $X$ into equivalence classes -- subsets of $X$. Two elements $x$
      and $x'$ belong to the same equivalence class only if they satisfy the
      equivalence relation.

     \item Let's say that we have a set of individuals, and our equivalence relation
      is \textit{eye color}. In other words:

      \[x ∼ x' \text{iff} \ml{eye-color} x = \ml{eye-color} x'\]

      \item For each individual $y$, their equivalence class is the following set:

      \[\set{x ∈ X|x \sim y}\]

      \item This maps, e.g., Britta to the set of blue-eyed individuals, if she has
      blue eyes, and Jeff to the set of brown-eyed individuals, if he has brown eyes.

      \item The set of equivalence classes is therefore the following set

      \[\bigcup\limits_{y∈X}\set{X'|X' = \set{x ∈ X|x ∼ y}}\]

      \item It can be shown that the set of equivalence classes on $X$, relative to
      some equivalence relation, is always a \textit{partition} of $X$. For
      example, \textit{eye color} partitions $D_{\type{e}}$ into cells where
      everyone has the same eye color.

    \end{itemize}

\subsubsection{Hamblin sets provide equivalence relations}

\begin{itemize}

    \item As pointed out by Fox, and many others, it's easy to retrieve a
    \textit{partition of the logical space (i.e., the set of possible worlds
    $W$)} from a Hamblin set $Q$, by taking the Hamblin set to provide an
    equivalence relation on $W$.

    \item The idea, informally, is that two worlds $w$ and $w'$ are equivalent
    wrt a question $Q$, iff, each answer in $Q$ is mapped to the same
    truth-value in both worlds.

    \ex
    $w \sim w'\text{ iff }∀p∈Q[p w = p w']$
    \xe

    \item Once we have an equivalence relation, we can compute the set of
    equivalence classes of $W$. For each world in $w$, its equivalence class is
    the set of worlds which map each answer in $Q$ to the same truth-value.

    \item Gathering together the resulting equivalence classes is guaranteed to
    partition the logical space (the set of equivalence classes is always a partition).

    \ex
    \(\ml{PART}_{Q} W ≔ \bigcup\limits_{w'∈ W}\set{w ∈ W|∀p∈Q[p w = p w']}\)
    \xe

    \item Fox gives a simple example of how this works, when the Hamblin set
    simply contains two logically independent propositions:

    \[Q = \set{p,q}\]

    \item \(\ml{PART}_{Q} W\) will map the logical space to the set of
    equivalence classes under the equivalence relation: $w ∼ w'\text{ iff }p w =
    p w' ∧ q w = q w'$. This results in the following partition:

    \[
    \Set{p ∧ ¬q, q ∧ ¬p, p ∧ q, ¬p ∧ ¬q}
    \]

    \item Let's see how this works in a slightly more involved case, by applying this reasoning
    to the Hamblin denotation of a constituent question:

        \[\eval{who arrived} = \set{p|∃ x[p = \set{w|\ml{arrived}_{w} x}]}\]

    \[= \Set{
    \begin{aligned}[c]
      &\set{w|\ml{arrived}_{w} \ml{p}}\\
      &\set{w|\ml{arrived}_{w} \ml{s}}\\
      &\set{w|\ml{arrived}_{w} \ml{p}⊕\ml{s}}
      \end{aligned}
    }\]

    \item The Hamblin set $Q$ provides an equivalence relation -- $w \sim w'$
    iff every proposition of the form $x \ml{arrived}$, maps $w$ and $w'$ to the
    same truth-value.

    \[w \sim w' \text{ iff }∀p ∈ \eval{who arrived}[p w = p w']\]

    \item This equivalence relation can be used to divide up the logical space
    $W$ into equivalence classes/cells, i.e., just those worlds that
    \enquote{agree on} exactly who arrived.

    \[\bigcup_{w' ∈ W}\set{p|p = \set{w ∈ W|∀ p ∈ \eval{who arrived}[p w = p w']}}\]

  \item This gives us the following set:

    \[
    \Set{
    \begin{aligned}[c]
    &\Set{w|\begin{aligned}[c]
         &\ml{arrived}_{w} \ml{p} = 1\\
      &\ml{arrived}_{w} \ml{s} = 0\\
      &\ml{arrived}_{w} \ml{p}⊕\ml{s} = 0
    \end{aligned}},
&\Set{w|\begin{aligned}[c]
         &\ml{arrived}_{w} \ml{p} = 0\\
      &\ml{arrived}_{w} \ml{s} = 1\\
      &\ml{arrived}_{w} \ml{p}⊕\ml{s} = 0
      \end{aligned}}\\
&\Set{w|\begin{aligned}[c]
         &\ml{arrived}_{w} \ml{p} = 1\\
      &\ml{arrived}_{w} \ml{s} = 1\\
      &\ml{arrived}_{w} \ml{p}⊕\ml{s} = 1
      \end{aligned}},
&\Set{w|\begin{aligned}[c]
         &\ml{arrived}_{w} \ml{p} = 0\\
      &\ml{arrived}_{w} \ml{s} = 0\\
      &\ml{arrived}_{w} \ml{p}⊕\ml{s} = 0
      \end{aligned}}
    \end{aligned}
    }
    \]

    \[
    = \Set{\begin{aligned}[c]
        &\ml{only Paul arrived}\\
        &\ml{only Sophie arrived}\\
        &\ml{Paul and Sophie arrived}\\
        &\ml{Nobody arrived}
      \end{aligned}}
    \]

\end{itemize}

\section{Partition by exhaustification}

\subsection{An arithmetic problem arising from question-partition mismatch}

\begin{itemize}

    \item Fox's insight is that, questions are typically answered by uttering a
    proposition in the Hamblin set.

    \item Propositions in the Hamblin set are nevertheless too weak to identify a cell in the
    corresponding partition.

    \pex
    \a Question: Who (of Yasu, Andy, Daniele) is here?
    \a Answer: Andy is here.
    \xe

    \item The proposition \textit{Andy is here} is consistent with multiple cells in
    the corresponding logical partition: (i) \textit{only Andy is here}, (ii) \textit{only
    Andy and Yasu are here}, (iii) \textit{only Andy and Daniele are here}, and (iv)
    \textit{everyone is here}.

    \item However, if we \textit{exhaustify} the proposition \enquote{Andy is
    here}, relative to the alternatives provided by the Hamblin set, then the
    result is the (strengthened) proposition \textit{only Andy is here}.

    \item Fox points out that this gives rise to what he calls an
    \enquote{arithmetic problem} -- since the number of cells in the
    logical partition often outstrips the number of propositions in the answer
    set, then there will often be cells which simply \textit{can't} be
    identified by exhaustifying a member of the Hamblin set.

    \item For example, in the context just given, there is no member in the
    Hamblin set, which, once strengthened, picks out the cell where
    \textit{nobody is here}.

\end{itemize}

\subsection{Dayal's presupposition}

\begin{itemize}

    \item In an extremely influential book, \citet{Dayal} proposed that
    interrogatives carry a particular kind of presupposition -- Fox demonstrates
    that, once this presupposition is adopted, the arithmetic problem no longer
    arises.

    \item contexts which satisfy Dayal's presupposition are exactly those
    where each member of the Hamblin set can pick out a cell of the (contextual)
    partition via exhaustification.

    \item Let's see how this works.

    \pex\textbf{Dayal's presupposition}
    \a \(\ml{Ans}_{D} Q ≔ λ w:∃p∈ Q[p = \ml{MaxInf} Q w] . \ml{MaxInf} Q w\)
    \a \(\ml{MaxInf} Q w ≔ \begin{cases}
      p & p w ∧ ∀q ∈ Q[q w → p ⊆ q]\\
      ♯&\text{else}
      \end{cases}\)
    \xe

    \item If the presupposition of $\ml{Ans}_{D}$ is met, then, given a
    Stalnakerian context set $C$ (representing the common ground), every world
    $w ∈ C$ will be such that $\ml{MaxInf} Q w = p$, for some $p ∈ Q$.

    \item The flip side of the presupposition being satisfied, will be that
    every cell in the partition of $C$ will be identifiable via (the
    strengthened meaning of) each member of the Hamblin set.

    \item Imagine that we have a Hamblin set consisting of two logically
    independent propositions $Q = \set{p,q}$

    \item And, let's say that $C$ initially consists of the following worlds.
    For Dayal's presupposition to be satisfied in the context set, for every
    world, there must be a unique maximally informative element of $Q$.

      \begin{table}[H]
\begin{tabular}{llllll}
 & $w_{p,q}$ & $w_{p,¬q}$ & $w_{¬p,q}$ & $w_{¬p,¬q}$ &  \\
  \hline
$p$ & 1 & 1 & 0 & 0 &  \\
$q$ & 1 & 0 & 1 & 0 &
\end{tabular}
\end{table}

    \begin{itemize}

    \item \xmark\,In $w_{p,q}$, there is no such proposition; $p$ and $q$ are equally informative.

    \item \cmark\,In $w_{p,¬q}$, there is such a proposition: $p$.

    \item \cmark\,In $w_{¬p,q}$, there is such a proposition: $q$.

    \item \xmark\,In $w_{¬p,¬q}$, there is no such proposition; both $p$ and $q$
    are false.

    \end{itemize}

    \item Since Dayal's presupposition isn't met in this context, we must
    \textit{accommodate}, i.e., minimally shrink $C$ such that Dayal's
    presupposition is satisfied.

    \item What is left is the revised context set $C' = \set{w_{p,¬q},w_{¬p,q}}$.

    \item Now, if we partition this revised context set relative to $Q$, the
    result is simply $\set{\set{w_{p,¬q}},\set{w_{¬q,p}}}$. This corresponds
    exactly to what we get if we \textit{exhaustify} each proposition in the
    Hamblin set.

    \item Fox suggests that this provides a more general solution to the
    arithmetic problem raised earlier.

    \ex
    \textbf{Dayal's presupposition as an answer to the arithmetic challenge}\\
    If A is a context set that satisfies the presupposition of $\ml{Ans}_{D} Q$,
    then every cell in $\ml{Part}_{C} Q A$ is identifiable by (the
    exhaustification of) a member of $Q$:\\
   $∀ C ∈ \ml{Part}_{C} Q A[∃p ∈ Q[[\ml{Exh} Q p = C]]]$
    \xe

    \end{itemize}

    \section{Evidence for Dayal's presupposition}

    \subsection{Existence, uniqueness, and maximality}

    \begin{itemize}

        \item Assumption: \textit{wh-}expressions with singular restrictors
        range over atomic individuals \textit{only}.

        \item Consequence: the Hamblin-set denoted by a singular
        \textit{which-}question, such as \enquote{which linguist is here?} will
        be the set of propositions of the form $x$ \textit{is here}, where $x$
        is a linguist.

        \item The propositions in the resulting Hamblin set are logically independent:

        \[
        \eval{which linguist is here?} = \Set{\begin{aligned}[c]
            &\ml{Yasu is here}\\
            &\ml{Dani is here}\\
            &\ml{Andy is here}
          \end{aligned}}
        \]

        \item For Dayal's presupposition to be true in a context set $C$, it
        must be the case that for every world $w ∈ C$, there is some member of
        the Hamblin set that is a unique, maximally informative true answer.

        \item This is equivalent to the requirement that every cell in the
        partition induced by the question by identifiable by an exhaustified
        member of the Hamblin set.

        \item The exhaustified answers -- which correspond to the contextual
        partition, if Dayal's presupposition is met -- are given below:

        \[
        \Set{
        \begin{aligned}[c]
          &\ml{only Yasu is here}&(= \ml{Exh} Q (\ml{Yasu is here}))\\
          &\ml{only Dani is here}&(= \ml{Exh} Q (\ml{Dani is here}))\\
          &\ml{only Andy is here}&(= \ml{Exh} Q (\ml{Andy is here}))
        \end{aligned}
        }
        \]

        \item This means that the presupposition imposed by \(\ml{Ans}_{d}\)
        will only be satisfied in a context where \textit{there exists a unique
        linguist who is here}.

      \item Let's now move on to consider plural and simplex \textit{wh-}expressions.

        \item Assumption: the domain of simplex and plural
        \textit{wh-}expressions is \textit{closed under sum formation}.

        \item Consequence: the Hamblin set denoted by a question such as
        \enquote{which linguists are here} no longer consists of a set of
        logically independent propositions.

        \[
        \eval{which linguists are here} = \Set{\begin{array}{c}
            \ml{here y},\ml{here d},\ml{here a}\\
            \ml{here y}⊕\ml{a},\ml{here y}⊕\ml{d},\ml{here d}⊕\ml{a}\\
            \ml{here }\ml{y}⊕\ml{d}⊕\ml{a}
          \end{aligned}}
        \]

        \item Since \textit{to be here} is a distributive predicate, the answer
        set is, in fact, closed under conjunction:

        \[
        = \Set{\begin{array}{c}
                 \ml{Yasu is here},\ml{Dani is here},\ml{Andy is here}\\
                 \ml{Yasu and Dani are here},\ml{Yasu and Andy are
                 here},\ml{Dani and Andy are here}\\
                 \ml{everyone is here}
                 \end{array}}
        \]

        \item Applying \(λp . \m{Exh} Q p\) to each member of the set results in
        the following set. Note that exhaustification of the answer
        $\ml{everyone is here}$ is vacuous, since there are no excludable alternatives.

                \[
        = \Set{\begin{array}{c}
                 \ml{only Yasu is here},\ml{only Dani is here},\ml{only Andy is here}\\
                 \ml{only Yasu and Dani are here},\\
                 \ml{only Yasu and Andy are
                 here},\\
                 \ml{only Dani and Andy are here}\\
                 \ml{everyone is here}
                 \end{array}}
        \]

        \item The prediction of Dayal's presupposition is that the
        presupposition of this question will only be satisfied in a context
        where either: (i) there is a unique linguist who is here, or (ii) there
        is a unique, maximal group of linguists who are here.

        \item The (accurate) prediction then, is that the following answer
        should convey a maximality inference:

        \pex
        \a Question: Which linguists (out of Yasu, Dani, and Andy) are here?
        \a Answer: Yasu and Andy.\hfill $⇝$ \textit{Dani isn't here}
        \xe

        \item Fox suggests that the same reasoning applies to simplex
        \textit{wh-}questions -- essentially, the suggestion is to treat
        \textit{who} as meaning the same thing as \textit{which people}.

        \item The prediction, then, is that Dayal's presupposition should \enquote{knock
        out} the cell in the logical partition where nobody is here, just as
        with the plural \textit{which-}question.

        \item \citet{elliottNicolaeSauerlandMs} observe that this would
        (erroneously) predict that the question \enquote{who is here?} should be
        a presupposition failure in a context compatible with nobody being here.

        \pex
        \a Question: Who is here?
        \a Answer: Nobody.
        \xe

      \item It's even easier to see this in an embedded context.

        \ex\label{pre}For each day of the week, David knows who was here.
        \xe

        \item If we assume universal projection from out of the scope of
        \textit{each day of the week}, Fox (and Dayal) predict that this should presuppose that, on each
        day of the week, at least
        one person was here.

        \item Our judgement is that (\ref{pre}) is judged felicitous in a
        context where, on, e.g. Tuesday, nobody was here, and David knows that
        nobody was here on Tuesday.

        \item We'll come back to this in our discussion of higher-order readings.

    \end{itemize}

    \subsection{Weak islands}

    \begin{itemize}

        \item The argument from weak islands is based on the observation that,
        depending on the logical properties of the domain that the
        \textit{wh}-expression ranges over, there are certain environments in
        which a question cannot have a maximally informative true answer. In
        such environments, the question is judged to be unacceptable.

        \pex
        \a Tell me how fast you drove?
        \a\ljudge{*}Tell me how fast you didn't drive?
        \xe

       \item Fox's discussion of this contrast is based on \citet{foxHackl2007}.

        \item F\&H assume that the domain of degrees is \textit{densely
        ordered}; consequently degree questions denote infinite sets of
        propositions, densely ordered by entailment.

        \pex
        \a\label{max1}$\eval{how fast did you drive?} = \set{p|∃d∈ D[p = \set{w|\ml{yourSpeed}_{w} ≥ d}]}$
        \a\label{max2}$\eval{how fast didn't you drive?} = \set{p|∃d∈ D[p = \set{w|\ml{yourSpeed}_{w} < d}]}$
        \xe

        \item We won't go into the details here, but the observation is that
        (\ref{max1}) \textit{can} have a unique maximally informative true
        answer, namely the proposition: $\set{w|\ml{yourSpeed}_{w} ≥ d^{*}}$, where
        $d*$ is the addressee's actual driving speed.

        \item (\ref{max2}) can \textit{never} have a maximally informative true
        answer, since, assuming that the domain of degrees is densely ordered,
        there is no smallest degreegreater than $d^{*}$.

    \end{itemize}

        \subsection{Cell Identification}

        \begin{itemize}

            \item Fox note that, if Dayal's presupposition is met, so is the
            constraint \textbf{Cell Identification (CI)}:

            \begin{tcolorbox}
            \textbf{Cell Identification}
            \tcblower
            A question $Q$ and a context-set $A$ meet CI if\\
            $∀C ∈ \ml{Part}_{C} Q A[∃ p ∈ Q[[\ml{Exh} Q p]_{A} = C]]$
            \end{tcolorbox}

         \end{itemize}

         \section{Challenges for Dayal's presupposition}

         \subsection{\textit{Mention some}}

         \begin{itemize}

             \item Not every question presupposes the existence of a
             \textit{unique} maximally-informative true answer.

             \item A notable exception to this is \textit{mention some} (MS)
             questions.

             \pex
             Mary knows where we can get gas in Cambridge.
             \a Mary knows one location where we can get the gas.\hfill
             \textit{mention some}
             \a Mary knows all locations where we can get the gas.\hfill
             \textit{mention all}
             \xe

             \item Dayal's presupposition here seems to demand \textit{too much}
             of the question denotation.

         \end{itemize}

         \subsection{Higher order readings}

         \begin{itemize}

             \item Assumption: \textit{wh-}expressions range over the domain of
             individuals (perhaps closed under sum-formation).

             \item Now, consider the Hamblin set predicted for the following
             modalised question.

             \ex
             \label{spector1}What are we required to read for this class?
             \xe

             \[
             \Set{\begin{aligned}[c]
                 &□ \ml{we read War and Peace}\\
                 &□ \ml{we read Brothers Karamazov}\\
                 &□ \ml{we read War and Peace and Brothers Karamazov}\\
               \end{aligned}}
             \]

             \item Dayal's presupposition predicts that, if there is nothing in
             particular that we are required to read, the question should be undefined.

             \item \citet{spector2007salt} asks us to imagine a context such as
             the following:

             \item \textit{There is no particular thing $x$ s.t. we are required
             to read $x$, but there are still requirements pertaining to
             reading. For example, imagine that it would be sufficient to read
             all the French books or all the Russian books in order to satisfy
             the requirement, but it is up to us which ones we choose}.

             \item He observes that, in such a context, (\ref{spector1}) is askable.

             \item Consider a modalised question under an embedding predicate in
             this context:

             \ex
             Mary knows what we are required to read.
             \xe

             \item Intuitively, this can be true if we are subject to the
             following reading requirement: \textit{we must read all the French
             books, or all the Russian books}, and Mary knows this.

             \item Our standard assumptions -- i.e., quantification over plural
             individuals, plus Dayal's presupposition -- can't derive this.

         \end{itemize}

             \subsubsection{The shift to higher-order quantification}

             \begin{itemize}

                 \item In order to reconcile this reading with Dayal's
                 presupposition, Spector suggests that we allow
                 \textit{wh-}expressions to range over upward-entailing GQs.

                 \item This would result in the following
                 Hamblin-set:\footnote{The set of all upward entailing
                 quantifiers living on $D_{\type{e}}$}

                 \[
                 \Set{
                 \begin{aligned}[c]
                   &□ (\set{\set{w},\set{w,b}} (λx . \ml{we read }x)) & (= w^{↑})\\
                   &□ (\set{\set{w},\set{w,b}} (λx . \ml{we read }x))& (= b^{↑})\\
                   &□ (\set{\set{w,b}} (λx . \ml{we read }x))& (= w^{↑} ∧ b^{↑})\\
                   &□ (\set{\set{w},\set{b},\set{w,b}} (λx . \ml{we read }x))&(=
                   w^{↑} ∨ b^{↑})\\
                   &\textcolor{gray}{□ (\set{∅,\set{w},\set{b},\set{w,b}} (λx . \ml{we read }x))}&\textcolor{gray}{(=
                   ⊤)}
                 \end{aligned}
                 }
                 \]

                 \item Why just the UE generalised quantifiers? If we dispense
                 with this restriction, we'd expect that a modalised question should be
                 felicitous in a context where: \textit{we're not subject to any
                 particular reading requirements, but we're forbidden from
                 reading the German books}:

                 \pex
                 \a Question: What are we required to read?
                 \a Answer: \# We're required to read none of the German books.
                 \xe

                 \item There is a DE quantifier which corresponds to a maximally
                 informative true answer in this context, namely: \textit{none
                 of the German books} (extensionally, the set of sets that
                 contain no German books).

             \end{itemize}

                 \subsubsection{Sensitivity to negative islands}

                 \begin{itemize}

                     \item In a positive context, modalised questions are
                     ambiguous between an (ordinary) reading about individuals,
                     and a higher-order reading.

                     \pex
                     \a What are you required to read for this class?
                     \a War and Peace or Brothers Karamazov.\hfill $□>∨;∨>□$
                     \xe

                     \item Assuming that \textit{wh-}expressions have the
                     capacity to range over GQs, it's surprising that, in a
                     negative question, we don't find a corresponding ambiguity:

                     \pex
                     \a What did you not read for this class?
                     \a\label{ans}War and Peace or Brothers Karamazov.\hfill$*¬>∨;∨>¬$
                     \xe

                     \item Fox notes that, if the unattested reading were
                     available, the answer in (\ref{ans}) should convey no
                     speaker ignorance about what was read. However, (\ref{ans})
                     obligatorily conveys that the speaker doesn't know if they
                     didn't read War and Peace, and doesn't know if they didn't
                     read Brothers Karamazov.

                     \item The answer therefore sounds somewhat odd, or biases
                     an interpretation according to which the speaker has
                     additional information, but doesn't want to share it.

                     \item Much like with degree questions, the presence of a
                     higher modal makes the higher order reading available again
                     -- this parallels the obviation effect we observe in the
                     domain of degree questions.

                     \pex
                     \a What are you not allowed to read for this class?
                     \a War and Peace or Brothers Karamazov.\hfill $¬ > ∨;∨>¬$
                     \xe

                     \ex~
                     How fast are we not allowed to drive?
                     \xe

                     \item Dayal's presupposition however doesn't predict the
                     \textit{un}availability of a higher-order question in a
                     negative context. To see why consider the predicted Hamblin set.

                     \ex
                     What didn't you read?
                     \xe

                     \[
                 \Set{
                 = \begin{aligned}[c]
                   &¬ \ml{you read w}\\
                   &¬ \ml{you read b}\\
                   &¬ \ml{you read w}∧\ml{b}\\
                   &¬ \ml{you read w}∨\ml{b}
                 \end{aligned}
                 }
                 \]

                     \item In a context in which you read neither book, there is
                     a unique, maximally-informative true answer to the question
                     -- namely \textit{you read neither w nor b}. Dayal's
                     presupposition is satisfied.

                     \item If we want a uniform explanation for negative
                     islands, we'd like to extend our explanation to these cases
                     too. The current formulation of Dayal's presupposition
                     doesn't do that.

                 \end{itemize}

                 \begin{tcolorbox}
                   Interim summary
                   \tcblower
                   \begin{itemize}

                       \item Mention some interpretations of questions show that
                       there are cases in which Dayal's presupposition
                       \textit{demands too much}.

                       \item The negative-island sensitivity of higher-order
                       questions shows that there are cases in which Dayal's
                       presupposition \textit{demands too little}.

                    \end{itemize}
                 \end{tcolorbox}

\section{Question-Partition Matching}

\begin{itemize}

    \item Fox notes that we can rule out the unattested higher-order reading in
    a negative context by strengthening Dayal's presupposition. The new
    constraint (which subsumes Dayal's presupposition) is called
    \textbf{Question Partition Matching} (QPM).

    \begin{tcolorbox}
      Question Partition Matching (QPM)
      \tcblower
      A question $Q$ and a context-set $A$ meet QPM if they meet \textbf{Cell
        Identification} (CI) and \textbf{Non-Vacuity} (NV):

      \begin{itemize}

          \item CI: $∀ C ∈ \ml{Part}_{C} Q A[∃p ∈ Q[[\ml{Exh} Q p]_{A} = C]]$
          \item NV: $∀ p ∈ Q[∃C ∈ \ml{Part}_{C} Q A[[\ml{Exh} Q p]_{A} = C]]$

      \end{itemize}

      Informally, this says that: (i) the exhaustification of each answer in the
      Hamblin set must correspond to a cell in the contextual partition, and
      furthermore (ii) each cell in the contextual partition must correspond to
      the exhaustification of an answer in the Hamblin set.
    \end{tcolorbox}

\end{itemize}

\subsection{Addressing undergeneration}

\begin{itemize}

    \item To see how QPM rules out higher-order questions in a negative context,
    consider again the denotation of a negative higher-order question:

    \ex
    What didn't you read?
    \xe

    \[\Set{
                 \begin{aligned}[c]
                   &¬ \ml{you read w}\\
                   &¬ \ml{you read b}\\
                   &¬ \ml{you read w}∧\ml{b}\\
                   &¬ \ml{you read w}∨\ml{b}
                 \end{aligned}
                 }
                 \]

  \item The result of exhaustification is as follows:

     \[\Set{
                 \begin{aligned}[c]
                   &\text{\cmark} (¬ \ml{you read w}) ∧ \ml{you read b}\\
                   &\text{\cmark} (¬ \ml{you read b}) ∧ \ml{you read w}\\
                   &\text{\xmark} (¬ \ml{you read w}∧\ml{b}) ∧ (\ml{you read w} ∨ \ml{b})\\
                   &\text{\cmark} ¬ (\ml{you read w}∨\ml{b})
                 \end{aligned}
                 }
                 \]

    \item The exhaustification of the negative conjunctive answer gives back the
    proposition \textit{that you read War and Peace or Brothers Karamazov, but
    not both}. This can never pick out a cell in a contextual partition -- it
    overlaps with both the first and the second exhaustified answer.

\end{itemize}

    \subsection{Addressing overgeneration}


    \begin{itemize}

        \item Recall that \textit{mention some} interpretations suggested that
        the requirements imposed by Dayal's presupposition were \textit{too
        strong}. Now, we've strengthened Dayal's presupposition to QPM, so the
        problem should be even worse.

        \item Fox suggests that we can address this worry if, rather, then
        formulating $\ml{Exh}$ in terms of \textit{maximal informativity}, we
        use the \textit{innocent inclusion} formulation of $\ml{Exh}$ due to \citet{barLevFox2017}.


    \end{itemize}


    \subsubsection{Background on Innocent Inclusion}

    \begin{itemize}

      \item \textit{Innocent Inclusion} is basically a refinement of Fox's
      innocent exclusion algorithm (which we discussed last week). It has two
      primary virtues:

      \begin{itemize}

      \item it can derive free choice inferences without using
      recursive exhaustification; only one layer of $\ml{Exh}$ is required.

      \item it can derive universal free choice inferences, as in (\Next).

      \end{itemize}

    \pex
Every boy is allowed to eat ice cream or cake.
\a $⇝$ every boy is allowed to eat ice cream.
\a $⇝$ every boy is allowed to eat cake.
    \xe

    \pex~
    No student is required to solve problem A and problem B.
\a $⇝$ No student is required to solve problem A.
    \a $⇝$ No student is required to solve problem B:w
    \xe

 \item Procedure for applying \(\ml{Exh}^{II+IE}\) to a proposition \(p\):

\begin{itemize}
\tightlist
\item
  Negate the set of \emph{innocently excludable} members of
  \(\ml{Alt}(p)\).

  \begin{itemize}
  \tightlist
  \item
    To get the \emph{innocently excludable} members of
    \(\ml{Alt}(p)\), we gather the maximal members of
    \(𝓟(\ml{Alt}(p))\) that can be negated consistently with \(p\),
    and intersect them.
  \end{itemize}
\item
  Assert the set of \emph{innocently includable} members of
  \(\ml{Alt}(p)\).

  \begin{itemize}
  \tightlist
  \item
    To get the \emph{innocently includable} members of
    \(\ml{Alt}(p)\), we gather the maximal members of
    \(𝓟(\ml{Alt}(p))\) that can be asserted consistently with \(p\)
    and the negation of the innocently excludable alternatives, and
    intersect them.
  \end{itemize}
\end{itemize}

    \pex\label{universalFC} Every boy is allowed to eat ice cream or cake.
\hfill \(∀ x ◇(Px ∨ Qx)\)
\a \textit{Every boy is allowed to eat ice cream}. \hfill \(∀ x ◇ Px\)
\a \textit{Every boy is allowed to eat cake}. \hfill \(∀ x ◇ Qx\) \xe

\item First, let's compute the set of alternatives:

\[\begin{aligned}[t]
&\ml{Alt}(∀ x ◇ (Px ∨ Qx))\\
&= \Set{\begin{array}{c}
\overbrace{① ∀ x ◇ (Px ∨ Qx)}^{\text{prejacent}}, \overbrace{② ∀ x ◇ Px,③ ∀ x ◇ Qx}^{\text{universal disjunctive alts}}, \overbrace{④ ∀ x ◇ (Px ∧ Qx)}^{\text{universal conjunctive alt}}\\
\underbrace{⑤ ∃ x ◇ (Px ∨ Qx)}_{\text{existential alt}}, \underbrace{⑥ ∃ x ◇ Px,⑦ ∃ x ◇ Qx}_{\text{existential disjunctive alts}}, \underbrace{⑧ ∃ x ◇ (Px ∧ Qx) }_{\text{existential conjunctive alt}}
\end{array}}
\end{aligned}\]

\item Let's gather together the maximal subsets of
\(\ml{Alt}(∀ x ◇ (Px ∨ Qx))\) that can be negated consistently with
\(∀ x ◇ (Px ∨ Qx))\).

\[\Set{② ∀ x ◇ Px,③ ∀ x ◇ Qx, \alert{④ ∀ x ◇ (Px ∧ Qx)}, \alert{⑧ ∃ x ◇ (Px ∧ Qx) }}\]

\[\Set{② ∀ x ◇ Px,⑥ ∃ x ◇ Px, \alert{④ ∀ x ◇ (Px ∧ Qx)}, \alert{⑧ ∃ x ◇ (Px ∧ Qx)}}\]

\[\Set{③ ∀ x ◇ Qx,⑦ ∃ x ◇ Qx, \alert{④ ∀ x ◇ (Px ∧ qx)}, \alert{⑧ ∃ x ◇ (px ∧ qx)}}\]

    \[
\mathrm{IE}(∀ x ◇ (Px ∨ Qx)) = \Set{{④ ∀ x ◇ (Px ∧ Qx)}, {⑧ ∃ x ◇ (Px ∧ Qx)}}
\]

\item If we take the prejacent together with the negation of the IE
alternatives, we end up in a world where, either all boys are allowed to
P and not Q, all boys are allowed to Q and not P, or some boys are
allowed to P and not Q and some boys are allowed to Q and not P (no boys
are allowed to P and Q), i.e.

\[
∀ x ◇ (Px ∨ Qx) ∧ ¬ ∃ x ◇ (Px ∧ Qx)
\]

\item Let's gather together the maximal subsets of
\(\ml{Alt}(∀ x ◇ (Px ∨ Qx))\) that can be asserted consistently with
\(∀ x ◇ (Px ∨ Qx)) ∧ ¬ ∃ x ◇ (Px ∧ Qx)\).

\item It turns out there is only one such set, consisting of all the non-IE
alts:

\[
\Set{
\begin{array}{c}
① ∀ x ◇ (Px ∨ Qx), ② ∀ x ◇ Px,③ ∀ x ◇ Qx,\\
⑤ ∃ x ◇ (Px ∨ Qx), ⑥ ∃ x ◇ Px,⑦ ∃ x ◇ Qx
\end{array}
}
\]

\item Asserting the II alts together with the prejacent and negation of the IE
alts results in the following enriched meaning:

\[
∀ x ◇ (Px ∨ Qx) \alert{∧ ¬ ∃ x ◇ (Px ∧ Qx) ∧ ∀ x ◇ Px ∧ ∀ x ◇ Qx}
\]


\end{itemize}

\subsubsection{Accounting for \textit{mention some}}

\begin{itemize}

    \item \textbf{Modification 1:} Answers in the Hamblin set are exhaustified via $\ml{Exh}^{IE+II}$.

    \item \textbf{Modification 2:} The answerhood operator returns a set of propositions, rather than a
    unique proposition.

    \item Allowing for the possibility of the \textit{wh-}expression ranging
    over higher order GQs, we get the following denotation for the mention some
    question below:

    \ex
    Where can we buy gas?
    \xe

    \ex
    $= \set{p|∃Q ∈ \ml{UGQ} L[p = ◇ (Q (λ i . \ml{we get gas at }i))]}$
    \xe

    \item Assume that there are three locations $l_{1},l_{2},$ and $l_{3}$.

    \item The result is the following Hamblin set (Fox notes we must assume that
    the conjunctive alternatives are pruned).

    \[= \Set{\begin{array}{c}
               ◇ l_{1}, ◇ l_{2}, ◇ l_{3}\\
               ◇ (l_{1} ∨ l_{2}), ◇ (l_{1} ∨ l_{3}), ◇ (l_{2} ∨ l_{3})\\
               ◇ (l_{1} ∨ l_{2} ∨ l_{3})\\
               \end{array}}\]

    \item Now we exhaustify each member of the set via $\ml{Exh}^{IE+II}$

    \[=\Set{\begin{aligned}[c]
          &◇ l_{1} ∧ ¬ ◇ l_{2} ∧ ¬ ◇ l_{3} &= (\ml{Exh} ◇ Q l_{1})\\
          &◇ l_{2} ∧ ¬ ◇ l_{1} ∧ ¬ ◇ l_{3}&= (\ml{Exh} ◇ Q l_{2})\\
          &◇ l_{3} ∧ ¬ ◇ l_{1} ∧ ¬ ◇ l_{2}&= (\ml{Exh} ◇ Q l_{3})\\
          &◇ l_{1} ∧ ◇ l_{2} ∧ ¬ ◇ l_{3}&=(\ml{Exh} ◇ (l_{1} ∨ l_{2}))\\
          &◇ l_{1} ∧ ◇ l_{3} ∧ ¬ ◇ l_{2}&=(\ml{Exh} ◇ (l_{1} ∨ l_{3}))\\
          &◇ l_{2} ∧ ◇ l_{3} ∧ ¬ ◇ l_{1}&=(\ml{Exh} ◇ (l_{2} ∨ l_{3}))\\
          &◇ l_{2} ∧ ◇ l_{3} ∧ ◇ l_{1}&=(\ml{Exh} ◇ (l_{1} ∨ l_{2} ∨ l_{3}))\\
        \end{aligned}}\]

    \item Note that the exhaustification of each proposition identifies a cell.
   If there is more than one location where one can get gas, the proposition
    that identifies the cell does not mention all the locations.

    \item The final step in accounting for mention-some is to allow the
    answerhood operator to return the set of propositions that entail the
    cell-identifier.

    \ex
    $\ml{Ans} Q = λ w\begin{aligned}[t]
      &:∃p ∈ Q[\ml{Exh} Q p w = 1]\\
      &. \set{q ∈ Q|q w ∧ q ⊆ (ιp ∈
        Q[\ml{Exh} Q p w = 1])}
      \end{aligned}$
    \xe

    \item If there are two locations where we can gas, $l_{1}$ and $l_{2}$, the
    cell-identifier is $◇ (l_{1} ∨ l_{2})$. Applying the reformulated
    answer-hood operator returns the set $\set{◇ l_{1},◇ l_{2}, ◇ (l_{1}∨ l_{2})$

\end{itemize}


\printbibliography

\end{document}

%%% Local Variables:
%%% mode: latex
%%% TeX-master: t
%%% End:
