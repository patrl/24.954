\documentclass[nols,twoside,nofonts,nobib,nohyper]{tufte-handout}

\usepackage{fixltx2e}
\usepackage{tikz-cd}
\usepackage[most]{tcolorbox}
\usepackage{appendix}
\usepackage{listings}
\lstset{language=TeX,
       frame=single,
       basicstyle=\ttfamily,
       captionpos=b,
       tabsize=4,
  }

\input{acronyms}
\renewcommand*{\acsfont}[1]{\textsc{#1}}

\usepackage[font=footnotesize]{caption}

\makeatletter
% Paragraph indentation and separation for normal text
\renewcommand{\@tufte@reset@par}{%
  \setlength{\RaggedRightParindent}{0pt}%
  \setlength{\JustifyingParindent}{0pt}%
  \setlength{\parindent}{0pt}%
  \setlength{\parskip}{\baselineskip}%
}
\@tufte@reset@par

% Paragraph indentation and separation for marginal text
\renewcommand{\@tufte@margin@par}{%
  \setlength{\RaggedRightParindent}{0pt}%
  \setlength{\JustifyingParindent}{0pt}%
  \setlength{\parindent}{0pt}%
  \setlength{\parskip}{\baselineskip}%
}
\makeatother

\usepackage{multicol}
\usepackage{float}
% \usepackage{subcaption}
\usepackage{capt-of}

\setcounter{secnumdepth}{3}

% Set up the spacing using fontspec features
\renewcommand\allcapsspacing[1]{{\addfontfeature{LetterSpace=15}#1}}
\renewcommand\smallcapsspacing[1]{{\addfontfeature{LetterSpace=10}#1}}

\usepackage{amsthm}
\usepackage{diagbox}

\theoremstyle{definition}
\newtheorem{definition}{Definition}[section]

\title{An explanatory theory of presupposition projection}

\author[Patrick D. Elliott \& Danny Fox]{Patrick~D. Elliott \& Danny Fox}

\addbibresource[location=remote]{/home/patrl/repos/bibliography/master.bib}

\lingset{
  belowexskip=-1\baselineskip,
  aboveglftskip=0pt,
  belowglpreambleskip=0pt,
  belowpreambleskip=0pt,
  interpartskip=0pt,
  extraglskip=0pt,
  Everyex={\parskip=0pt}
}

\usepackage{float}


% \usepackage{booktabs} % book-quality tables
% \usepackage{units}    % non-stacked fractions and better unit spacing
% \usepackage{lipsum}   % filler text
% \usepackage{fancyvrb} % extended verbatim environments
%   \fvset{fontsize=\normalsize}% default font size for fancy-verbatim environments

% % Standardize command font styles and environments
% \newcommand{\doccmd}[1]{\texttt{\textbackslash#1}}% command name -- adds backslash automatically
% \newcommand{\docopt}[1]{\ensuremath{\langle}\textrm{\textit{#1}}\ensuremath{\rangle}}% optional command argument
% \newcommand{\docarg}[1]{\textrm{\textit{#1}}}% (required) command argument
% \newcommand{\docenv}[1]{\textsf{#1}}% environment name
% \newcommand{\docpkg}[1]{\texttt{#1}}% package name
% \newcommand{\doccls}[1]{\texttt{#1}}% document class name
% \newcommand{\docclsopt}[1]{\texttt{#1}}% document class option name
% \newenvironment{docspec}{\begin{quote}\noindent}{\end{quote}}% command specification environment

\begin{document}

\maketitle% this prints the handout title, author, and date

\section{Weak Kleene}

\begin{tcolorbox}
  \textbf{Weak Kleene recipe}
  \tcblower
  Where the classical semantics is silent, always return \#.
\end{tcolorbox}

One way of thinking of the third truth value, \#, is as representing \textit{undefinedness}.

This interpretation gives rise to a Weak Kleene logic.

\begin{fullwidth}
  \begin{tcolorbox}[title=Weak Kleene truth-tables]
    \begin{minipage}{.5\linewidth}
      \centering
      \begin{tabular}{cc}
              $ϕ$ & $\mathbf{¬ ϕ}$ \\
              \midrule
              1   & 0              \\
              0   & 1              \\
              \#  & \#
          \end{tabular}
          \captionof{figure}{Negated formulas}
    \end{minipage}
    \begin{minipage}{.5\linewidth}
      \centering
          %
          \begin{tabular}{c|ccc}
              \multicolumn{4}{c}{$\mathbf{ϕ ∧ ψ}$} \\
              \midrule
              \diagbox{$ϕ$}{$ψ$} & 1  & 0  & \#    \\
              \midrule
              1                  & 1  & 0  & \#    \\
              0                  & 0  & 0  & \#    \\
              \#                 & \# & \# & \#
          \end{tabular}
          \captionof{figure}{Conjunctive formulas}
          \end{minipage}
    \begin{minipage}{.5\linewidth}
      \centering
          \begin{tabular}{c|ccc}
              \multicolumn{4}{c}{$\mathbf{ϕ ∨ ψ}$} \\
              \midrule
              \diagbox{$ϕ$}{$ψ$} & 1  & 0  & \#    \\
              \midrule
              1                  & 1  & 1  & \#    \\
              0                  & 1  & 0  & \#    \\
              \#                 & \# & \# & \#
          \end{tabular}
          \captionof{figure}{Disjunctive formulas}
    \end{minipage}
    \begin{minipage}{.5\linewidth}
      \centering
          \begin{tabular}{c|ccc}
              \multicolumn{4}{c}{$\mathbf{ϕ → ψ}$} \\
              \midrule
              \diagbox{$ϕ$}{$ψ$} & 1  & 0  & \#    \\
              \midrule
              1                  & 1  & 0  & \#    \\
              0                  & 1  & 1  & \#    \\
              \#                 & \# & \# & \#
          \end{tabular}
          \captionof{figure}{Conditional formulas}
    \end{minipage}
\end{tcolorbox}
\end{fullwidth}


\section{Strong Kleene (symmetric)}

We can think of the third truth value, \#, as representing \textit{uncertainty whether 1 or 0}, which we can represent as the set $\set{1,0}$.

In order to explain the recipe, it will be helpful to think of our three truth-values as the following isomorphic three-membered set: $\set{\set{1},\set{0},\set{1,0}}$, with $\set{1}$ representing \textit{definitely true}, $\set{0}$ representing \textit{definitiely false}, and $\set{1,0}$ representing \textit{maybe true and maybe false}.

$$
\set{\overbrace{\set{1}}^{\text{true}},\underbrace{\set{0}}_{\text{false}},\overbrace{\set{1,0}}^{\text{uncertain}}}
$$

The intuition behind our recipe will be as follows:

\begin{itemize}
    \item Given a complex formula with an $n$-place truth-functional connective $f$, $⌜f ϕ_{1}…ϕ_{n}⌝$.
  \item Assuming that $I^{bi}$ gives the bivalent interpretation of $f$ as a function, compute $\set{I^{bi}(f) t_{1}\ldots t_{n}|t_{1} ∈ \eval*[tri]{ϕ_{1}},…,t_{n} ∈ \eval*[tri]{ϕ_{n}}}$.
    \item The result is the value of $\eval*[tri]{f ϕ_{1} … ϕ_{n}}$
\end{itemize}

\subsection{Applying the Strong Kleene algorithm to conjunction}

When the values of the arguments of the connective are $\set{1}$ or $\set{0}$, the algorithm will simply deliver the classical semantics. We can illustrate this with conjunction.

$$
\eval*[tri]{p ∧ q} = \set{t ∧ u | t ∈ I(p) ∧ u ∈ I(q)}
$$

If $I(p)$ and $I(q)$ are singleton sets $\set{t}$ and $\set{u}$, this will obviously be equivalent to the classical semantics:

$$
= \set{t ∧ u}
$$

What if one of $I(p)$ is $\set{0,1}$? The value of the conjunctive formula will differ depending on whether $I(q)$ is $\set{1}$ or $\set{0}$. Assuming that $I(q) = \set{1}$:

\pex $I(p) = \set{0,1}, I(q) = 1$
\a $\eval*[tri]{p ∧ q} = \set{t ∧ u | t ∈ \set{0,1} ∧ u ∈ \set{1}}$
\a $= \set{t ∧ 1 | t ∈ \set{0,1}}$
\a $= \set{0,1}$
\xe

\pex~ $I(p) = \set{0,1}, I(q) = 0$
\a $\eval*[tri]{p ∧ q} = \set{t ∧ u | t ∈ \set{0,1} ∧ u ∈ \set{0}}$
\a $= \set{t ∧ 0 | t ∈ \set{0,1}}$
\a $= \set{0}$
\xe

\begin{tcolorbox}
  \textbf{Strong Kleene conjunction}\\
  \tcblower
  \begin{itemize}
      \item $⌜ϕ ∧ ψ⌝$ is \textit{defined} if either (a) $\eval*{ϕ}$ is false, (b) $\eval*{ψ}$ is false, or (c) both $\eval*{ϕ}$ and $\eval*{ϕ}$ are true.
      \item $⌜ϕ ∧ ψ⌝$ is \textit{true} if both $\eval*{ϕ}$ and $\eval*{ψ}$ are true.
      \item $⌜ϕ ∧ ψ⌝$ is \textit{false} if either (a) $\eval*{ϕ}$ is false, or (b) $\eval*{ψ}$ is false.
  \end{itemize}
\end{tcolorbox}

\subsection{Disjunction in Strong Kleene semantics}

\begin{tcolorbox}
  \textbf{Strong Kleene disjunction}\\
  \tcblower
  \begin{itemize}
      \item $⌜ϕ ∨ ψ⌝$ is \textit{defined} if either (a) $\eval*{ϕ}$ is true, (b) $\eval*{ψ}$ is true, or (c) both $\eval*{ϕ}$ and $\eval*{ϕ}$ are false.
      \item $⌜ϕ ∧ ψ⌝$ is \textit{false} if both $\eval*{ϕ}$ and $\eval*{ψ}$ are false.
      \item $⌜ϕ ∧ ψ⌝$ is \textit{true} if either (a) $\eval*{ϕ}$ is true, or (b) $\eval*{ψ}$ is true.
  \end{itemize}
\end{tcolorbox}

\subsection{Strong Kleene truth-tables}

If we apply the strong Kleene algorithm to the classical connectives, substituting in $\set{1,0,\#}$ for $\set{\set{1},\set{0},\set{1,0}}$, the result is the following truth-tables.

\begin{fullwidth}
  \begin{tcolorbox}[title=Strong Kleene truth-tables]
    The highlighted cells differ from weak Kleene.
    \tcblower
    \begin{minipage}{.5\linewidth}
      \centering
      \begin{tabular}{cc}
              $ϕ$ & $\mathbf{¬ ϕ}$ \\
              \midrule
              1   & 0              \\
              0   & 1              \\
              \#  & \#
          \end{tabular}
          \captionof{figure}{Negated formulas}
    \end{minipage}
    \begin{minipage}{.5\linewidth}
      \centering
          %
          \begin{tabular}{c|ccc}
              \multicolumn{4}{c}{$\mathbf{ϕ ∧ ψ}$} \\
              \midrule
              \diagbox{$ϕ$}{$ψ$} & 1  & 0  & \#    \\
              \midrule
              1                  & 1  & 0  & \#    \\
              0                  & 0  & 0  & \hl{0}    \\
              \#                 & \# & \hl{0} & \#
          \end{tabular}
          \captionof{figure}{Conjunctive formulas}
          \end{minipage}
    \begin{minipage}{.5\linewidth}
      \centering
          \begin{tabular}{c|ccc}
              \multicolumn{4}{c}{$\mathbf{ϕ ∨ ψ}$} \\
              \midrule
              \diagbox{$ϕ$}{$ψ$} & 1  & 0  & \#    \\
              \midrule
              1                  & 1  & 1  & \hl{1}    \\
              0                  & 1  & 0  & \#    \\
              \#                 & \hl{1} & \# & \#
          \end{tabular}
          \captionof{figure}{Disjunctive formulas}
    \end{minipage}
    \begin{minipage}{.5\linewidth}
      \centering
          \begin{tabular}{c|ccc}
              \multicolumn{4}{c}{$\mathbf{ϕ → ψ}$} \\
              \midrule
              \diagbox{$ϕ$}{$ψ$} & 1  & 0  & \#    \\
              \midrule
              1                  & 1  & 0  & \#    \\
              0                  & 1  & 1  & \hl{1}    \\
              \#                 & \hl{1} & \# & \#
          \end{tabular}
          \captionof{figure}{Conditional formulas}
    \end{minipage}
\end{tcolorbox}
\end{fullwidth}

    \subsection{The role of linear order in presupposition projection}

    As discussed by \citet{Schlenker2008}, it's not clear that the projection generalization we've been assuming for disjunctive sentences is correct.

    \pex\label{ex:disj}
    \a Either this house has no bathroom, or the bathroom is upstairs.\label{ex:disj-a}
    \a Either the bathroom is upstairs, or (else) this house has no bathroom.\\
    \phantom{,}\hfill\citep[p.\,185]{Schlenker2008}\label{ex:disj-a}
    \xe

    The entry we gave for disjunction in propositional update semantics (after \citealt{Beaver2001}) can capture the projection pattern illustrated by (\ref{ex:disj-a}), but not (\ref{ex:disj-b}).

    This is because, in update semantics, when we compute $c[ϕ ∨ ψ]$, the first disjunct $ϕ$ updates the global input context $c$, but the second disjunct updates a modified context $c[¬ ϕ]$.\sidenote{As a reminder, here's the semantics for disjunctive formulas in propositional update semantics:

      \begin{itemize}
          \item $c[ϕ ∨ ψ] = c[ϕ] ∪ c[¬ ϕ][ψ]$
      \end{itemize}

    }

    \citeauthor{Schlenker2008} suggests that there is a similar problem involving conditional sentences.

    \pex
    \a If this house has a bathroom, then the bathroom is well hidden.
    \a If the bathroom is well hidden, then this house has a bathroom.\\
    \phantom{,}\hfill\citep[p.\,186]{Schlenker2008}
    \xe

    He furthermore suggests that post-posing the antecedent makes no difference to the judgements.

    \pex
    \a The bathroom is well hidden, if this house has a bathroom.
    \a Mary's doctor knows she is expecting a child, if she is pregnant.\\
    \phantom{,}\hfill\citep[p.\,186]{Schlenker2008}
    \xe

\section{Middle Kleene/Peters (asymmetric)}

As we've seen, presupposition projection displays asymmetries based on \textit{linear order}; something that strong Kleene fails to capture.

We need to adjust the strong Kleene algorithm to account for ordering asymmetries.

\textbf{George's intuition:} currying, motivated by considerations of compositionality, imposes an asymmetry between arguments based on evaluation order. Strong Kleene can be modified to be sensitive to evaluation order.

\begin{fullwidth}
  \begin{tcolorbox}[title=Middle Kleene truth-tables]
    N.b. the highlighted cells diverge from strong Kleene.
    \tcblower
    \begin{minipage}{.5\linewidth}
      \centering
      \begin{tabular}{c|c}
              $ϕ$ & $\mathbf{¬ ϕ}$ \\
              \midrule
              1   & 0              \\
              0   & 1              \\
              \#  & \#
          \end{tabular}
          \captionof{figure}{Negated formulas}
    \end{minipage}
    \begin{minipage}{.5\linewidth}
      \centering
          %
          \begin{tabular}{c|ccc}
              \multicolumn{4}{c}{$\mathbf{ϕ ∧ ψ}$} \\
              \midrule
              \diagbox{$ϕ$}{$ψ$} & 1  & 0  & \#    \\
              \midrule
              1                  & 1  & 0  & \#    \\
              0                  & 0  & 0  & 0    \\
              \#                 & \# & \hl{\#} & \#
          \end{tabular}
          \captionof{figure}{Conjunctive formulas}
          \end{minipage}
    \begin{minipage}{.5\linewidth}
      \centering
          \begin{tabular}{c|ccc}
              \multicolumn{4}{c}{$\mathbf{ϕ ∨ ψ}$} \\
              \midrule
              \diagbox{$ϕ$}{$ψ$} & 1  & 0  & \#    \\
              \midrule
              1                  & 1  & 1  & 1    \\
              0                  & 1  & 0  & \#    \\
              \#                 & \hl{\#} & \# & \#
          \end{tabular}
          \captionof{figure}{Disjunctive formulas}
    \end{minipage}
    \begin{minipage}{.5\linewidth}
      \centering
          \begin{tabular}{c|ccc}
              \multicolumn{4}{c}{$\mathbf{ϕ → ψ}$} \\
              \midrule
              \diagbox{$ϕ$}{$ψ$} & 1  & 0  & \#    \\
              \midrule
              1                  & 1  & 0  & \#    \\
              0                  & 1  & 1  & 1    \\
              \#                 & \hl{\#} & \# & \#
          \end{tabular}
          \captionof{figure}{Conditional formulas}
    \end{minipage}
\end{tcolorbox}
\end{fullwidth}

\printbibliography

\begin{appendices}

  \section{Strong Kleene as an alternative semantics}

  Assume that $\type{t}$ is the type of \textit{bivalent} truth values.

  \ex
  $\eval[w]{stopped smoking} = λ x . \begin{cases}
    \set{1}&\ml{smoked}_{w} x ∧ ¬ (\ml{smokes}_{w} x)\\
    \set{0}&\ml{smoked}_{w} x ∧ \ml{smokes}_{w} x\\
    \set{1,0}&\text{otherwise}
    \end{cases}$\hfill$\type{e → \set{t}}$
  \xe

  \ex
  $\eval[w]{and} ≔ λ u . λ t . t ∧ u$\hfill$\type{t → t → t}$
  \xe

  As in a standard alternative semantics, we just need two truth-values to massage composition.

  \pex
  \a $x^{ρ} ≔ \set{x}$\hfill$\type{a → \set{a}}$
  \a $m ⊛ n ≔ \set{x \ml{A} y | x ∈ m ∧ y ∈ n}$\hfill$\type{\set{a → b} → \set{a} → \set{b}}$
  \xe

  \ex
  Paul smoked and he stopped smoking.
  \xe

  \begin{forest}
    [{$\set{t ∧ u | t ∈ \eval[w,\rho]{Paul smoked} ∧ u ∈ \eval[w]{Paul stopped smoking}}$\\$⊛$}
    [{$\begin{cases}
        \set{1}&\ml{smoked}_{w} \ml{p}\\
        \set{0}&\text{otherwise}
        \end{cases}$} [{$1$ iff Paul smoked in $w$} [{Paul smoked},roof]]]
      [{$⊛$}
        [{$\set{λ ut . t ∧ u}$\\and$^{ρ}$}]
        [{$\begin{cases}
    \set{1}&\ml{smoked}_{w} p ∧ ¬ (\ml{smokes}_{w} p)\\
    \set{0}&\ml{smoked}_{w} p ∧ \ml{smokes}_{w} p\\
    \set{1,0}&\text{otherwise}
    \end{cases}$}
        [{Paul stopped smoking},roof]]
      ]
    ]
  \end{forest}

\subsection{Strong Kleene for quantifiers}

\ex
$\set{(a → b) → c} → (a → \set{b}) → \set{c}$
\xe

\subsection{Middle Kleene as an alternative semantics}

\todo[inline]{Fill this is.}


\end{appendices}

\end{document}
