\documentclass[cronos,landscape,paper=letter]{ling-handout}

\usepackage[margin=0.5in]{geometry}

\usepackage{fontawesome,calc,enumitem,parskip}

\lingset{
  belowexskip=0pt,
  aboveglftskip=0pt,
  belowglpreambleskip=0pt,
  belowpreambleskip=0pt,
  interpartskip=0pt,
  extraglskip=0pt,
  Everyex={\parskip=0pt},
}

\addbibresource[location=remote]{/home/patrl/repos/bibliography/elliott_mybib.bib}

% \renewcommand*{\sectionformat}{}
% \renewcommand*{\subsectionformat}{}
% \renewcommand*{\subsubsectionformat}{}

\title{The satisfaction theory}
\subtitle{\texttt{24.954:} Pragmatics in Linguistic Theory}

\date{\today}

\author{Patrick~D. Elliott}

\begin{document}

\maketitle

\section{Introduction}

\subsection{Readings}

\begin{itemize}

    \item \fullcite{heim1983}

    \item Kadmon (1990) \textit{Formal Pragmatics}, chapter 6.

    \item \todo[inline]{Rothschild 2011?}

    \item \todo[inline]{Get them to read some Chierchia}

\end{itemize}

\subsection{Notational conventions}

\begin{itemize}

    \item \(λxy . …\) is to be read as \(λ x . λ y . …\)

    \todo[inline]{Be precise about other things here}

\end{itemize}

\section{Dynamic Semantics}

\subsection{Context Change Potentials}

A Context Change Potential (CCP) is a (possibly partial) function over contexts. For example, the CCP associated with the utterance \enquote{Paul vapes} is:

\ex
\label{ccp1}\(λ c . λ w . c w ∧ \ml{p vapes}_{w}\)
\xe

Assuming that sentences denote \textit{propositions}, how do we get from an ordinary (static) sentential meaning, to a CCP? Let's define an operator \textsc{Assert} (\(𝔸\)) to accomplish this for us:

\ex \textsc{Assert} (first attempt)\\
\(𝔸 p ≔ λ c . λ w . c w ∧ p w\)
\xe

\ex~
\(𝔸 (\eval{Paul vapes}) = (\ref{ccp1})\)
\xe

Suppose that the context set contains both worlds in which Paul vapes, and worlds in which he doesn't, representing a state of ignorance wrt whether or not Paul vapes, i.e.:

\ex
\(c_{1} = λ w . \ml{p vapes}_{w} ∨ ¬ \ml{p vapes}_{w}\)
\xe

We update the context set \(c_{1}\) with the CCP associated with \enquote{Paul vapes} by \textit{applying} the CCP to the context set. The result is an \textit{updated} context set which only includes worlds in which paul vapes.

\ex
\([λ c . λ w . c w ∧ \ml{p vapes}_{w}] c_{1} = λ w . \ml{p vapes}_{w}\)
\xe

Note that we can always get back from a CCP to a proposition. Let's define an operator \(↓\) to do this for us:

\todo[inline]{Check that this works out}

\ex
\(d^{↓} ≔ λ w' . (d (λ w . w = w')) = (λ w . w = w')\)
\xe

Before moving on to the analysis of presupposition in FCS, let's consider how we might analyse conjunction/discourse sequencing within this framework.

Consider the discourse in (\ref{discourse1}). Suppose that at \(c_{1}\), the participants are ignorant about both whether Paul vapes and whether Sophie smokes. At \(c_{2}\) the participants know that Paul vapes, but are ignorant about whether Sophie smokes. At \(c_{3}\), the participants know both that Paul vapes and that Sophie smokes.

\ex\label{discourse1}
\(c_{1}\) Paul vapes \(c_{2}\); Sophie smokes \(c_{3}\).
\xe

We can give a \textit{semantics} of the discourse sequencing operator \((;)\) which captures the intuition that a sequential utterances successively update the context set. To sequence CCPs, we simply \textit{compose} them.

\ex
\((;) ≔ λ q . λ p . q ∘ p\)
\xe

\ex~
\begin{forest}
  [{\(λ c . [𝔸 \eval{Sophie smokes}] ([𝔸 \eval{Paul vapes}] c)\)}
    [{...} [{Paul vapes},roof]]
    [{...}
      [{\(;\)}]
      [{...} [{Sophie smokes},roof]]
    ]
  ]
\end{forest}
\xe

As it stands, the CCPs we've dealt with so far have been \textit{total} functions over contexts. Furthermore, CCPs are defined in terms of conjunction, and since conjunction is associative:

\ex
Paul vapes; Sophie smokes = Sophie smokes; Paul vapes.
\xe

In fact, if we lower the result of sequencing the two sentences, we just get...classical conjunction. So what have we achieved exactly?

\todo[inline]{Get them to demonstrate this in the exercises.}

\subsection{Presupposition satisfaction}

\citeauthor{heim1983}'s (\citeyear{heim1983}) innovation was to treat presuppositions as, essentially, \textit{definedness conditions on CCPs}. We say that, if a context set \(c\) entails the presupposition of a sentence S, \(c\) \textbf{satisfies} the presupposition S.\footnote{You can also write definedness conditions on functions using \citeauthor{heimKratzer1998}'s colon notation, although it will quickly get quite unwieldy. For example:

  \ex
  \(λ c . c ⊆ \set{w|\ml{p used-to-vape}_w} . λ w . c w ∧ ¬ \ml{p vapes}_{w}\)
  \xe
}

We can now shift gears and treat the CCPs associated with sentences as (potentiall!) partial functions from contexts to contexts. The CCP associated with the presuppositional sentence \enquote{Paul quit vaping} is given below:

\ex
\(λ c . \begin{cases}
  λ w . c w ∧ ¬ \ml{p vapes}_{w} & c ⊆ \set{w | \ml{p used-to-vape}_{w}}\\
  ♯&\ml{else}
  \end{cases}\)
\xe

Now that CCPs can be \textit{partial}, treating discourse sequencing as function composition has an interesting consequence.

\subsection{Quantification}

\subsubsection{Assignment functions and variables}

In order to deal with anaphora, \citeauthor{heim1983}'s dynamic semantics deals with an enriched notion of context set -- namely, a \textit{file}. A file is a set of \textit{world-assignment function pairs} (or equivalently, a relation between worlds and assignments).

The file associated with the information that a person at identifier \(7\) vapes is:

\ex
\(\set{⟨g,w⟩ : g_{7} \ml{vapes}_{w}}\)
\xe

\ex~ Definition of a \textit{file}:\\
\(c\) is a file iff there is a subset \(ℕ'\) of \(ℕ\), and \(c\) is a set of assignment-world pairs, \(⟨g,w⟩\), where \(g: ℕ' ↦ D_{\ml{e}}\). We refer to \(ℕ'\) as the \textit{domain} of the file (i.e., \(\ml{dom} c\))
\xe

We'll use variables named \(c,c',c''…\) to range over \textit{files}. \(λ c . …\) can be read as an abbreviation of \(λ c :\ml{file} c . …\)

\(g[n]g'\) mean that \(g\) and \(g'\) differ only at \(n\), and that \(g_{n}\) is defined (\(g'_{n}\) may be undefined).

We can now give a CCP for a sentence with an indefinite:

\ex
Someone\(^{7}\) arrived = \(λ c . \set{⟨g,w⟩|∃⟨g',w⟩ ∈ c[g[7]g' ∧ g_{7} \ml{arrived}_{w}]}\)
\xe


\printbibliography

\end{document}

%%% Local Variables:
%%% TeX-engine: xetex
%%% TeX-master: t
%%% End:
