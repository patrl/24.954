\documentclass[cronos,landscape,paper=letter]{ling-handout}

\usepackage[margin=0.5in]{geometry}

\usepackage{fontawesome,calc,enumitem,parskip,tcolorbox}

\lingset{
  belowexskip=0pt,
  aboveglftskip=0pt,
  belowglpreambleskip=0pt,
  belowpreambleskip=0pt,
  interpartskip=0pt,
  extraglskip=0pt,
  Everyex={\parskip=0pt},
}

\addbibresource[location=remote]{/home/patrl/repos/bibliography/elliott_mybib.bib}

% \renewcommand*{\sectionformat}{}
% \renewcommand*{\subsectionformat}{}
% \renewcommand*{\subsubsectionformat}{}

\title{File Change Semantics}
\subtitle{\texttt{24.954:} Pragmatics in Linguistic Theory}

\date{\today}

\author{Patrick~D. Elliott}

\begin{document}

\maketitle

\section*{Readings}

% \subsection{Readings}

\begin{itemize}

    \item \fullcite{heim1983}

    \item \fullcite[chapter 6]{kadmon2001}

    \item \fullcite{rothschild2011}

\end{itemize}

\section{Sentences as \textit{updates}}

Recall the Heim/Karttunen generalisation concerning presupposition projection in \textit{conjunctive} sentences.

\ex
\textbf{Conjunction}\\If A\(_π\), and B\(_ρ\), then a sentence of the form \enquote{A and B} presupposes \(π\), and unless A entails \(ρ\), also presupposes \(ρ\)
\xe

\ex~\label{conj1}
Sam and Ka visited Rome and Venice last Summer, and Ka visited Venice again.\\
\textit{presuppositionless}
\xe

\begin{tcolorbox}
  Peter's intuition
  \tcblower
  An assertion of the first conjunct \textit{Sam and Ka visited Rome and Venice last Summer} should alter the common ground in such a way that an assertion of the second conjunct is felicitous (heavily paraphrased).
\end{tcolorbox}

One way of thinking about this intuition is that we should treat (\ref{conj1}) in a way parallel to the discourse in (\ref{conj2}).

\ex\label{conj2}
Sam and Ka visited Rome and Venice last Summer.\\
Ka visited Venice again.\hfill\textit{presuppositionless}
\xe

It's difficult to make sense of this idea given the theory introduced last week, since \textit{assertion} is a strictly \textit{pragmatic} notion.

\begin{itemize}

  \item In what sense can we say that each conjunct is a distinct assertion?

  \item What is the meaning of \textit{and} such that it can conjoin assertions?

\end{itemize}

In the 80s and early 90s, a family of theories, including Discourse Representation Theory (DRT; \citealt{kamp1981}), File Change Semantics (FCS; \citealt[ch.\,3]{heim1982}), and Dynamic Predicate Logic (DPL; \citealt{groenendijk_dynamic_1991}) attempted to cash out this intuition.

For reasons of time, we'll be focusing on the tradition begun by \citeauthor{heim1982}.

The core idea of FCS is that the denotation of a declarative sentence is not a \textit{proposition} but rather an \textit{instruction} for updating the conversational context -- such instructions are called \textbf{Context Change Potentials (CCPs)}.

\subsection{The Stalnakerian Common Ground (recap)}

Before we say something about what a CCP is, we first need to be precise about the notion of a \textit{conversational context}.

Following Stalnaker, we'll represent the Common Ground in \(c\) as the \textit{Context Set} -- the set of possible worlds compatible with the shared knowledge of the discourse participants.

\todo[inline]{Add actual Stalnaker references here}

\todo[inline]{Elaborate on this}

\subsection{The CCPs of declarative sentences}

The context set is a set of worlds (type \(\ml{st}\)). The denotation of a declarative sentence is an \textit{instruction} to take the current context set, and sift out all those possible worlds that aren't compatible with the information conveyed by a sentence.

We can model this \enquote{instruction} formally by treating the denotation of a declarative sentence as a \textit{function from context sets to updated context sets} of type \(\ml{⟨st,st⟩\}\).

  \ex
  \(\eval{Paul vapes} = λ c . c ∩ \set{w|\ml{p vapes}_{w}}\)
  \xe

    This meaning for \textit{Paul vapes} captures the dynamic flow of information over the course of a discourse.

    Imagine, we're in a context where we don't know whether or not Paul vapes:

    \[c = \set{w_{1},w_{2},w_{3},w_{4}}\]

    Paul vapes in \(w_{1}\) and \(w_{3}\), but not in \(w_{2}\) or \(w_{4}\)

    When we update a context set with a sentence, we simply \textit{apply} the sentence meaning to the context set:

    \[\eval{Paul vapes} (\set{w_{1},\ldots,w_{4}}) = \set{w_{1},…,w_{4}} ∩ \set{w | \ml{p vapes}_{w}} = \set{w_{1},w_{3}}\]

    The result is an \textit{updated} context set \(c'\) containing \textit{just those worlds} in which Paul vapes.

  \begin{tcolorbox}
    An aside on notation
    \tcblower
    Here, I'm following, e.g., \citet{chierchia_dynamics_1995} in using the lambda notation for CCPs. CCPs are also often written as follows:

    \ex
    \(c + [\text{Paul vapes}] = c ∩ \set{w|\ml{p vapes}_{w}}\)
    \xe

    \ex
    \(c[\text{Paul vapes}] = c ∩ \set{w|\ml{p vapes}_{w}}\)
    \xe
    \linebreak
    These different ways of writing CCPs are equivalent. The lambda notation has the advantage of already being familiar from, e.g., \citet{heimKratzer1998}.
  \end{tcolorbox}

  \subsection{From CCPs to propositions and back again}

  Our classical, static semantics is \textit{subsumed} by this new treatment of sentence meaning, since we can define an operator \((↓)\) to get back from CCPs to propositions.

  To retrieve a proposition from a CCP \(f\), we take the set of worlds \(w\), such that applying \(f\) to \(\set{w}\) returns \(w\).

  \ex
  \(f^{↓} = λ w . f (\set{w}) = \set{w}\)
  \xe

  \ex~
  \(\eval{Paul vapes}^{↓} \begin{aligned}[t]
    &= λ w . (\set{w} ∩ \set{w|\ml{p vapes}_{w}}) = \set{w}\\
    &= λ w . \set{w|\ml{p vapes}_{w}}
    \end{aligned}\)
  \xe

  \begin{tcolorbox}
    Exercise
    \tcblower
    Define an operator \(𝔸 ∷ ⟨\ml{st},⟨\ml{st},\ml{st}⟩⟩\) which takes a classical proposition and returns the corresponding CCP.
  \end{tcolorbox}

  % We can also define an operator \(𝔸\) (Assert), that takes a proposition and returns the corresponding CCP:

  % \ex
  % \(𝔸 p = λ c . c ∩ p\)
  % \xe

  \subsection{Modelling presuppositions}

  Heim's intuition is that presuppositions impose \textit{preconditions} for CCPs to update (i.e., apply to) the current context set.

  If these preconditions are met, we say that the presuppositions of a given CCP are \textbf{satisfied} relative to a context set \(c\).

  We can easily cash out this intuition formally by treating CCPs as \textit{partial} functions from context sets -- an utterance is infelicitous if the associated CCP is undefined when applied to the current context set.

  \ex
  \(\eval{Paul quit vaping} = λ c . \begin{cases}
    c ∩ \set{w| ¬ \ml{p vapes now}_{w} } &\set{w | ¬ \ml{p did vape}_{w}} ⊆ c\\
    ♯ & \text{else}
    \end{cases}\)
  \xe

  The CCP associated with \textit{Paul quit vaping} imposes as a precondition, that the current context \(c\) entails \textit{that Paul used to smoke}.

  If this precondition is satisfied, it updates \(c\) with the information that \textit{Paul doesn't smoke now}, otherwise the result is undefined (and therefore: infelicitous).

  \begin{tcolorbox}
    Writing partial functions
    \tcblower
    The following is to be read as: that function from \(x\) to \textit{output}, which is defined iff \textit{condition} holds.

    \[λ x . \begin{cases}
        \text{\it output} &\text{\it condition}\\
        ♯&\text{else}
      \end{cases}\]

    You can also use the colon notation introduced in \citet{heimKratzer1998}:

    \[λ x:\text{\it condition} . \text{\it output}\]
  \end{tcolorbox}

  \section{Presupposition projection}

  \subsection{Conjunctive sentences}

  \todo[inline]{Conjunction is just function composition}


% \subsection{Notational conventions}

% \begin{itemize}

%     \item \(λxy . …\) is to be read as \(λ x . λ y . …\)

%     \todo[inline]{Be precise about other things here}

% \end{itemize}

% \section{Dynamic Semantics}

% \subsection{Context Change Potentials}

% A Context Change Potential (CCP) is a (possibly partial) function over contexts. For example, the CCP associated with the utterance \enquote{Paul vapes} is:

% \ex
% \label{ccp1}\(λ c . λ w . c w ∧ \ml{p vapes}_{w}\)
% \xe

% Assuming that sentences denote \textit{propositions}, how do we get from an ordinary (static) sentential meaning, to a CCP? Let's define an operator \textsc{Assert} (\(𝔸\)) to accomplish this for us:

% \ex \textsc{Assert} (first attempt)\\
% \(𝔸 p ≔ λ c . λ w . c w ∧ p w\)
% \xe

% \ex~
% \(𝔸 (\eval{Paul vapes}) = (\ref{ccp1})\)
% \xe

% Suppose that the context set contains both worlds in which Paul vapes, and worlds in which he doesn't, representing a state of ignorance wrt whether or not Paul vapes, i.e.:

% \ex
% \(c_{1} = λ w . \ml{p vapes}_{w} ∨ ¬ \ml{p vapes}_{w}\)
% \xe

% We update the context set \(c_{1}\) with the CCP associated with \enquote{Paul vapes} by \textit{applying} the CCP to the context set. The result is an \textit{updated} context set which only includes worlds in which paul vapes.

% \ex
% \([λ c . λ w . c w ∧ \ml{p vapes}_{w}] c_{1} = λ w . \ml{p vapes}_{w}\)
% \xe

% Note that we can always get back from a CCP to a proposition. Let's define an operator \(↓\) to do this for us:

% \todo[inline]{Check that this works out}

% \ex
% \(d^{↓} ≔ λ w' . (d (λ w . w = w')) = (λ w . w = w')\)
% \xe

% Before moving on to the analysis of presupposition in FCS, let's consider how we might analyse conjunction/discourse sequencing within this framework.

% Consider the discourse in (\ref{discourse1}). Suppose that at \(c_{1}\), the participants are ignorant about both whether Paul vapes and whether Sophie smokes. At \(c_{2}\) the participants know that Paul vapes, but are ignorant about whether Sophie smokes. At \(c_{3}\), the participants know both that Paul vapes and that Sophie smokes.

% \ex\label{discourse1}
% \(c_{1}\) Paul vapes \(c_{2}\); Sophie smokes \(c_{3}\).
% \xe

% We can give a \textit{semantics} of the discourse sequencing operator \((;)\) which captures the intuition that a sequential utterances successively update the context set. To sequence CCPs, we simply \textit{compose} them.

% \ex
% \((;) ≔ λ q . λ p . q ∘ p\)
% \xe

% \ex~
% \begin{forest}
%   [{\(λ c . [𝔸 \eval{Sophie smokes}] ([𝔸 \eval{Paul vapes}] c)\)}
%     [{...} [{Paul vapes},roof]]
%     [{...}
%       [{\(;\)}]
%       [{...} [{Sophie smokes},roof]]
%     ]
%   ]
% \end{forest}
% \xe

% As it stands, the CCPs we've dealt with so far have been \textit{total} functions over contexts. Furthermore, CCPs are defined in terms of conjunction, and since conjunction is associative:

% \ex
% Paul vapes; Sophie smokes = Sophie smokes; Paul vapes.
% \xe

% In fact, if we lower the result of sequencing the two sentences, we just get...classical conjunction. So what have we achieved exactly?

% \todo[inline]{Get them to demonstrate this in the exercises.}

% \subsection{Presupposition satisfaction}

% \citeauthor{heim1983}'s (\citeyear{heim1983}) innovation was to treat presuppositions as, essentially, \textit{definedness conditions on CCPs}. We say that, if a context set \(c\) entails the presupposition of a sentence S, \(c\) \textbf{satisfies} the presupposition S.\footnote{You can also write definedness conditions on functions using \citeauthor{heimKratzer1998}'s colon notation, although it will quickly get quite unwieldy. For example:

%   \ex
%   \(λ c . c ⊆ \set{w|\ml{p used-to-vape}_w} . λ w . c w ∧ ¬ \ml{p vapes}_{w}\)
%   \xe
% }

% We can now shift gears and treat the CCPs associated with sentences as (potentiall!) partial functions from contexts to contexts. The CCP associated with the presuppositional sentence \enquote{Paul quit vaping} is given below:

% \ex
% \(λ c . \begin{cases}
%   λ w . c w ∧ ¬ \ml{p vapes}_{w} & c ⊆ \set{w | \ml{p used-to-vape}_{w}}\\
%   ♯&\ml{else}
%   \end{cases}\)
% \xe

% Now that CCPs can be \textit{partial}, treating discourse sequencing as function composition has an interesting consequence.

% \subsection{Quantification}

% \subsubsection{Assignment functions and variables}

% In order to deal with anaphora, \citeauthor{heim1983}'s dynamic semantics deals with an enriched notion of context set -- namely, a \textit{file}. A file is a set of \textit{world-assignment function pairs} (or equivalently, a relation between worlds and assignments).

% The file associated with the information that a person at identifier \(7\) vapes is:

% \ex
% \(\set{⟨g,w⟩ : g_{7} \ml{vapes}_{w}}\)
% \xe

% \ex~ Definition of a \textit{file}:\\
% \(c\) is a file iff there is a subset \(ℕ'\) of \(ℕ\), and \(c\) is a set of assignment-world pairs, \(⟨g,w⟩\), where \(g: ℕ' ↦ D_{\ml{e}}\). We refer to \(ℕ'\) as the \textit{domain} of the file (i.e., \(\ml{dom} c\))
% \xe

% We'll use variables named \(c,c',c''…\) to range over \textit{files}. \(λ c . …\) can be read as an abbreviation of \(λ c :\ml{file} c . …\)

% \(g[n]g'\) mean that \(g\) and \(g'\) differ only at \(n\), and that \(g_{n}\) is defined (\(g'_{n}\) may be undefined).

% We can now give a CCP for a sentence with an indefinite:

% \ex
% Someone\(^{7}\) arrived = \(λ c . \set{⟨g,w⟩|∃⟨g',w⟩ ∈ c[g[7]g' ∧ g_{7} \ml{arrived}_{w}]}\)
% \xe


\printbibliography

\end{document}

%%% Local Variables:
%%% TeX-engine: xetex
%%% TeX-master: t
%%% End:
