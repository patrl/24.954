\documentclass[cronos,landscape,paper=letter]{ling-handout}

\usepackage[margin=0.5in]{geometry}
\usepackage{fontawesome,calc,enumitem,parskip,tcolorbox}

\lingset{
  belowexskip=0pt,
  aboveglftskip=0pt,
  belowglpreambleskip=0pt,
  belowpreambleskip=0pt,
  interpartskip=0pt,
  extraglskip=0pt,
  Everyex={\parskip=0pt},
}

\addbibresource[location=remote]{/home/patrl/repos/bibliography/elliott_mybib.bib}

% \renewcommand*{\sectionformat}{}
% \renewcommand*{\subsectionformat}{}
% \renewcommand*{\subsubsectionformat}{}

\title{File Change Semantics}
\subtitle{\texttt{24.954:} Pragmatics in Linguistic Theory}

\date{September 13, 2019}

\author{Patrick~D. Elliott}

\begin{document}

\maketitle

\section*{Readings}

% \subsection{Readings}
%
The following readings can be found on the Stellar site.

Strongly recommended:

\begin{itemize}

    \item \fullcite{heim1983}

    \item \fullcite[chapter 6]{kadmon2001}

\end{itemize}

Optional:

\begin{itemize}

    \item \fullcite{rothschild2011}

\end{itemize}

\section*{Recap}

Last week we gave an overview of presupposition and the issues surrounding its analysis, including:
\begin{itemize}
  \item The projection problem.
  \item Presupposition accommodation.
    \item The triggering problem.
\end{itemize}

We also presented a take on one of the earliest treatments of presupposition in the literature -- the \textit{multidimensional theory} (\citealt{karttunenPeters1979}).

Remember, we treated presuppositions as a \textit{seperate dimension of meaning}; we took presuppositional meanings to be \textit{pairs} of ordinary semantic values and presuppositions, which we wrote as:

\[\displaystyle\frac{\text{\it presupposition}}{\text{\it assertion}}\]

This theory of presupposition had some fatal flaws, most crucial \textit{the binding problem}. We were unable to give a satisfactory analysis for:

\ex
Someone quit smoking.
\xe

This was one of the problems that motivated the move away from the multidimensional theory. There were two other problems, that are inherited to a certain extent by the account we'll introduce today (although it has other virtues). This is worth bearing in mind.

\begin{description}
  \item[The proviso problem] we predict weak, \textit{conditionalised} presuppositions in certain cases, e.g., (\ref{prov}) is predicted to presuppose that: \textit{if Mary is pregnant, then she has a brother}.

    \ex
    \label{prov}Mary is pregnant and her brother is happy.
    \xe

  \item[Explanatory adequacy] the projection properties of each connective are simply lexicalised as part of their meaning.

   Why don't we see more cross-linguistic variation in projection properties? Whence the left-to-right asymmetry?
\end{description}

\section{Sentences as \textit{updates}}

Recall the Heim/Karttunen generalisation concerning presupposition projection in \textit{conjunctive} sentences.

\ex
\textbf{Conjunction}\\If A\(_π\), and B\(_ρ\), then a sentence of the form \enquote{A and B} presupposes \(π\), and unless A entails \(ρ\), also presupposes \(ρ\)
\xe

\ex~\label{conj1}
Sam and Ka visited Rome and Venice last Summer, and Ka visited Venice again.\\
\textit{presuppositionless}
\xe

\begin{tcolorbox}
  Peter's conjecture
  \tcblower
  An assertion of the first conjunct \textit{Sam and Ka visited Rome and Venice last Summer} should alter the common ground in such a way that an assertion of the second conjunct is felicitous (heavily paraphrased).
\end{tcolorbox}

One way of thinking about this intuition is that we should treat (\ref{conj1}) in a way parallel to the discourse in (\ref{conj2}).

\pex\label{conj2}
\a\label{conj2a}Sam and Ka visited Rome and Venice last Summer.
\a\label{conj2b}Ka visited Venice again.\hfill\textit{presuppositionless}
\xe

The pragmatic felicity condition we introduced for presuppositional sentences last week \textit{does} actually capture this, since at the point that (\ref{conj2b}) is uttered, the presupposition \textit{that Ka has visited Venice before} is entailed by the common ground.

So, we have a story about (\ref{conj2b}), but it's not clear how to extend this to (\ref{conj1}), since \textit{assertion} is a strictly \textit{pragmatic} notion.

\begin{itemize}

  \item In what sense can we say that each conjunct is a distinct assertion?

  \item What is the meaning of \textit{and} such that it can conjoin assertions?

\end{itemize}

In the 80s and early 90s, a family of theories, including Discourse Representation Theory (DRT; \citealt{kamp1981}), File Change Semantics (FCS; \citealt[ch.\,3]{heim1982}), and Dynamic Predicate Logic (DPL; \citealt{groenendijk_dynamic_1991}) attempted to cash out this intuition.

For reasons of time, we'll be focusing on the tradition begun by \citeauthor{heim1982}.

The core idea of FCS is that the denotation of a declarative sentence is not a \textit{proposition} but rather an \textit{instruction} for updating the conversational context -- such instructions are called \textbf{Context Change Potentials (CCPs)}.

\subsection{The Stalnakerian Common Ground (recap)}

Before we say something about what a CCP is, we first need to be precise about the notion of a \textit{conversational context}.

Following Stalnaker, we'll represent the Common Ground in \(c\) as the \textit{Context Set} -- the set of possible worlds compatible with the shared knowledge of the discourse participants (Stalnaker 1973, 1974, 1978, 1998, 2002).

\pex Context Set (def.)
Given conversational participants \(a_{1},\ldots , a_{n}\), the context set \(C\) is the strongest proposition (i.e., the smallest set of possible worlds), such that:
\a Each \(a_{i}\) believes \(C\);
\a Each \(a_{i}\) believes that each \(a_{j}\) believes \(C\);
\a Each \(a_{i}\) believes that each \(a_{j}\) believes that each \(a_{k}\) believes \(C\)\\
\ldots
\xe

Informally: \(C\) is the grand conjunction of all the propositions mutually believed to be true by the discourse participants.

\subsection{The CCPs of declarative sentences}

The context set can be thought of as either a set of worlds, or equivalently as a proposition (type \(\ml{st}\)). The denotation of a declarative sentence is an \textit{instruction} to take the current context set, and sift out all those possible worlds that aren't compatible with the information conveyed by a sentence.

We can model this \enquote{instruction} formally by treating the denotation of a declarative sentence as a \textit{function from context sets to updated context sets} of type \(\ml{⟨st,st⟩\}\). Let's write this type as \(\ml{u}\)

  \ex
  \(\eval{Paul vapes} = λ c . c ∩ \set{w|\ml{p vapes}_{w}}\) \hfill \(∷\ml{u}\)
  \xe

  \begin{tcolorbox}
  Let's stop for a moment to consider this move -- typically, we model the role of \textit{assertions} in discourse as updating the current context set. Here, we're \textit{semanticising} assertion, such that the the \textit{semantic value} of a sentence is its effect on the context set. This is a substantive hypothesis about the semantics-pragmatics interface!
  \end{tcolorbox}

    This meaning for \textit{Paul vapes} captures the dynamic flow of information over the course of a discourse.

    Imagine, we're in a context where we don't know whether or not Paul vapes:

    \[c = \set{w_{1},w_{2},w_{3},w_{4}}\]

    Paul vapes in \(w_{1}\) and \(w_{3}\), but not in \(w_{2}\) or \(w_{4}\)

    When we update a context set with a sentence, we simply \textit{apply} the sentence meaning to the context set:

    \[\eval{Paul vapes} (\set{w_{1},\ldots,w_{4}}) = \set{w_{1},…,w_{4}} ∩ \set{w | \ml{p vapes}_{w}} = \set{w_{1},w_{3}}\]

    The result is an \textit{updated} context set \(c'\) containing \textit{just those worlds} in which Paul vapes.

  \begin{tcolorbox}
    An aside on notation
    \tcblower
    Here, I'm following, e.g., \citet{chierchia_dynamics_1995} in using the lambda notation for CCPs. CCPs are also often written as follows:

    \ex
    \(c + [\text{Paul vapes}] = c ∩ \set{w|\ml{p vapes}_{w}}\)
    \xe

    \ex
    \(c[\text{Paul vapes}] = c ∩ \set{w|\ml{p vapes}_{w}}\)
    \xe
    \linebreak
    These different ways of writing CCPs are equivalent. The lambda notation has the advantage of already being familiar from, e.g., \citet{heimKratzer1998}.
  \end{tcolorbox}

  \subsection{From CCPs to propositions and back again}

  Our classical, static semantics is \textit{subsumed} by this new treatment of sentence meaning, since we can define an operator \((↓)\) to get back from CCPs to propositions.

  To retrieve a proposition from a CCP \(f\), we take the set of worlds \(w\), such that applying \(f\) to \(\set{w}\) returns \(w\).

  \ex
  \(f^{↓} = λ w . f (\set{w}) = \set{w}\)\hfill\((↓) ∷ ⟨\ml{u},\ml{st}⟩\)
  \xe

  \ex~
  \(\eval{Paul vapes}^{↓} \begin{aligned}[t]
    &= λ w . (\set{w} ∩ \set{w|\ml{p vapes}_{w}}) = \set{w}\\
    &= λ w . \set{w|\ml{p vapes}_{w}}
    \end{aligned}\)
  \xe

  \begin{tcolorbox}
    Exercise
    \tcblower
    Define an operator \(𝔸 ∷ ⟨\ml{st},\ml{u}⟩\) which takes a classical proposition and returns the corresponding CCP.
  \end{tcolorbox}

  \subsection{Modelling presuppositions}

  \subsubsection{Presuppositions as preconditions on updates}

  Heim's intuition is that presuppositions impose \textit{preconditions} for CCPs to update (i.e., apply to) the current context set.

  If these preconditions are met, we say that the presuppositions of a given CCP are \textbf{satisfied} relative to a context set \(c\) -- sometimes the dynamic theory of presupposition projection is called the \textbf{satisfaction theory}.

  We can cash out this intuition formally by treating CCPs as \textit{partial} functions from context sets -- an utterance is infelicitous if the associated CCP is undefined when applied to the current context set.

  \ex
  \(\eval{Paul quit vaping} = λ c . \begin{cases}
    c ∩ \set{w| ¬ \ml{p vapes now}_{w} } &\set{w | ¬ \ml{p did vape}_{w}} ⊆ c\\
    ♯ & \text{else}
    \end{cases}\)
  \xe

  The CCP associated with \textit{Paul quit vaping} imposes as a precondition, that the current context \(c\) entails \textit{that Paul used to smoke}.

  If this precondition is satisfied, it updates \(c\) with the information that \textit{Paul doesn't smoke now}, otherwise the result is undefined (and therefore: infelicitous).

  \subsubsection{The bridge principle}

  In this theory, the felicity principle bridging between the semantic value of a sentence and its pragmatic contribution is extremely straightforward:

  \ex
  An utterance of sentence S by agents \(a_{1}…a_{n}\) in a context set \(C\) is only felicitous if \(\eval{S} C\) is defined.
  \xe

  Arguably, this falls out from the following more general principle:

  \ex
  An utterance of sentence S by agents \(a_{1} \ldots a_{n}\) in a context \(C\) results in an updated context \(\eval{S} C\).
  \xe


  \begin{tcolorbox}
    Writing partial functions
    \tcblower
    The following is to be read as: that function from \(x\) to \textit{output}, which is defined iff \textit{condition} holds.

    \[λ x . \begin{cases}
        \text{\it output} &\text{\it condition}\\
        ♯&\text{else}
      \end{cases}\]

    You can also use the colon notation introduced in \citet{heimKratzer1998}:

    \[λ x:\text{\it condition} . \text{\it output}\]
  \end{tcolorbox}

  \section{Presupposition projection}

  \subsection{From discourse sequencing to dynamic conjunction}

  One...dare I say...\textit{beautiful} result of dynamic semantics is that \textit{discourse sequencing}, which we'll write as \((;)\), is just \textit{function composition}.

  \begin{tcolorbox}
    Function composition
    \tcblower
    The composition of \(f\) of type \(⟨ρ,τ⟩\) with \(g\) of type \(⟨σ,ρ⟩\), written \(f ∘ g\) is defined as follows:

    \ex
    \(f∘g ≔ λ x . f (g x)\)
    \xe
  \end{tcolorbox}

  The intuition here: the current conversational context \(c\) is first updated by \(p\), the first sentence, and then the updated context \(c'\) is updated by the second sentence \(q\).

  In the literature, people often use the following language: the \enquote{local context} of \(q\) is \(p c\) (i.e., the context updated with \(p\)).

  \ex Discourse sequencing \((;)\) (def.)\\
  \(p;q ≔q∘p \)\hfill\(∷⟨\ml{u},⟨\ml{u,u}⟩⟩\)
  \xe

  Using \((;)\) we can assign a \textit{meaning}, compositionally, to a discourse.

  \ex
  \(\eval{\begin{aligned}[c]
      &\text{Hubert smokes;}\\
      &\text{Paul vapes}
    \end{aligned}} \begin{aligned}[t]
    &=(λ c ∙ c ∩ \set{w|\ml{p vapes}_{w}}) ∘ (λ c ∙ c ∩ \set{w|\ml{h smokes}_{w}})\\
    &= λ c . (c ∩ \set{w|\ml{h smokes}_{w}}) ∩ \set{w|\ml{p vapes}_{w}}\\
    \end{aligned}\)
  \xe

  Now, let's take a scenario where the first sentence in a discourse \textit{entails} the presupposition of the second. If we treat discourse sequencing is function composition, it falls out automatically that the presupposition is \textit{satisfied} within the confines of the sentence, and therefore fails to project.

  Let's take the discourse \textit{Paul vaped last year; Paul quit vaping}. Recall, we're treating \textit{Paul quit vaping} as a partial function from contexts to contexts.

  \ex
  \(\eval{Paul vaped last year; Paul quit vaping}\)
  \xe

  How do we compute the meaning? First take the meaning of \textit{Paul vaped last year}, and update \(c\) with it:

  \ex
  \(\textcolor{gray}{λ c . }c ∩ \set{w|\ml{p vaped}_{w}}\)
  \xe

  Now, update the result with \textit{Paul quit vaping}:

  \ex
  \(\begin{aligned}[t]
    &\textcolor{gray}{λ c . }\begin{cases}
      \begin{aligned}[t]
        &(c ∩ \set{w|\ml{p vaped}_{w}})\\
        &∩ \set{w|¬ \ml{p vapes-now}_{w}}\end{aligned} & \overbrace{\set{w|\ml{p vaped}_{w}} ⊆ (c ∩ \set{w|\ml{p vaped}_{w}})}^{\text{this is a tautology!}}\\
    ♯&\text{else}
  \end{cases}\\
  &= \textcolor{gray}{λ c . }\begin{cases}
    \begin{aligned}[t]
      &(c ∩ \set{w|\ml{p vaped}_{w}})\\
      &∩ \set{w|¬ \ml{p vapes-now}_{w}}
      \end{aligned}& ⊤\\
    ♯&\text{else}
  \end{cases}\\
  &= \textcolor{gray}{λ c . }(c ∩ \set{w|\ml{p vaped}_{w}}) ∩ \set{w|¬ \ml{p vapes-now}_{w}}
  \end{aligned}\)
  \xe

  Now, in order to capture the projection properties of conjunctive sentence, we make the following claim!

  \ex
  \(\ml{and}_{d} ≔ (;)\)\hfill\(∷⟨\ml{u},⟨\ml{u,u}⟩⟩\)
  \xe

  In other words, \textbf{\textit{and} sequences CCPs}.

  As an automatic consequence, we derive the following facts:

  \ex
  Paul vaped last year and Paul quit vaping.\hfill\textit{presuppositionless}
  \xe

  \ex~
  Paul quit vaping and Paul vaped last year.\hfill\textit{presupposes that Paul vaped}
  \xe

  \begin{tcolorbox}
    Exercise
    \tcblower
    Demonstrate the following equivalence:

    \[(p;q)^{↓} ≡ p^{↓} ∩ q^{↓}\]

    A successful demonstration shows us that \((;)\) subsumes the static contribution of \textit{and}.
  \end{tcolorbox}

  We've now cashed out Peter's intuition -- the meaning of a discourse is built up compositionally; \textit{and} operates on CCPs, not propositions.

  Ultimately, our new entry for \textit{and} is able to get the linear asymmetry in projection because of the following fact:


  \ex Fact: function composition is \textit{asymmetric}:\\
  \(f∘g ≠ g∘f\)
  \xe

  \ex~ Fact: conjunction/intersection is \textit{symmetric}:\\
  \(p ∧ q = q ∧ p\)
  \xe

  \begin{tcolorbox}
    On the associativity of composition
    \tcblower
    Another fact about function composition is that -- like conjunction/intersection -- it's \textit{associative}.

    \[(f∘g)∘h = f∘(g∘h)\]

    As a consequence, discourse sequencing/dynamic conjunction is associative too! Rather than writing, (\ref{bad}) we can just write (\ref{good}):

    \pex
    \a\label{bad}\((\eval{Paul vapes};\eval{Hubert smokes});\eval{Uli is straight-edge}\)
    \a\label{good}\(\eval{Paul vapes};\eval{Hubert smokes};\eval{Uli is straight-edge}\)
    \xe

  \end{tcolorbox}

  \subsection{An early worry}

  What do we predict the following sentence to presuppose?

  \ex
  It's raining, and Paul quit vaping.
  \xe

  We predict \textit{exactly the same presupposition as before}. Namely, that a context \(c\) updated with the proposition \textit{it's raining}, should entail that Paul used to vape. In other words, we the global presupposition is predicted to be: \textit{if it's raining then Paul used to vape}.

  This is an early illustration that, much like the multidimensional theory, FCS is subject to the \textit{proviso problem}. In certain cases, we generate weak, conditionalised presuppositions, rather than the stronger ones that are attested.

  Notice that this problem only arises with \textit{accommodation}. If we are in a context where it is known that \textit{Paul used to vape}, then it is also known that \text{if it's raining then Paul used to vaped}

  The standard solution to the proviso problem in FCS is to say that, indeed, what gets accommodated is the weaker, conditional presupposition \textit{if it's raining than Paul used to vape}, but this is pragmatically an odd thing to accommodate, so it gets strengthened to \textit{Paul used to vape} (see, e.g., \citealt{beaver_presupposition_2001,kadmon2001}).

  We'll see reasons to disbelieve this story in next week's class.

  \subsection{Negation}

  \citet{heim1983} defines the CCP of a negative sentence as follows:

  \ex
  \(\ml{neg}_{d} p = λ c . c - (p c)\)\hfill\(\ml{neg}_{d} ∷ ⟨\ml{u,u}⟩\)
  \xe

  Informally, in order to update \(c\) with \textit{not \(p\)}:

  \begin{itemize}

    \item First, update \(c\) with p, returning an updated context \(c'\).

    \item subtract the updated context \(c'\) from \(c\).

  \end{itemize}

  Since \(not p\) first triggers an update of \(c\) by \(p\), the negative update is only defined in \(c ∈ \ml{dom} p\). Therefore the negative sentence inherits the presuppositions of its positive counterpart. As we saw last week, this seems to be a correct prediction.

  \pex
  \a Paul quit vaping.\hfill\textit{presupposes that Paul used to vape}
  \a It's false that Paul quit vaping.\hfill\textit{presupposes that Paul used to vape}
  \xe

  \begin{tcolorbox}
    Exercise
    \tcblower
    Does the following equivalence hold?:

    \[(\ml{neg}_{d} p)^{↓} = \ml{not} (p^{↓})\]

    In a multi-dimensional setting, despite worries concerning \textit{explanatory adequacy} we had a way of systematically lifting negation into a multi-dimensional setting via \(π\). Can we define such an operator to lift propositional negation into the dynamic setting?
  \end{tcolorbox}

  \subsection{Conditionals}

  If you did the problem set, you hopefully noticed that the projection properties of \textit{if...then...} conditions are the same as for conjunctive sentences.

\ex
\textbf{Conditionals}\\If A\(_π\), and B\(_ρ\), then a sentence of the form \enquote{if A then B} presupposes \(π\), and unless A entails \(ρ\), also presupposes \(ρ\)
\xe

\ex~
If Paul has a good theory, he'll tell you about his theory.\hfill \textit{presuppositionless}
\xe

\ex~
If Paul tells you about his theory, he'll be happy.\hfill\textit{presupposes Paul has a theory}
\xe

Heim's CCP for conditionals:

\ex
\(p \ml{if...then...}_{d} q ≔ λ c . c - (p c - q (p c))\)
\xe

The local context for the second disjunct \(q\) is \(c\) updated with \(p\).

This doesn't seem very realistic as a meaning for natural language conditionals though -- truth-conditionally, this leads to the expectation that they should always express material implication (a worry pointed out by Heim).

  \subsection{Disjunction}

  As we observed last week, it seems that if the \textit{negation} of the first disjunct entails the presupposition of the second, the presupposition fails to project:

  \ex
  Either John has no children, or his children do not live with him.
  \xe

  \ex~
  \(p \ml{or}_{d} q ≔ λ c . (p c) ∪ q (c - (p c))\)
  \xe

  \begin{tcolorbox}
    Exercise
    \tcblower
    Consider the following equivalence:

    \[p → q ≣ ¬ (p ∧ ¬ q)\]

    What happens if we try to derive the dynamic entry for the conditional from the above equivalence (taking the corresponding dynamic entries for the logical operators).

    \[p \ml{if}_{d} q =?= \ml{not}_{d} (p;(\ml{not}_{d} q))\]

    What about disjunction? Start from the following equivalence -- does it derive the dynamic entry for disjunction?

    \[p ∨ q ≡ ¬ (¬ p ∧ ¬ q)\]

  \end{tcolorbox}

  \subsection{Explanatory adequacy}

  \begin{tcolorbox}
    Possible squib topic alert! -- beyond the canonical logical operators
    \tcblower
    Consider the expression \textit{unless}. Naïvely, it looks like it means something similar to (strengthened) NL disjunction:

    \ex
    {}[Paul vapes]\(_{p}\) unless [Sophie is around]\(_{q}\).\\
    \textit{False if (a) \(p\) is false and \(q\) is false,\\
      (b) \(p\) is true and \(q\) is true.}
    \xe

    What are it's projection properties? They look like the exact reverse of disjunction (thanks Roger for discussing this with me!).

    \ex
    Paul will tell you about his theory, unless he has no theory.
    \xe

    Does this raise issues for the idea that the projection properties of a connective should be predictable from its logical properties?

    What about other connectives which carry discourse-related inferences?
  \end{tcolorbox}

  \section{The dynamics of anaphora}

  The way we've presented dynamic semantics doesn't quite mirror its historical trajectory. DySem was originally developed to account for \textit{anaphora}, and specifically its ability to span across domains which are ordinarily boundaries for syntactic/semantic relations.

  \ex
  A philosophy student\(^{x}\) walked in. They\(_{x}\) sat down.
  \xe

  So-called \textit{donkey sentences} such as (\ref{donk}) pose an especially acute problem, since we can't make recourse to exceptional scope of \textit{a donkey}.

  \ex\label{donk}
  Every farmer who owns a donkey\(^{x}\) beats it\(_{x}\)
  \xe

  In order to account for the dynamics of anaphora, we'll first need to extend our fragment to quantificational sentences.

  In doing so, we'll also attempt to solve the \textit{binding problem} -- one of our primary motivations for moving away from a multi-dimensional system.

  \subsection{Assignments}

  Assignment functions assign referents to identifiers, often modelled as \textit{indices}. We'll assume that assignments are \textit{partial}.

  We can model contextual knowledge about which identifier is mapped to which referent as a \textit{set of assignments}.

  let's assume that our domain is \(\ml{andy, dani, yasu}\), and we have three indices \(1,2,3\). The following represents a context where we are certain who to map identifiers \(1,2\) to, but ignorant about who \(3\) gets mapped to.

  \[\Set{\left[\begin{aligned}[c]
          &1 ↦\ml{andy}\\
          &2 ↦\ml{dani}\\
          &3 ↦\ml{andy}
        \end{aligned}\right],\left[\begin{aligned}[c]
          &1 ↦\ml{andy}\\
          &2 ↦\ml{dani}\\
          &3 ↦\ml{dani}
        \end{aligned}\right],\left[\begin{aligned}[c]
          &1 ↦\ml{andy}\\
          &2 ↦\ml{dani}\\
          &3 ↦\ml{yasu}
        \end{aligned}\right]}\]

  \subsection{Extending contexts}

  Rather than treating contexts as sets of worlds (i.e., Stalnakerian context sets), we're going to extend this notion and treat contexts as \textit{sets of world-assignment pairs}.

  \citeauthor{heim1982} calls such objects \textbf{files} (hence, file change semantics).

  \begin{tcolorbox}
    A note on partial assignments
    \tcblower
    Since we're treating assignments as \textit{partial} functions from indices to referents our refined notion of \enquote{file} will need a small caveat. Concretely, a context needs to be defined relative to a \textit{domain} of indices \(N\).

    \pex
    File (def.)\\
    A file \(c\) with a domain \(N\) is a set of assignment-world pairs, s.t.
    \a \(\set{w|∃ g[⟨g,w⟩ ∈ c]}\) is the Stalnakerian context set (or, the \textit{worldly content} of the file)
    \a For any \(⟨g,w⟩, ⟨g',w'⟩ ∈ c\), \(\ml{dom} g = \ml{dom} g' = N\)
    \xe

  \end{tcolorbox}

  This extension has no effect on sentences without pronouns or quantifiers:

  \ex
  \(\eval{Paul vapes} = λ c . \set{⟨g,w⟩|\ml{p vapes}_{w}}\)
  \xe

  Pronouns denote variables, and impose a \textit{familiarity condition}, i.e., the induce a presupposition that the index they carry is in the domain of the current conversational context.

  \ex
  \(\eval{he\(_{7}\) vapes} = λ c:7 ∈ \ml{dom} c . \set{⟨g,w⟩|g_7 \ml{vapes}_w}\)
  \xe

  % In order to account for the binding problem, let's treat indefinites as a function from a dynamic predicate to a CCP.

  % \ex
  % \(\eval{quit vaping} = λ x . λ c : \set{<g,w>|x \ml{vaped}_{w}} ⊆ c . c ∩ \set{⟨g,w⟩|¬ x \ml{vapes}_lj {w}}\)
  % \xe

  % \ex~
  % \(\eval{someone\(^{n}\)} = λ f . λ c : n ∉ \ml{dom} c . c ∩ \set{⟨g,w⟩|∃x,g'[⟨g',w⟩ ∈ f x ∧ g[n ↦ x]g']}\)
  % \xe

  % Now, finally!

  % \ex
  % \(\eval{someone\(^{n}\) quit vaping}= λ c:n ∉ \ml{dom} c ∧ \set{}\)
  % \xe

  % \todo[inline]{Is the following correct??? We could also just give a CCP for a sentence with an indefinite}
  % \ex
  % \(\eval{someone\(^{7}\)} = λ f . λ c . \set{⟨g,w⟩|∃g',x[⟨g',w⟩ ∈ (f x) ∧ g[7 ↦ x]g']}\)
  % \xe




  % \[\Set{\left[\begin{aligned}[c]
  %         1 ↦ \ml{andy}\\
  %         2 ↦\ml{yasu}\\
  %         3 ↦ \ml{andy}
  %       \end{aligned}\right],\Set{\left[\begin{aligned}[c]
  %         1 ↦\ml{andy}\\
  %         2 ↦\ml{yasu}\\
  %         3 ↦\ml{dani}
  %       \end{aligned}\right],\Set{\left[\begin{aligned}[c]
  %         1 ↦\ml{andy}\\
  %         2 ↦\ml{yasu}\\
  %         3 ↦\ml{yasu}
  %       \end{aligned}\right]}\]



% \subsection{Notational conventions}

% \begin{itemize}

%     \item \(λxy . …\) is to be read as \(λ x . λ y . …\)

%     \todo[inline]{Be precise about other things here}

% \end{itemize}

% \section{Dynamic Semantics}

% \subsection{Context Change Potentials}

% A Context Change Potential (CCP) is a (possibly partial) function over contexts. For example, the CCP associated with the utterance \enquote{Paul vapes} is:

% \ex
% \label{ccp1}\(λ c . λ w . c w ∧ \ml{p vapes}_{w}\)
% \xe

% Assuming that sentences denote \textit{propositions}, how do we get from an ordinary (static) sentential meaning, to a CCP? Let's define an operator \textsc{Assert} (\(𝔸\)) to accomplish this for us:

% \ex \textsc{Assert} (first attempt)\\
% \(𝔸 p ≔ λ c . λ w . c w ∧ p w\)
% \xe

% \ex~
% \(𝔸 (\eval{Paul vapes}) = (\ref{ccp1})\)
% \xe

% Suppose that the context set contains both worlds in which Paul vapes, and worlds in which he doesn't, representing a state of ignorance wrt whether or not Paul vapes, i.e.:

% \ex
% \(c_{1} = λ w . \ml{p vapes}_{w} ∨ ¬ \ml{p vapes}_{w}\)
% \xe

% We update the context set \(c_{1}\) with the CCP associated with \enquote{Paul vapes} by \textit{applying} the CCP to the context set. The result is an \textit{updated} context set which only includes worlds in which paul vapes.

% \ex
% \([λ c . λ w . c w ∧ \ml{p vapes}_{w}] c_{1} = λ w . \ml{p vapes}_{w}\)
% \xe

% Note that we can always get back from a CCP to a proposition. Let's define an operator \(↓\) to do this for us:

% \todo[inline]{Check that this works out}

% \ex
% \(d^{↓} ≔ λ w' . (d (λ w . w = w')) = (λ w . w = w')\)
% \xe

% Before moving on to the analysis of presupposition in FCS, let's consider how we might analyse conjunction/discourse sequencing within this framework.

% Consider the discourse in (\ref{discourse1}). Suppose that at \(c_{1}\), the participants are ignorant about both whether Paul vapes and whether Sophie smokes. At \(c_{2}\) the participants know that Paul vapes, but are ignorant about whether Sophie smokes. At \(c_{3}\), the participants know both that Paul vapes and that Sophie smokes.

% \ex\label{discourse1}
% \(c_{1}\) Paul vapes \(c_{2}\); Sophie smokes \(c_{3}\).
% \xe

% We can give a \textit{semantics} of the discourse sequencing operator \((;)\) which captures the intuition that a sequential utterances successively update the context set. To sequence CCPs, we simply \textit{compose} them.

% \ex
% \((;) ≔ λ q . λ p . q ∘ p\)
% \xe

% \ex~
% \begin{forest}
%   [{\(λ c . [𝔸 \eval{Sophie smokes}] ([𝔸 \eval{Paul vapes}] c)\)}
%     [{...} [{Paul vapes},roof]]
%     [{...}
%       [{\(;\)}]
%       [{...} [{Sophie smokes},roof]]
%     ]
%   ]
% \end{forest}
% \xe

% As it stands, the CCPs we've dealt with so far have been \textit{total} functions over contexts. Furthermore, CCPs are defined in terms of conjunction, and since conjunction is associative:

% \ex
% Paul vapes; Sophie smokes = Sophie smokes; Paul vapes.
% \xe

% In fact, if we lower the result of sequencing the two sentences, we just get...classical conjunction. So what have we achieved exactly?

% \todo[inline]{Get them to demonstrate this in the exercises.}

% \subsection{Presupposition satisfaction}

% \citeauthor{heim1983}'s (\citeyear{heim1983}) innovation was to treat presuppositions as, essentially, \textit{definedness conditions on CCPs}. We say that, if a context set \(c\) entails the presupposition of a sentence S, \(c\) \textbf{satisfies} the presupposition S.\footnote{You can also write definedness conditions on functions using \citeauthor{heimKratzer1998}'s colon notation, although it will quickly get quite unwieldy. For example:

%   \ex
%   \(λ c . c ⊆ \set{w|\ml{p used-to-vape}_w} . λ w . c w ∧ ¬ \ml{p vapes}_{w}\)
%   \xe
% }

% We can now shift gears and treat the CCPs associated with sentences as (potentiall!) partial functions from contexts to contexts. The CCP associated with the presuppositional sentence \enquote{Paul quit vaping} is given below:

% \ex
% \(λ c . \begin{cases}
%   λ w . c w ∧ ¬ \ml{p vapes}_{w} & c ⊆ \set{w | \ml{p used-to-vape}_{w}}\\
%   ♯&\ml{else}
%   \end{cases}\)
% \xe

% Now that CCPs can be \textit{partial}, treating discourse sequencing as function composition has an interesting consequence.

% \subsection{Quantification}

% \subsubsection{Assignment functions and variables}

% In order to deal with anaphora, \citeauthor{heim1983}'s dynamic semantics deals with an enriched notion of context set -- namely, a \textit{file}. A file is a set of \textit{world-assignment function pairs} (or equivalently, a relation between worlds and assignments).

% The file associated with the information that a person at identifier \(7\) vapes is:

% \ex
% \(\set{⟨g,w⟩ : g_{7} \ml{vapes}_{w}}\)
% \xe

% \ex~ Definition of a \textit{file}:\\
% \(c\) is a file iff there is a subset \(ℕ'\) of \(ℕ\), and \(c\) is a set of assignment-world pairs, \(⟨g,w⟩\), where \(g: ℕ' ↦ D_{\ml{e}}\). We refer to \(ℕ'\) as the \textit{domain} of the file (i.e., \(\ml{dom} c\))
% \xe

% We'll use variables named \(c,c',c''…\) to range over \textit{files}. \(λ c . …\) can be read as an abbreviation of \(λ c :\ml{file} c . …\)

% \(g[n]g'\) mean that \(g\) and \(g'\) differ only at \(n\), and that \(g_{n}\) is defined (\(g'_{n}\) may be undefined).

% We can now give a CCP for a sentence with an indefinite:

% \ex
% Someone\(^{7}\) arrived = \(λ c . \set{⟨g,w⟩|∃⟨g',w⟩ ∈ c[g[7]g' ∧ g_{7} \ml{arrived}_{w}]}\)
% \xe


\printbibliography

\begin{appendix}

  \section{Last week's problem set}

\subsection{From the law on non-contradiction to the law of the excluded middle}

\ex\label{nc}The law of non-contradicition\\
\(¬ (p ∧ ¬ p)\)
\xe

For (\ref{nc}) to be true, the following must be false:

\ex
\(p ∧ ¬ p\)
\xe

What does it take for a conjunction to be false? Either the first conjunct is false, or the second conjunct is false. In other words, the following must hold:

\ex \(¬ p ∨ ¬ ¬ p\)
\xe

This is of course equivalent to...the law of the excluded middle:

\ex
\(¬ p ∨ p\)
\xe

\begin{tcolorbox}
  Discussion
  \tcblower
Now, even the ancient Greeks knew that natural language doesn't obey the law of the excluded middle...nevertheless, we want to maintain the law of non-contradiction as a basic axiom...we can conclude that the inferential system underlying natural language semantics must be non-classical in nature, since these two things are equivalent in a classical setting.
\end{tcolorbox}

\subsection{Scoping in presuppositions}

\ex
\(\displaystyle\frac{p}{x} \bind m ≔ \frac{p ∧ ℙ (m x)}{𝔸 (m x)}\)
\xe

\ex~
\begin{forest}
  [{\(\displaystyle\frac{g_{1} \ml{ids fem} ∧ g_{1} \ml{did smoke}}{¬ g_{1} \ml{smokes now}}\)}
  [{\(λ m . \displaystyle\frac{g_{1} \ml{ids fem} ∧ ℙ (m g_{1})}{𝔸 (m g_{1})}\)}
    [{\(\bind\)}]
    [{\(\displaystyle\frac{g_{1} \ml{ids fem}}{g_{1}}\)\\she\(_{1}\)}]
  ]
    [{\(λ x . \displaystyle\frac{x \ml{did smoke}}{¬ x \ml{smokes now}}\)} [{quit smoking},roof]]
  ]
\end{forest}
\xe

\begin{tcolorbox}
  Discussion
  \tcblower
  In order to get a presuppositional individual to compose with a presuppositional predicate, we need to define a new operator \((\bind)\) -- note that applying \((\bind)\) to the pronoun returns something of type \(\left\langle\left\langle\ml{e},\displaystyle\frac{\ml{st}}{\ml{st}}\right\rangle,\displaystyle\frac{\ml{st}}{\ml{st}}\right\rangle\). This operator is a way of shifting a presuppositional individual into a presuppositional scope-taker.
\end{tcolorbox}

\subsection{More on the binding problem}

We suggested a way of solving the binding problem by having the at-issue meaning entail the presupposition. What we predict for \textit{someone quit smoking} is:

\ex
\(\displaystyle\frac{∃ x[x \ml{did smoke}]}{∃y[¬ \ml{smoke now} ∧ y \ml{did smoke}]}\)
\xe

\begin{tcolorbox}
  Discussion
  \tcblower
  This only solves the binding problem in a veridical context. In a non-veridical context, the predicted at-issue meaning is too weak.
\end{tcolorbox}

\end{appendix}

\end{document}

%%% Local Variables:
%%% TeX-engine: xetex
%%% TeX-master: t
%%% End:
