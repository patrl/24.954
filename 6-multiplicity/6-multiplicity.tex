\documentclass[landscape,twocolumn,cronos,paper=letter]{ling-handout}

\usepackage[margin=1in]{geometry}
\usepackage{fontawesome,calc,enumitem,parskip,tcolorbox}

\usepackage[normalem]{ulem}

\lingset{
  belowexskip=0pt,
  aboveglftskip=0pt,
  belowglpreambleskip=0pt,
  belowpreambleskip=0pt,
  interpartskip=0pt,
  extraglskip=0pt,
  Everyex={\parskip=0pt},
}


\usepackage{float,soul}

\addbibresource[location=remote]{/home/patrl/repos/bibliography/elliott_mybib.bib}

% \renewcommand*{\sectionformat}{}
% \renewcommand*{\subsectionformat}{}
% \renewcommand*{\subsubsectionformat}{}

\title{Multiplicity}
\subtitle{Plurality at the semantics-pragmatics interface}

\date{November 8, 2019}

\author{Patrick~D. Elliott
  \and
Roger Schwarzchild}

\begin{document}

\maketitle

\section{The weak theory of plurality}

\begin{itemize}

  \item What does plural marking on nouns contribute to meaning? Naively:
    \enquote{[Plurality] just means \enquote{more than one}} (\citealt[ix]{lasersohn_plurality_1995})

  \item We'll refer to this view as the \textbf{strong theory of plurality}.

  \item The strong theory still has many
    champions; see e.g., \citet{harbour_paucity_2014}.\footnote{\citeauthor{harbour_paucity_2014}'s work is a
    particularly interesting example (see, e.g.,
    \citealt{harbour_paucity_2014}). In order to account for generalisations
    concerning possible number systems cross-linguistically, it is crucial for
    Harbour that the plural is semantically strong.}

  \item As has been noticed by numerous researchers
    (\citealt{sauerlandAnderssenYatsushiro2005} cite Hoeksema 1983, van Eijck
    1983, Schwarzchild 1996, and Beck \& Sauerland 2000) this seems to get the facts
    right in simple cases, but faces problems in negative contexts.

    \ex
    Some chairs are available.\\
    \(⇝ \ml{there exist chairs } X\ml{ s.t. }X\ml{ are available and }♯X ≥ 2\)
    \xe

    \pex\label{st}No chairs are available.
    \a\label{st1}\(⇝ \ml{there are no chairs }X\ml{ s.t. }X\ml{ are available}\)
    \a \(\not\rightsquigarrow \ml{there are no chairs }X\ml{ s.t. }X\ml{ are available and }♯X≥2\)
    \xe

    \item Taken at face value, the strong theory predicts that if
    \textit{exactly one chair is available}, then (\ref{st}) is true.

    \item The weak theory takes as its starting point sentences such as
    (\ref{st}).

    \item In order to account for the truth-conditions of (\ref{st}), it must be
    the case that plurality doesn't necessarily exclude the atoms.

    \item This leaves positive sentences as the residue. By far and away the
    most well-represented approach in the literature is to derive the
    \textit{more than one} interpretation somehow via competition with the singular.

\end{itemize}

\subsection{Implementing the weak theory}

\begin{itemize}

    \item Perhaps the most influential implementation of the weak theory is
    \cite{sauerland2003}.

  \item \citeauthor{sauerland2003} assumes that number features are of type
    \(\type{e → e}\).

    \item The feature \textsc{plur} is semantically vacuous (i.e., the identity
    function), whereas the feature \textsc{sing} is a \textit{partial} identity
    function, that is defined iff its input has a cardinality of \textit{one}.

    \ex
    \(\eval{\textsc{plur}} ≔ \ml{id}\)\hfill\(\type{e → e}\)
    \xe

    \ex~
    \(\eval{\textsc{sing}} ≔ λ x:♯x = 1 . x\)\hfill\(\type{e → e}\)
    \xe

    \item This gets the facts right for \textit{no chairs are available}.

    \item But hang on a sec, doesn't that mean we lose an account of why, from
    an utterance of the following, we can infer that \textit{more than one
    chair} is available?

    \ex
    (Some) chairs are still available.
    \xe

    \item Apparently, according to the weak theory, this should just mean that
    \textit{one or more} chairs are available.

    \item Even worse, we predict the following to be
    fine.

    \ex\ljudge{*}John are here.
    \xe

    \item The plural morphology on the
    predicate is licensed by the plural $ϕP$.

    \item Since \texsc{plur} is semantically vacuous -- i.e., an
    \textit{identity function} -- compositionally nothing goes wrong.

    \ex
    \begin{forest}
      [{$\ml{isHere j}$}
      [{$ϕP$}
        [{$ϕ$\\\textsc{plur}}]
        [{John}]
      ]
        [{$λx . \ml{isHere} x$} [{are here},roof]]
      ]
    \end{forest}
    \xe

    \end{itemize}

    \subsection{\textit{Maximize Presupposition} to the rescue}

    \begin{itemize}

        \item Over the last couple of weeks, we've discussed mechanisms for
        strengthening the \textit{at-issue} part of a sentence's meaning,
        through the lens of the exhaustivity operator $\ml{exh}$.

        \item In the basic case, the logic was as follows: for an expression
        $\alpha$, the strengthened meaning (the \enquote{implicature}) can be derived by negating each
        alternative $\psi$ to $\alpha$, s.t., $\psi$ is $\alpha$-\textit{excludable}.

        \item We defined our notion of \textit{excludable} relative to logical
        relations between the at-issue meaning components of $\alpha$ and
        $\psi$.

        \item We began by considering a version of the exhaustivity operator
        according to which the \textit{logically non-weaker} alternatives are
        excludable; in order to derive free choice inferences, we shifted to a
        more refined version of excludability -- \textit{innocent
        excludability}.

        \item We handily abstracted away from any cases involving sentences with
        \textit{presuppositions} (the other half of this course).

        \item Relevant to this point, there are a number of expressions which seem to have equivalent
        at-issue meanings, but which differ in the logical strength of their
        presuppositional component:

        \begin{table}[h!]
        \begin{center}
\begin{tabular}{lll}
 weaker item & stronger item  & differential presupposition  \\
  \hline
 a & the  & uniqueness  \\
  all/every & both  & domain contains exactly two elements\\
  believe & know & complement is true
\end{tabular}
\end{center}
\end{table}

        \item For such pairs, two kinds of interpretive effects have been observed.

        \begin{itemize}

            \item The use of a presuppositionally weaker form in a context that
            does not satisfy the presupposition of its stronger complement tends
            to implicate that this presupposition is false. In the literature,
            these are commonly referred to as
            \textit{anti-presuppositions}.\footnote{A term coined by Kai von
            Fintel; introduced in \cite{percus2006antipresuppositions}.}

        \end{itemize}

        \item The following examples are taken from \citet[p.,14]{marty2017}:

        \item $\langle \text{a}, \text{the}\rangle$

        \item Anti-presupposition for (\ref{ap1}): $¬ [\ml{J has exactly one brother}]$

        \pex
        John lives with a brother of his.
        \a\label{ap1}Presupposition: $∅$
        \a Assertion: $\ml{J lives with a brother of his}$
        \xe

        \pex~
        John lives with his brother.
        \a Presupposition: $\ml{John has exactly one brother}$
        \a Assertion: $\ml{J lives with a brother of his}$
        \xe

        \item $\langle\text{all},\text{both}\rangle$

        \item Anti presupposition for (\ref{ap2}): $¬ [\ml{J has exactly two children}]$

        \pex
        John brought all his children.
        \a\label{ap2}Presupposition: $∅$
        \a Assertion: John brought all his children.
        \xe

        \pex~
        John brought both his children.
        \a Presupposition: $\ml{John has exactly two children.}$
        \a Assertion: $\ml{John brought all of his children}$
        \xe

        \item $\langle\text{believe},\text{know}\rangle$

        \item Anti-presupposition for (\ref{ap3}): $¬ [\ml{spkr is 6ft tall}]$

        \pex
        John believes that I am six feet tall.
        \a\label{ap3}Presupposition: $∅$
        \a Assertion: $\ml{J believes that spkr is 6ft tall}$.
        \xe

        \pex~
        John knows that I am six feet tall.
        \a Presupposition: $\ml{spkr is 6ft tall}$
        \a Assertion: $\ml{J believes that spkr is 6ft tall}$
        \xe

        \item Note that these effects don't go through when it is clearly not
        the case that the speaker is opinionated:

        \pex
        \a I have no idea whether John has only one or two brothers; however, I
        know for sure that he lives with a brother of his.
        \a I don't know how many children John has, but he will bring them all.
        \a I don't know how tall I am. John believes I am 6 feet tall.
        \xe


        \item Second, use of the presuppositionally weaker item is perceived as
        \enquote{odd} when the presupposition of the stronger alternative is
        entailed by the common ground (\citealt{heim1991}).

        \pex
        $C ⊆ \ml{there is exactly one sun}$\hfill $\langle \text{a}, \text{the}\rangle$
        \a\ljudge{\#}A sun is shining.
        \a The sun is shining.
        \xe

        \pex~
        $C ⊆ \ml{John has exactly two parents}$\hfill$\langle\text{all},\text{both}\rangle$
        \a\ljudge{\#}John talked to all his parents.
        \a John talked to both his parents.
        \xe

        \pex
        $C ⊆ \ml{2+2 equals 4, and John just proved it}$\hfill$\langle\text{believe},\text{know}\rangle$
        \a\ljudge{\#}John believes that 2+2 equals 4.
        \a John knows that 2+2 equals 4.
        \xe

        \item Intuitively, this oddness obtains because the speaker does not
        take for granted the truth of the presuppositions of their (b)
        alternatives. \citet{marty2017} refers to these phenomena as
        \textit{anti-presuppositional effects}.

        \item In order to capture these effects, \citet{heim1991} proposed an
        independent pragmatic principle, which has come to be known as
        \textit{Maximize Presupposition!} (after \citealt{sauerland_2008}).

        \begin{tcolorbox}
          \textsc{Maximize Presupposition} (MP)
          \tcblower
          If the following is true for any $\psi \in \ml{alt} ϕ$, then the
          sentence $ϕ$ cannot be felicitously uttered in context $C$:
          \begin{itemize}
          \item $ψ$'s presupposition \textit{asymmetrically entails} $ϕ$'s
          presupposition, and
          \item $ϕ$ and $ψ$ are contextually equivalent.
              \end{itemize}
        \end{tcolorbox}

        \item Informally, if there is an alternative $\psi$ to a sentence $\phi$
        that makes the same assertive contribution but has a stronger
        presupposition, then $\psi$ should be preferred.

        \item If $\psi$ is not used, then MP triggers the inference that $\psi$
        \textit{could not have been used}.

      \item An example of MP-style reasoning:

        \begin{itemize}

            \item Let $ϕ$ and $\phi_{\pi}$ be two sentences with the same
            assertive content, which minimally differ in that $\phi_{\pi}$
            carries an additional semantic presupposition $\pi$.

            \item Suppose that a cooperative speaker chooses to utter $\phi$
            instead of $\phi_{\pi}$ in a conversation.

            \item Since the speaker did not presuppose $\pi$, MP predicts that
            the speaker shall not be in a position to presuppose $\pi$, e.g.,
            because the speaker does not believe that $\pi$ holds.

        \end{itemize}

     \item This general reasoning process correctly accounts for
        \textit{anti-presuppositional effects} discussed aboce.

      \item Furthermore, if one assumes that the speaker is opinionated about
        the truth of $\phi_{\pi}$, this derives the anti-presuppositions
        initially discussed.

    \end{itemize}

    \subsection{MP-effects as a reflex of grammatical exhaustification}

\begin{itemize}

    \item It has long been observed that anti-presuppositions are very similar
    to classical cases of scalar implicatures; the use of an
    \textit{assertively} weaker sentence entails that the non-weaker alternatives
    are false.

    \item Furthermore, as observed by \citet{magri2009}, the use of an
    assertively weaker sentence is perceived as odd in contexts where this
    sentence is contextually equivalent to an assertively stronger alternative.

    \pex
    \(C ⊆ \ml{the prof gave the same grade to every student}\)\hfill\(\langle\text{some},\text{all}\rangle\)
    \a\ljudge{\#}The professor gave an A to some students.
    \a The professor gave an A to all students.
    \xe

    \pex~
    \(C ⊆ \ml{John has three children of the same gender}\)\hfill\(\langle\text{two},\text{three}\rangle\)
    \a\ljudge{\#}John has two sons.
    \a John has three sons.
    \xe

    \item \citet{magri2009} proposes that antipresuppositions are simply
    implicatures computed in the presuppositional dimension (he adopts a
    multi-dimensional approach to presupposition).

    \ex
    \(\eval{[\(\ml{exh}_{M}\) ϕ]} ≔ \displaystyle\frac{\ml{exh}_P (ℙ ϕ)}{\ml{exh}_A (𝔸 ϕ)}\)
    \xe

    \pex \textbf{Strengthened presupposition}
    \a \(\ml{excl}_{P} ϕ ≔ \set{ψ ∈ \ml{alt} ϕ|ℙ ϕ ↛ ℙ ψ}\)
    \a \(\ml{exh}_{P} ϕ ≔ ℙ ϕ ∧ ∀ ψ ∈ \ml{excl}_{P} ϕ[¬ ℙ ψ]\)
    \xe

    \pex \textbf{Strengthened assertion}
    \a \(\ml{excl}_{A} ϕ ≔ \set{ψ ∈ \ml{alt} ϕ|𝔸 ϕ ↛ 𝔸 ψ}\)
    \a \(\ml{exh}_{A} ϕ ≔ 𝔸 ϕ ∧ ∀ ψ ∈ \ml{excl}_{P} ϕ[¬ 𝔸 ψ]\)
    \xe

    \item In order to capture both anti-presuppositional effects and oddness
    effects in the domain of implicature, \citeauthor{magri2009} proposed two
    highly influential principles of implicature computation:

    \begin{itemize}

      \item \textit{Blindness}

      \item \textit{Mandatoriness}

    \end{itemize}

    \begin{tcolorbox}
      Blindness
      \tcblower
      The notion of entailment relevant for the computation of excludable
      alternatives is \textit{logical entailment} (rather than entailment
      relative to common knowledge).
    \end{tcolorbox}

    \begin{tcolorbox}
      Mandatoriness (Condition on Prunability)
      \tcblower
      Whenever an excludable alternative $ψ$ to a sentence $ϕ$ is
      contextually-equivalent to $ϕ$, the computation of the implicature
      associated with $ψ$ is mandatory.
    \end{tcolorbox}

\end{itemize}

    \subsection{Back to the weak theory of plurality}

    \begin{itemize}

        \item \citet{sauerland2003} claims that MP predicts the following
        contraint on the distribution of the singular vs. plural:

        \ex
        Use the most specific agreement feature possible whose presupposition is satisfied.
        \xe

        \item This captures the following contrast:

        \pex
        \a\ljudge{*}John are here.
        \a John is here.
        \xe

        \item We'll come back to how to derive the multiplicity inference for
        plural indefinites. In the meantime, both \citet{sauerland2003} and
        \citet{sauerlandAnderssenYatsushiro2005} present a raft of additional
        arguments for the weak theory of plurality, and
        \citeauthor{sauerland2003l}'s particular implementation.

        \end{itemize}

        \subsubsection{Coordinations and Pronouns}

        \begin{itemize}

            \item \citet{sauerland2003} argues that his proposal is
            independently necessary in order to account for environments in
            which number agreement is found.

            \pex
            \a Kai and Lina are playing.
            \a\ljudge{*}Kai and Lina is playing.
            \xe

            \begin{forest}
              [{...}
              [{$\phi$P}
                [{$\phi$\\\textsc{plur}}]
                [{andP}
                [{$\phi$P}
                  [{$\phi$\\\textsc{sing}}]
                  [{DP} [{Kai},roof]]
                ]
                [{and'}
                  [{and}]
                  [{$\phi$P}
                    [{$\phi$\\\textsc{sing}}]
                    [{DP} [{Lina},roof]]
                  ]
                ]
                ]
              ]
                [{...} [{are playing},roof]]
              ]
            \end{forest}

           \item An alternative analysis would be to say that conjunction is
            inherently plural (Sauerland cites Vanek 1977), but notes that this
            proposal faces a problem with examples such as the following.

            \pex
            \a Strawberries and cream is on the menu.
            \a Beans and rice is a basic staple around here.
            \xe

            \item Sauerland claims that singular agreement is possible just in
            case the denotation of the conjunction can be viewed as an atomic individual.

          \item Sauerland gives two more examples involving
            names:\footnote{\citet{winter_flexibility_2001} already discussed
            cases such as these in order to argue for a distinction between
            genuine pluralities and \enquote{impure atoms}. This is consistent
            with Sauerland's proposal.}

            \pex
            \a Kai and Lina makes a good combination.
            \a Tim and Sarah is a nice couple.
            \xe

        \end{itemize}

        \subsubsection{Pronouns}

        \begin{itemize}

            \item The second argument that Sauerland gives for his theory comes
            from pronouns.

            \item The structure Sauerland assumes for, e.g., (plural)
            \textit{they} is as follows:

            \ex
            \begin{forest}
              [{$\phi$P}
                [{$\phi$\\\textsc{plur}}]
                [{DP\\$i$}]
              ]
            \end{forest}
            \xe

          \item According to Sauerland's approach, the exclusivity inference
            associated with \textit{they} is derived via MP style reasoning,
            from competition with the singular pronouns.

            \item The prediction is, that is alternatives involving singular
            pronouns are independently unusable, then the exclusivity inference
            should no longer obtain.

            \item This accounts for uses of singular \textit{they} when the
            speaker is ignorant of the gender of the referent (note that
            Sauerland does not consider the possibility of singular
            \textit{they} used to refer to a non-binary individual -- this will
            complicate the analysis).

            \ex
            Some student left their umbrella.
            \xe

            \item Singular forms can also be blocked by politeness
            considerations, in which case the plural form is used. This can be
            observed in, e.g., German.

            \pex
            \a
            \begingl
            \gla Können Sie bitte etwas rücken!//
            \glb could they please a-little move//
            \glft \enquote{Could you please move a little} (formal)//
            \endgl
            \a
            \begingl
            \gla Eure Majestät haben euren Silberlöffel geschluckt//
            \glb you majesty have your silver spoon swallowed.//
            \glft \enquote{Our majesty has swallowed our silver spoon}//
            \endgl
            \xe

        \end{itemize}

        \subsubsection{Definites}

        \begin{itemize}

            \item One unusual feature of Sauerland's analysis is that number
            features are only interpreted at the DP level -- not on the noun itself.

            \item Sauerland explicitly argues for this on the basis of definites.

            \ex
            The students wrote a paper.
            \xe

          \item The \textbf{distributivity operator} $*$:

            \ex
            $* P ≔ λ X . ∃C[\ml{cover} C X ∧ ∀ x ∈ C[P x]]$
            \xe

            \ex~
            \(\ml{cover} C X = 1\text{ iff }⊕ C = X\)
            \xe

          \item The distributivity operator can apply to nominal and verbal
            predicates equally.

            \ex
            \(\eval{student} ≔ \set{\ml{Tom},\ml{Tina},\ml{Tanya}}\)
            \xe

            \ex~
            \(\eval{\(*\) student} = \Set{\begin{array}{c}
                \ml{Tom},\ml{Tina},\ml{Tanya}\\
                \ml{Tom}\oplus\ml{Tina},\ml{Tom}\oplus\ml{Tanya},\ml{Tina}\oplus\ml{Tanya}\\
                \ml{Tom}\oplus\ml{Tina}\oplus\ml{Tanya}
              \end{array}}\)
            \xe

            \ex~
            \(\eval{wrote a paper} = \Set{\begin{array}{c}
                                            \ml{Tom},\ml{Tanja},\\
                                            \ml{Tom}\oplus\ml{Tina}
                                            \end{array}}\)
            \xe

            \ex~
            \(\eval{\(*\) wrote a paper} = \Set{\begin{array}{c}
                                                  \ml{Tom},\ml{Tanya},\\
                                                  \ml{Tom}\oplus\ml{Tina},\ml{Tom}\oplus\ml{Tanya},\\
                                                  \ml{Tom}\oplus\ml{Tina}\oplus\ml{Tanya}
                                                  \end{array}}\)
            \xe

            \item According to the standard view (e.g.,
            \citealt{bennett1974,Link1983-LINTLA-2}, etc.) number on an NP is
            interpreted as $*$.

            \item According to such theories, what the definite article does is
            (a) presuppose that its complement has a unique maximal element $X$,
            and, if defined, returns $X$.

            \item The $*$-operator has no morphological reflex on the VP. The LF
            of \textit{the students wrote the paper} is given below. It is
            predicted, on the standard theory, to be defined and true in a
            scenario where \textit{Tom and Tanya wrote papers by themselves, and
            Tom and Tina collaborated on a paper}.

            \ex
            \begin{forest}
              [{$\ml{true}$}
              [{$\ml{Tom}\oplus\ml{Tina}\oplus\ml{Tanya}$}
                [{\(σ\)\\the}]
                [{$\Set{\begin{array}{c}
                \ml{Tom},\ml{Tina},\ml{Tanya}\\
                \ml{Tom}\oplus\ml{Tina}\oplus\ml{Tanya}
              \end{array}}$}
                  [{$*$}]
                  [{$\set{\ml{Tom},\ml{Tina},\ml{Tanya}}$\\student}]
                ]
              ]
              [{$\Set{\begin{array}{c}
                                                  \ml{Tom},\ml{Tanya},\\
                                                  \ml{Tom}\oplus\ml{Tina},\ml{Tom}\oplus\ml{Tanya},\\
                                                  \ml{Tom}\oplus\ml{Tina}\oplus\ml{Tanya}
                                                  \end{array}}$}
                [{$*$}]
                [{$\Set{\begin{array}{c}
                                            \ml{Tom},\ml{Tanja},\\
                                            \ml{Tom}\oplus\ml{Tina}
                                            \end{array}}$} [{wrote a paper},roof]]
              ]
              ]
            \end{forest}
            \xe

            \item The standard theory furthermore predicts that a singular noun
            can't be used in this scenario, since it has no unique maximum.

            \item Why can't \textit{The students wrote a paper} be used in a
            scenario where exactly one student wrote a paper?

            \item Nothing goes wrong
            semantically. The standard theory must appeal to pragmatic factors
            similar to the ones that Sauerland appeals to -- namely, don't use
            the plural if using the singular would result in an equivalent
            meaning.\footnote{Unlike MP however, arguably this follows directly
            from existing Gricean principles, such as \textit{brevity}.}

            \item Contrast now Sauerland's semantics for number, according to
            which $*$ can be applied to any predicate, and is \textit{never} pronounced.

            \item Sauerland observes that, in order to account for the cases
            below, the standard theory must say that plural marking is
            \textit{ambiguous} between $*$ and the cumulativity operator $**$ (\citealt{beck_cumulation_2000}).

            \pex
            \a The daughters of the defense players/Bill and James...
            \a The residents of these cities...
            \a The winners of a gold medal at the 1992 and 1996 olympics...
            \xe

            \pex~
            $** \ml{daughter} X Y$ iff there are both:\\
            \a a cover $C_{X}$ of $X$, s.t. $∀x∈C_{X}∃y ≤ Y[\ml{daughter} x y]$
            \a a cover $C_{Y}$ of $Y$, s.t. $∀y∈C_{Y}∃x ≤ X[\ml{daughter} x y]$
            \xe

            \pex~
            \a \(\ml{daughter} = \set{\langle\ml{DB},\ml{Bill}\rangle,\langle\ml{DJ},\ml{James}\rangle}\)
            \a \(** \ml{daughter} =
            \set{\langle\ml{DB},\ml{Bill}\rangle,\langle\ml{DJ},\ml{James}\rangle,\langle\ml{DB}\oplus\ml{DJ},\ml{Bill}\oplus\ml{James}\rangle}\)
            \xe

          \item Singular nouns allow cumulative readings too:

            \pex
            \a Every daughter of the defense players is watching the game.
            \a Every winner of a gold medal at these events can be proud.
            \a Every resident of these cities has a bicycle.
            \xe

         \item Sauerland concludes that plural marking cannot be consistently
            interpreted as either $*$ or $**$; the standard theory does not have
            an account of why a cumulated noun in a definite description must be
            plural.

            \item However, according to Sauerland's theory, this
            straightforwardly follows -- the number head above the DP must
            contain the [\textsc{plur}] feature, since [\textsc{sing}] is ruled
            out here.

            \ex
            \textsc{plur} the ([$**$ daughter] of Bill and James)
            \xe

        \end{itemize}

        \subsection{Evidence for the weak theory}

        \begin{itemize}

            \item The data in this section is from \citet{sauerlandAnderssenYatsushiro2005}.

        \end{itemize}

        \subsubsection{Mixed reference}

        \pex
        \a You are welcome to bring your children.
        \a\ljudge{\#}You are welcome to bring your child.
        \a You are welcome to bring your child or your children.
        \xe

        \pex
        \a Every boy should invite his sister to the party.
        \a\ljudge{\#}Every boy should invite his sister to the party.
        \xe

        \subsubsection{Indefinites in DE contexts}

        \begin{itemize}

        \item Number marking in a DE context doesn't affect truth-conditions:

        \pex
        \a Kai hasn't found any eggs.
        \a Kai has found no eggs.
        \xe

        \pex
        \a Without artificial ingredients.
        \a If John had eaten any apples from the basket, there would be at least
        one/\#two less in the basket.
        \xe

        \item In a UE context, number marking on an indefinite \textit{does}
            affect truth-conditions.

            \pex
            \a Some eggs are still hidden.
            \a Some egg is still hidden.
            \xe

            \pex
            Kai couldn't find certain/some eggs.
            \xe

          \item The analysis:

            \ex
            \begin{forest}
              [{...}
                [{...} [{some egg},roof]]
                [{...}
                  [{$λ x$}]
                  [{...}
                  [{$ϕ$P}
                    [{$ϕ$\\\textsc{sing}}]
                    [{$x$}]
                  ]
                    [{...} [{is still hidden},roof]]
                  ]
                ]
              ]
            \end{forest}
            \xe

           \item The presupposition introduced by \textsc{sing} is
            \textit{locally accommodated} in the nuclear scope of the GQ. 

            \item What about the plural feature?

            \ex
            \begin{forest}
              [{...}
                [{...} [{some egg},roof]]
                [{...}
                  [{$λ x$}]
                  [{...}
                  [{MP}]
                  [{...}
                  [{$ϕ$P}
                    [{$ϕ$\\\textsc{sing}}]
                    [{$x$}]
                  ]
                    [{...} [{is still hidden},roof]]
                  ]]
                ]
              ]
            \end{forest}
            \xe

            \begin{tcolorbox}
              Local \textit{Maximize Presupposition}
              \tcblower
              MP applies to the scope of an existential if this strengthens the
              entire utterance.
            \end{tcolorbox}

            \item Evidence for local maximise presupposition (based on
            \citealt{percus2006antipresuppositions}):

            \pex
            \a Every professor with exactly two students told \textbf{both} of his
            students to quit.
            \a\ljudge{\#}Every professor with exactly two students told \textbf{all} of
            his students to quit.
            \xe

          \item We see further parallels with scalar implicature:

            \pex
            \a After every one of the professor's students had arrived, the
            professor asked all of them to leave again.
            \a After every one of the professor's students had arrived, the
            professor asked some of them to leave again.\\
            (\# if the professor asked all the students to leave)
            \xe

        \end{itemize}





    % \item We can also implement the strong theory within
    % \citeauthor{sauerand2003}'s framework:

    % \ex
    % \(\eval{\textsc{plur}\(^{*}\)} ≔ λ x:♯x > 1 . x\)\hfill\(\type{e → e}\)
    % \xe

    % \ex
    % \(\eval{\textsc{sing}\(^{*}\)} ≔ \ml{id}\)\hfill\(\type{e → e}\)
    % \xe

    % \begin{forest}
    %   [{\(ϕ\)P}
    %     [{$ϕ$\\{[\textsc{sg}/\textsc{pl}]}}]
    %     [{DP}
    %       [{D\\the}]
    %       [{NP} [{boy$^{*}$},roof]]
    %     ]
    %   ]
    % \end{forest}


\section{\citeauthor{spector2007}'s puzzle}

\begin{itemize}

    \item \citeauthor{spector2007} is largely concerned with accounting for
    inferences associated with plural indefinites.

    \item We won't focus on all of the details of \citeauthor{spector2007}'s
    theory, but its flaws will be instructive.

    \item Like previous authors, \citeauthor{spector2007} observes that the
    exclusivity inference obtains in a UE environment, and disappears in a DE environment.

    \item But further to this, \citet{spector2007} observes that when a plural
    indefinite appears in the scope of some \textit{non-monotonic operator} such
    as \textit{exactly one of my students}, we find something unexpected.

\pex
\a\label{spector}Exactly one of my students has solved difficult problems.
\a Exactly one of my students has solved some difficult problems.
\xe

  \item (\ref{spector}) is equivalent to none of the following sentences:

    \pex
    \a\label{spectora}Exactly one of my students has solved at least one difficult problem.
    \a\label{spectorb}Exactly one of my students has solved two, or more than two, difficult problems.
    \xe

  \item (\ref{spectora}) is true (taking the scope of \textit{at least one} to be
    fixed below \textit{exactly one}) in a situation where one student, say
    Jack, has solved exactly one difficult problem, and no other student solved
    any problem.

  \item (\ref{spector}) however, suggests that the only student who has solved a
    problem has solved more than one problems.

  \item (\ref{spectorb}) on the other hand, is true where one student, say Jack,
    has solved exactly two problems, and all other students have solved exactly
    one problem; but intuitively, (\ref{spector}) entails that no student, apart
    from the unique one who solved problems, solved any problem.

  \item On the face of it then, (\ref{spector}) is a probelm for both strong AND
    weak theories of plurality.

  \item The attested interpretation is as follows:

    \ex
    One of my students has solved several difficult problems, and all other
    students have solved no difficult problems at all.
    \xe

  \item Spector's account: recursive exhaustification.

  \item Spector assumes a so-called \enquote{minimal worlds} formulation of the
    exhaustivity operator, defined below:

    \ex
    \(\ml{exh}_{mw} ϕ ≔ \set{w| ϕ w ∧ ¬∃w'[ϕ w' ∧ w' <_{S} w]}\)
    \xe

    \ex~
    \(w' <_{S} w\) iff the members of \(\ml{alt} ϕ\) true in \(w'\) are a proper
    subset of the members of \(\ml{alt} ϕ\) true in \(w\).
    \xe

    \item Let's apply this exhaustivity operator to a concrete case: the
    exclusivity inference associated with disjunction.

    \ex
    \(ϕ = p ∨ q\)
    \xe

    \ex~
    \(\ml{alt} ϕ = \set{p,q,p∧q,p∨q}\)
    \xe

  \item Let's gather up the members of \(\ml{alt} ϕ\) true in each world:

    \begin{itemize}

      \item \(w_{∅} = ∅\)
      \item \(w_{p¬q} = \set{p,p∨q}\)
      \item \(w_{¬pq} = \set{q,p∨q}\)
      \item \(w_{pq} = \set{p,q,p∨q,p∧q}\)

    \end{itemize}

  \item This gives us the following ordering on worlds.

    \[
    w_{∅} <_{S} w_{p¬q} ; w_{¬pq} <_{S} w_{p∧q}
    \]

  \item \(\ml{exh}_{mw} ϕ\) gathers up the set of \(\phi\)-worlds which make as
    few of the alternatives true as possible without contradicting the
    prejacent. In this scenario this delivers the following set:

    \[
    \set{w_{p¬q},w_{¬pq}} = p ∨ q ∧ ¬ (p ∧ q)
    \]

  \item Note that the minimal worlds formulation of the exhaustivity operator,
    like \(\ml{exh}_{ie}\) is \textit{contradiction free}, and consistent with
    considering the individual disjuncts as alternatives.

  \item In fact, Spector (2016) demonstrates that whenever the set of
    alternatives is semantically closed under conjunction, innocent exclusion
    and the minimal worlds formulation are \textit{equivalent}.

  \item \textbf{Assumptions regarding plurality}: Spector assumes that singular
    NPs denote sets of atoms, and plural marking is interpreted as
    \citeauthor{landman_events_2000}'s $*$-operator.

    \ex
    \(* P ≔ λ X . ∀x[(\ml{atom} x ∧ x ≤ X) → P x]\)
    \xe

    \item The singular, on the other hand, is assumed to be semantically
    vacuous.

  \item This directly accounts for the interpretation of indefinite plurals in
    DE contexts, but doesn't account for the exclusivity inference in UE
    contexts, or Spector's central puzzle.

  \item This proposal renders singular and plural indefinites
    \textit{equivalent} (at least for distributive predictates).

    \ex
    Jack saw a horse.\\
    \(∃ x ∈ \ml{horse}[\ml{j saw }x]\)
    \xe

    \ex~
    Jack saw horses.\\
    \(∃ X[∀x[(\ml{atom} x ∧ x ≤ X) → x ∈ \eval{horse}] ∧ \ml{j saw }X]\)
    \xe

    \item Since \(λ X . \ml{j saw }X\) is distributive, the following
    equivalence holds for any \(X\):

    \[\ml{j saw }X ⇔ ∀ x[(\ml{atom} x ∧ x ≤ X) → \ml{j saw }x]\]

    \item The puzzle then, is how to derive a multiplicitly inference for the
    plural via competition with the singular. Since the two are equivalent, this
    shouldn't be possible.

    \item Spector suggests that the plural doesn't compete with the literal
    meaning of the singular alternative, but rather the pragmatically
    strengthened meaning. The pragmatically strengthened meaning of the singular
    alternative is taken to convey an \textit{exactly one} inference.

 \end{itemize}

 \subsection{Spector's assumptions regarding alternatives}

 \begin{itemize}

   \item Spector makes the following assumptions regarding alternatives:

     \pex
     \a\label{sp1}Jack saw horses.
     \a\label{sp2}Jack saw a horse.
     \a\label{sp3}Jack saw several horses.
     \xe

     \pex
     \a \(\ml{alt (\ref{sp1})} = \set{\text{Jack saw horses, Jack saw a horse}}\)
     \a \(\ml{alt (\ref{sp2})} = \set{\text{Jack saw a horse, Jack saw horses,
     Jack saw several horses}}\)
     \a \(\ml{alt (\ref{sp3})} = \set{\text{Jack saw several horses}}\)
     \xe

     \begin{tcolorbox}
       Question
       \tcblower
       What assumptions is Spector making regarding the relation of
       \textit{alternativehood}? Can you spot anything problematic.
     \end{tcolorbox}

   \item We begin by computing the strengthened meaning of each sentence:

     \item The only alternative to the bare plural is the (equivalent) singular,
     which is of course not excludable, so exhaustification is vacuous.

     \ex
     \(\ml{exh}_{mw} (\text{Jack saw horses}) = \text{Jack saw horses}\)
     \xe

     \item The singular indefinite can be exhaustified relative to \textit{Jack
     saw several horses} -- the minimal set of worlds consistent with the
     prejacent that make as many of the alternatives false as possible, are the
     ones in which Jack saw exactly one horse.

     \ex
     \(\ml{exh}_{mw} (\text{Jack saw a horse}) = \text{Jack saw exactly one horse}\)
     \xe

     \item Exhaustification of (\ref{sp3}) is vacuous, since there are no
     non-equivalent alternatives.

     \item Now let's consider the alternatives to the exhaustification of the
     bare plural.

     \ex
     \label{spp}\(\ml{exh}_{mw} \text{Jack saw horses}\)
     \xe

     \ex
     \(\ml{alt (\ref{spp})} = \Set{\begin{aligned}[c]
         \ml{exh}_{mw} (\text{Jack saw horses})\\
         \ml{exh}_{mw} (\text{Jack saw a horse})\\
         \ml{exh}_{mw} (\text{Jack saw several horse})
       \end{aligned}} = \Set{\begin{aligned}[c]
         \text{Jack saw horses}\\
         \text{Jack saw exactly one horse}\\
         \text{Jack saw several horse}
       \end{aligned}}\)
     \xe

     \item What is set of worlds which makes as many of the exhaustified
     alternatives false, without contradicting the prejacent? The worlds in
     which Jack saw more than one horse, of course.

     \item Spector thereby derives the exclusivity inference.

\end{itemize}

\subsection{Back to Spector's puzzle}

\begin{itemize}

    \item Recall:

\ex
\label{puzzle}Exactly one of my students has solved difficult problems.
\xe

  \item The only alternative to (\ref{puzzle}) involves a singular indefinite:

    \ex
    \(\ml{alt (\ref{puzzle})} = \set{\text{Exactly one of my students solved a
    difficult problem}}\)
    \xe

 \item First, let's compute the strengthened meaning of the alternative. It
    competes with the following, which is logically non-weaker:

    \ex
   Exactly one of my students has solved several difficult problems.
    \xe

    \ex
    \(\begin{aligned}[t]
      &\ml{exh}_{mw} \text{Exactly one of my students has solved a difficult
        problem}\\
      &= \begin{aligned}[t]
        &\text{exactly one of my students has solved a difficult problem}\\
        &\wedge \neg (\text{exactly one of my students has solved several
          difficult problems})
        \end{aligned}\end{aligned}\)
    \xe

  \item This is equivalent to:

    \ex
    Exactly one of my students has solved exactly one difficult problem, and no
    other student has solved any problem at all.
    \xe

    \item We now compute the meaning of (\ref{puzzle}) relative to its
    strengthened alternative, just like last time. The result is:

    \ex\label{lessbrief}
    Exactly one of my students solved at least one difficult problem, and its
    not the case that exactly one of my students has solved exactly one
    difficult problem and no other student has solved any difficult problem at all.
    \xe

  \item This is equivalent to:

    \ex\label{brief}
    Exactly one of my students has solved at least two difficult problems, and
    all other students have solved no difficult problem at all.
    \xe

  \item Informal proof:

    \begin{itemize}

        \item  Suppose (\ref{lessbrief}) is true: then there is a student, say
        Jack, who solved a difficult problem, and is such that all other
        students have solved no difficult problem.

      \item Suppose Jack solved exactly one difficult problem.

        \item Since the following is false -- \textit{exactly one student has
        solved exactly one difficult problem, and no other students has solved
        any difficult problem} -- it follows that a student distinct from Jack
        has solved exactly one difficult problem, or Jack is the only one who
        has solved exactly one difficult problem but there are other students
        who have solved difficult problems.

      \item In both cases, there must be students different from Jack who have
        solved a difficult problem. This is a contradiction.

      \item Therefore, Jack has solved several difficult problems; no other
        student solved any difficult problem -- it follows that a student
        distinct from Jack has solved exactly one difficult problem, or Jack is
        the only one who has solved exactly one difficult problem but there are
        other students who have solved difficult problems.

      \item In both cases, there must be students different from Jack who have
        solved a difficult problem. This is a contradiction.

      \item Therefore, Jack has solved several difficult problem; no other
        student solved any difficult problem.

     \end{itemize}

\end{itemize}

\subsection{Towards an account of Spector's puzzle}

\begin{itemize}

    \begin{tcolorbox}
    \item Note that the results reported here are the result of ongoing joint
    work with Paul Marty.
    \end{tcolorbox}

  \item Recall:

    \pex
    Exactly one of my students has solved difficult problems.\label{elliott1}
    \a one of my students has solved \textit{more than one} difficult problem\label{elliott1a}
    \a none of my other students have solved \textit{one or more} difficult problems\label{elliott1b}
    \xe

    \item \textbf{Ingredient 1:} following \citet{sauerland2013}, we treat
    \textit{exactly} as a focus-sensitive operator.

    \item \text{Exactly}, much like, e.g., \textit{only}, takes a proposition
    $p$ that contains a focused element (such as a numeral), and returns (i)
    that $p$ is true, and (ii) for every $q ∈ \ml{alt} p$ that is not entailed
    by $p$, $¬ q$ is true.

    \pex
    Exactly/only [\textsc{one}$_{F}$ students came to the meetings].
    \a One student came to the meeting.
    \a \(¬\) [$n$ students came to the meeting], for any numeral $n > 1$
    \xe

    \item \textbf{Ingredient 2:} a number of researchers have noted that, in the
    scope of \textit{only}:

    \begin{itemize}

        \item implicatures \textit{are} generated in the upward-entailing
        (UE) component (i.e., the prejacent).

        \item implicatures \textit{disappear} in the
    downward-entailing (DE) component (i.e., the negated alternatives).

        \end{itemize}

    \item See, e.g., \citet{gajewski2012defense,alxatib2014,bar2018free}.
    \citeauthor{bar2018free} describes this as a \enquote{two-sided inference}.

    \item We illustrate this phenomena here with a scalar expression
    \textit{some} in the scope of \textit{exactly/only}:

    \pex
    Exactly/Only [\textsc{one}$_F$ student ate some of the cookies]\label{builder3}
    \a \ul{UE component: implicature}\\
    one student ate some \textit{but not all} of the cookies
    \a \ul{DE component: no implicature}\\
    $\neg$[$n$ students ate some of the cookies], for any numeral \(n > \textit{one}\)
    \xe

    \item We propose that the case in (\ref{elliott1}) is another instance of
    the above phenomenon.

    \item \textbf{Ingredient 3:} following \citet{mayr2015}, we assume
    \textit{predicate-level exhaustification}.

    \ex
    \(\ml{exh} f ≔ λ X . f X ∧ ∀ g ∈ \ml{alt} f[f \not\subseteq g → ¬ g X]\)
    \xe

    \item a multiplicity implicature is generated in the UE-prejacent of
    \textit{exactly}, delivering (\ref{elliott1a}), but not in its
    DE-alternatives, hence (\ref{elliott1b}).

    \item The intuition here is that \textsf{exh} can be rendered vacuous in
    these DE-alternatives as its working would otherwise weaken their meaning,
    (\ref{builder4}).

    \item This should ultimately follow from the Economy condition constraining
    the distribution of \textsf{exh} (see \citealt{foxSpector2018} and others).

    \pex
    Exactly [\textsc{one}$_F$ student solved \textsf{exh} [difficult problems]]\label{builder4}
    \a one student solved \textsf{exh} [difficult problems]\\
    $\Rightarrow$ one student solved \textit{more than one} difficult problems
    \a $\neg$[$n$ student solved \sout{\textsf{exh}} [difficult problems]], for any numeral \(n > \textit{one}\)\\
    $\Rightarrow$ none of the other students have solved \textit{one or more} difficult problems
    \xe

\end{itemize}

\printbibliography

\end{document}

%%% Local Variables:
%%% mode: latex
%%% TeX-master: t
%%% End:
