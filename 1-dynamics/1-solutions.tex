\documentclass[nols,twoside,nofonts,nobib,nohyper]{tufte-handout}

\usepackage{fixltx2e}
\usepackage{tikz-cd}
\usepackage{tcolorbox}
\usepackage{appendix}
\usepackage{listings}
\lstset{language=TeX,
       frame=single,
       basicstyle=\ttfamily,
       captionpos=b,
       tabsize=4,
  }

\input{acronyms}
\renewcommand*{\acsfont}[1]{\textsc{#1}}

\usepackage[font=footnotesize]{caption}

\makeatletter
% Paragraph indentation and separation for normal text
\renewcommand{\@tufte@reset@par}{%
  \setlength{\RaggedRightParindent}{0pt}%
  \setlength{\JustifyingParindent}{0pt}%
  \setlength{\parindent}{0pt}%
  \setlength{\parskip}{\baselineskip}%
}
\@tufte@reset@par

% Paragraph indentation and separation for marginal text
\renewcommand{\@tufte@margin@par}{%
  \setlength{\RaggedRightParindent}{0pt}%
  \setlength{\JustifyingParindent}{0pt}%
  \setlength{\parindent}{0pt}%
  \setlength{\parskip}{\baselineskip}%
}
\makeatother

\usepackage{multicol}

\setcounter{secnumdepth}{3}

% Set up the spacing using fontspec features
\renewcommand\allcapsspacing[1]{{\addfontfeature{LetterSpace=15}#1}}
\renewcommand\smallcapsspacing[1]{{\addfontfeature{LetterSpace=10}#1}}

\usepackage{amsthm}

\theoremstyle{definition}
\newtheorem{definition}{Definition}[section]

\title{The dynamic turn I:\\
Update semantics}

\author[Patrick D. Elliott \& Danny Fox]{Patrick~D. Elliott \& Danny Fox}

\addbibresource[location=remote]{/home/patrl/repos/bibliography/master.bib}

\lingset{
  belowexskip=-1\baselineskip,
  aboveglftskip=0pt,
  belowglpreambleskip=0pt,
  belowpreambleskip=0pt,
  interpartskip=0pt,
  extraglskip=0pt,
  Everyex={\parskip=0pt}
}

\usepackage{float}


% \usepackage{booktabs} % book-quality tables
% \usepackage{units}    % non-stacked fractions and better unit spacing
% \usepackage{lipsum}   % filler text
% \usepackage{fancyvrb} % extended verbatim environments
%   \fvset{fontsize=\normalsize}% default font size for fancy-verbatim environments

% % Standardize command font styles and environments
% \newcommand{\doccmd}[1]{\texttt{\textbackslash#1}}% command name -- adds backslash automatically
% \newcommand{\docopt}[1]{\ensuremath{\langle}\textrm{\textit{#1}}\ensuremath{\rangle}}% optional command argument
% \newcommand{\docarg}[1]{\textrm{\textit{#1}}}% (required) command argument
% \newcommand{\docenv}[1]{\textsf{#1}}% environment name
% \newcommand{\docpkg}[1]{\texttt{#1}}% package name
% \newcommand{\doccls}[1]{\texttt{#1}}% document class name
% \newcommand{\docclsopt}[1]{\texttt{#1}}% document class option name
% \newenvironment{docspec}{\begin{quote}\noindent}{\end{quote}}% command specification environment

\begin{document}

\maketitle% this prints the handout title, author, and date

\section{Problem set solutions}

  \subsection{Dynamic semantics and classical equivalence}

  In today's handout, we stated a dynamic semantics for negated sentences, conjunctive and disjunctive sentences, as well as material implications.

  In fact, the only primitives we need are a dynamic semantics for negated and conjunctive sentences. We can define disjunction and material implication via classical equivalence, but not just any classical equivalences will do.

  \begin{tcolorbox}
    \textbf{Exercise}
    \tcblower

    \textbf{Part 1:} Informally prove the following equivalences:
    \begin{itemize}
        \item $c[ϕ ∨ ψ] ≡ c[¬ (¬ ϕ ∧ ¬ ψ)]$
        \item $c[ϕ → ψ] ≡ c[¬ (ϕ ∧ ¬ ψ)]$
    \end{itemize}

    \textbf{Part 2:} Provide formulas using only conjunction and negation that are classically equivalent to $⌜ϕ ∨ ψ⌝$, $⌜ϕ→ψ⌝$, which nevertheless aren't equivalent in propositional dynamic semantics. Demonstrate where the equivalence breaks down.

    \textbf{Part 3:} Comment briefly on what this tells us about the explanatory potential of propositional dynamic semantics.
  \end{tcolorbox}

  \subsection{Staticization}

  As noted in today's handout, we can \textit{staticize} a (bivalent) propositional update semantics by taking the \textit{proposition expressed by $p$} to be $W[p]$, i.e., the logical space updated with $p$. $\eval*{p} ≔ W[p]$.

  \begin{tcolorbox}
    \textbf{Exercise}
    \tcblower
    \textbf{Part 1:} Prove whether the following equivalences (an informal demonstration is fine).

  $$
  \begin{aligned}[l]
    &\eval*{p}  &= I(p)\\
    &\eval*{¬ ϕ} &= W - \eval*{ϕ}\\
    &\eval*{\phi ∧ \psi}  &= \eval*{\phi} ∩ \eval*{\psi}\\
    &\eval*{\phi ∨ \psi} &= \eval*{\phi} ∪ \eval*{\psi}\\
    &\eval*{\phi → \psi} &= \eval*{\phi} ⊆ \eval*{\psi}\\
  \end{aligned}
  $$

  \textbf{Part 2:} What is the staticization of a modalized sentence $⌜◇ ϕ⌝$? Comment on the significance of the result.
  \end{tcolorbox}

  $$
  \begin{aligned}[t]
    &\eval*{p}\\
    &= W[p]\\
    &= W ∩ I(p)\\
    &= I(p)&(I(p) ⊂ W, ∀p)
    \end{aligned}
  $$

  $$
  \begin{aligned}[t]
    &\eval*{¬ ϕ}\\
    &= W[¬ ϕ]\\
    &= W - W[ϕ]\\
    &= W - \eval*{ϕ}
    \end{aligned}
  $$

  $$
  \begin{aligned}[t]
    &\eval*{ϕ ∧ ψ}\\
    &= W[ϕ ∧ ψ]\\
    &= W[ϕ][ψ]
    &= \eval{ϕ}[ψ]
    \end{aligned}
  $$

  \subsection{Backwards connectives}

  \begin{tcolorbox}
    \textbf{Exercise}
    \tcblower
    \textbf{Part 1:} give a dynamic semantics for \enquote{backwards disjunction} $⊻$, which (i) captures the classical contribution of disjunction (demonstrate this via staticization), and (ii) predicts the presupposition of the first disjunct to be satisfied in the following (provided without judgement).

    \enquote{Either Sarah's corgi is sleepy, or Sarah has no corgi.}

    Is this prediction good, in the general case? Feel free to use raw data from whichever language(s) you speak in the discussion here.

    \textbf{Part 2:} do the same thing for \eqnuote{backwards implication}, $←$, with respect to the following sentence:

    \enquote{If Sarah's corgi is sleepy, then Sarah has a corgi.}
  \end{tcolorbox}

  \subsection{\textit{Must}}

\begin{tcolorbox}
  \textbf{Exercise}
  \tcblower
  Can we state the meaning of epistemic \textit{must} ($□$) as the dual of Veltmann's $◇$?
  \begin{itemize}
    \item If so, demonstrate that this delivers intuitively correct results.
    \item If not, show why not.
  \end{itemize}

\end{tcolorbox}

 \textbf{Answer}

  Yes, we can define \textit{must} as the dual of Veltmann's \textit{might}.

  \begin{definition}{Epistemic \textit{must} in update semantics}
        $$c[□ ϕ] ≔ c[¬ ◇ ¬ ϕ]$$
  \end{definition}

  Epistemic \textit{must} has a test semantics. It does a tentative update of $c$ with $¬ ϕ$, if the result is \textit{not} the absurd state, then the test fails, and we get the absurd state. Otherwise, the test succeeds, and we get back $c$. This is shown below:

  $$\begin{aligned}[t]
    c[¬ ◇ ¬ ϕ] &= c - c[◇ ¬ ϕ]\\
      &= c - \begin{cases}
    c&c[¬ ϕ] ≠ ∅\\
    ∅&\text{otherwise}
    \end{cases}\\
&= \begin{cases}
    \emptyset & (c - c[ϕ]) ≠ ∅\\
    c&\text{otherwise}
  \end{cases}
\end{aligned}
  $$

\printbibliography

\end{document}
