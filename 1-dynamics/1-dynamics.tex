\documentclass[nols,twoside,nofonts,nobib,nohyper]{tufte-handout}

\usepackage{fixltx2e}
\usepackage{tikz-cd}
\usepackage{tcolorbox}
\usepackage{appendix}
\usepackage{listings}
\lstset{language=TeX,
       frame=single,
       basicstyle=\ttfamily,
       captionpos=b,
       tabsize=4,
  }

\input{acronyms}
\renewcommand*{\acsfont}[1]{\textsc{#1}}

\usepackage[font=footnotesize]{caption}

\makeatletter
% Paragraph indentation and separation for normal text
\renewcommand{\@tufte@reset@par}{%
  \setlength{\RaggedRightParindent}{0pt}%
  \setlength{\JustifyingParindent}{0pt}%
  \setlength{\parindent}{0pt}%
  \setlength{\parskip}{\baselineskip}%
}
\@tufte@reset@par

% Paragraph indentation and separation for marginal text
\renewcommand{\@tufte@margin@par}{%
  \setlength{\RaggedRightParindent}{0pt}%
  \setlength{\JustifyingParindent}{0pt}%
  \setlength{\parindent}{0pt}%
  \setlength{\parskip}{\baselineskip}%
}
\makeatother

\usepackage{multicol}

\setcounter{secnumdepth}{3}

% Set up the spacing using fontspec features
\renewcommand\allcapsspacing[1]{{\addfontfeature{LetterSpace=15}#1}}
\renewcommand\smallcapsspacing[1]{{\addfontfeature{LetterSpace=10}#1}}

\title{The dynamic turn I\marginnote{24.954: Pragmatics in Linguistic Theory}}

\author[Patrick D. Elliott \& Danny Fox]{Patrick~D. Elliott \& Danny Fox}

\addbibresource[location=remote]{/home/patrl/repos/bibliography/master.bib}

\lingset{
  belowexskip=0pt,
  aboveglftskip=0pt,
  belowglpreambleskip=0pt,
  belowpreambleskip=0pt,
  interpartskip=0pt,
  extraglskip=0pt,
  Everyex={\parskip=0pt}
}

\usepackage{float}


% \usepackage{booktabs} % book-quality tables
% \usepackage{units}    % non-stacked fractions and better unit spacing
% \usepackage{lipsum}   % filler text
% \usepackage{fancyvrb} % extended verbatim environments
%   \fvset{fontsize=\normalsize}% default font size for fancy-verbatim environments

% % Standardize command font styles and environments
% \newcommand{\doccmd}[1]{\texttt{\textbackslash#1}}% command name -- adds backslash automatically
% \newcommand{\docopt}[1]{\ensuremath{\langle}\textrm{\textit{#1}}\ensuremath{\rangle}}% optional command argument
% \newcommand{\docarg}[1]{\textrm{\textit{#1}}}% (required) command argument
% \newcommand{\docenv}[1]{\textsf{#1}}% environment name
% \newcommand{\docpkg}[1]{\texttt{#1}}% package name
% \newcommand{\doccls}[1]{\texttt{#1}}% document class name
% \newcommand{\docclsopt}[1]{\texttt{#1}}% document class option name
% \newenvironment{docspec}{\begin{quote}\noindent}{\end{quote}}% command specification environment

\begin{document}

\maketitle% this prints the handout title, author, and date

\section{Recap: Stalnakerian pragmatics}

\subsection{Trivalence}

A sentential meaning is a function $p:W ↦ \set{1,0,\#}$. Here's a simple example:

\ex
$\eval{Sarah's corgi is sleepy} = \begin{cases}
  1&\textsf{Sarah has a corgi} \& \textsf{Sarah's corgi is sleepy}\\
  0&\textsf{Sarah has a corgi} \& \textsf{Sarah's corgi isn't sleepy}\\
  \#&\textsf{otherwise}
  \end{cases}$
\xe

In trivalent semantics, the \textit{semantic presupposition of a sentence $S$} is the set of worlds $w$, such that $\eval{S} w$ is either true or false.

\ex
The semantic presupposition of $S$ (def.),\\
$S^{π} ≔ \set{w | \eval*{S} w = 1 ∨ \eval*{S} w = 0}$
\xe

\subsection{Update and Stalnaker's bridge}

The \textit{update} induced by a a sentence $S$, written as $c[S]$ is a partial function $u:W ↦ W$.

\ex Stalnakerian update (def.)\\
$c[S] ≔ \begin{cases}
  \set{w | w ∈ c ∧ \eval*{S} w}&c ⊆ S^{π}\\
  \textsf{undefined}&\textsf{otherwise}
  \end{cases}$
\xe

A (bivalent) proposition $p$ is \textit{redundant} wrt a context set $c$ if $c ⊆ \set{w | p w}$.

Stalnaker's bridge places a precondition on an update of $c$ by $S$ --- $S^{\pi}$ must be \textit{redundant} wrt $C$.

\subsection{Successive update}

A (trivial?) observation: updating $c$ with a sentence $S$ can make the presupposition of a sentence $S'$ redundant, thus ensuring that $c[S']$ is guaranteed to be defined.

\ex
Sarah has a corgi. Sarah's corgi is sleepy.
\xe

Stalnakerian pragmatics directly captures this, since successive assertion gives rise to a successive update.

We can write a successive update of $c$ with $S$ followed by $S'$ as $c[S][S']$.

\ex
$c[S][S'] ≔ (c[S])[S']$
\xe

$$
c[\text{Sarah has a corgi}] = \overbrace{\set{w | w ∈ c ∧ \ml{Sarah has a corgi in }w}}^{c'}
$$

$$
c'[\text{Sarah's corgi is sleepy}] = \begin{cases}
  \set{w | w ∈ c' ∧ \ml{Sarah's corgi is sleep in }w}&\begin{aligned}[t]
    &c ∩ \set{w | \ml{Sarah has a corgi in }w}\\
    &⊆ \set{w | \ml{Sarah has a corgi in }w}
  \end{aligned}\\
  \ml{undefined}&\ml{otherwise}
  \end{cases}
$$

$$
c'[\text{Sarah's corgi is sleepy}] =
  \set{w | w ∈ c' ∧ \ml{Sarah's corgi is sleep in }w}
$$

\subsection{Towards an update semantics}

Successive assertion patterns with \textit{conjunction} wrt presupposition projection (Danny's handout from last week; Karttunen's generalization).

A natural way of cashing this out: a conjunctive sentence induces successive update.

\ex Conjunctive sentences in update semantics (def.)\\
$c[S\text{ and }S'] ≔ c[S][S']$
\xe

What kind of rule is this? It looks very much like a \textit{construction-specific} update rule.

\printbibliography


\end{document}
